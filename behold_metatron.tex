\documentclass[11pt,twoside,draft]{memoir}

\usepackage{salitter}
\usletterlayout

% ---- pkgs
\usepackage{mwe}
\usepackage{csquotes}
\antiquafont

% -- lib
\newcommand{\cA}{$\mathcal{A}$}
\newcommand{\cB}{$\mathcal{B}$}
\newcommand{\cC}{$\mathcal{C}$}
\newcommand{\cD}{$\mathcal{D}$}

\newcommand{\titlePageTitle}[1]{ 
{ \Large #1 }}

\newcommand{\titlePageAuthor}[1]{ 
{ \large #1 }}

\newcommand{\ts}[1]{
	\textsuperscript{#1}}

\renewcommand\thesection{\arabic{section}}

\begin{document}
\hyphenation{a-long-side ma-ni-che-an me-chan-el-ec-tron-is-tic me-ta-phor-ic pa-ra-me-mo-ries phy-sics trans-na-tion-al u-ni-ty wi-thin}
\frontmatter
\nofolios

% -- the amazon description for 'behold metatron', which is baffling and amazing
% { \begin{quotation} Brooklyn Sol Yurick \& Anticipation of the Night A fire sale at Bear Stearns. Markets in turmoil. Sub-prime disasters. IRAs and TDAs drained overnight. Crises of modern times but forseen, at least in its potentiality and broadest strokes, by Brooklyn author Sol Yurick. I remember reading \booktitle{Behold Metatron, the Recording Angel}, an essay by Brooklyn author Sol Yurick, published by Autonomedia (Foreign Agents Press) back in 1985. I think I picked up my copy in Park Slope's Community Books, back in the day. Yurick also is the author of \booktitle{The Warriors}, made into a classic, apocalyptic gang film, as well as the excellent novels \booktitle{Fertig}, \booktitle{The Bag}, \booktitle{Someone Just Like You}, \booktitle{Richard A.} and \booktitle{Confession,} and other articles and essays. \booktitle{Behold Metatron} is heavy stuff, relentlessly visionary, the material problem seen through a lens of advanced capitalism and electronic philosophy. Picture \journaltitle{Wired Magazine} crossed with \journaltitle{Fortune Magazine} but edited by William Blake. Metaphysics, economics, art and intellect of an high order, coalescing into an interpretation of an emerging electronic universe. Forget Al Gore\footnote{In a March 9, 1999, interview with CNN's Late Edition with Wolf Blitzer, Al Gore was discussing his history as a senator who extensively supported internet development. A sentence said by Gore in the interview, \enquote{I took the initiative in creating the Internet,} became a widespread meme around Y2K. The joke was that Gore claimed to have invented the internet. --- Salitter Workings Editor}, perhaps Mr. Yurick conceptualized, if not anticipated, the Internet, globalization, the flow of information and data across galaxies of cable and wireless realms, sometimes directed, sometimes chaotic, but always having impact. Mr. Yurick \enquote{\ldots\ The old philosopher's stone could convert base metals into gold. Now humans, real estate, social relations are converted into electronic signs carried in an electronic plasma. the dream of magical control has never been exorcised. Perhaps, after all, modern capitalism is a great factory for the production of angels.} In 1988, the journal \journaltitle{Social Text} published Mr. Yurick's \booktitle{The Destiny Algorithm} which appeared to further mine the cybernetic\slash human nexus. Globalization and the 'net got its philosophic underprinnings where else, Brooklyn NY. \end{quotation} \signoff{--- the Amazon.com description for \booktitle{Metatron}\footnote{sic} by Sol Yurick.} }

% \clearpage

% -- title page
{
	\raggedright
\titlePageTitle{Behold Metatron, the Recording Angel} \linebreak
\titlePageAuthor{Sol Yurick} \par
}

\clearpage 


% -- colophon
{
FOREIGN AGENTS SERIES
Semiotext(e)

© 1985 Sol Yurick \\
All rights reserved. \\
Printed in the United States of America. \\
Publication funded in part by the New York \\
State Council on the Arts. \\

Semiotext(e), Inc.\\

522 Philosophy Hall \\
Columbia University \\
New York City, N.Y. 10027 U.S.A. 
}

\clearpage

% -- dedications, etc
{
	To Michiko Sawada, for conversations, friendship, analysis, and most of all companionship and support.

\plainbreak{2}

To suppose that any author of fiction, poetry or discursive writing works alone is a conceit fostered by the Nineteenth Century. Of course one always has collaborators, advice, and even opposition. Here are some of the people who helped me: Bert Cowlan, Martin Elton, Karen Paulsell, Hesh Wiener, Peter Krass, John Verity, Joan Greenbaum, Peter Brooks, Robert Shapiro, Ken Donow, Jean McDermott, Herb Schiller, Oscar Urgeteche, George Caffentzis, Silvia Federici, Robert Greenblatt. Of course, they might not agree with my ideas or the way I used their help. I alone am responsible for what has resulted. \emph{Metatron} is also in partial fulfillment of a Guggenheim Foundation grant.
}

\clearpage

\mainmatter
\chapterstyle{section}
\openany
\pagestyle{simple}

\chapter{}

We are in the middle of that Great Transformation into what is called the Information Age, or Post-Industrial Society. As in all Grand Transitions, \emph{fin du siecle's} and climacterics, perceptions of reality are once again being redistorted by the insertion of a vast new mediational system into an already multiplexed, historically accreted maze of mediations. In the context of this forced march, the relationship of information to society and nature has to be rethought.

Call information capital-intensive knowedge, a mechanelectronistic metaphor made to dominate more and more of life. All knowledge is in the process of being converted to computer-compatibility. The old philosopher's stone could convert base metals into gold. Now humans, real estate, social relations \ldots\ are converted into electronic signs carried in an electronic plasma. This would merely be an amusing game if people (in fact only a small subset of the world's population: 90\% of all information processing is controlled by a small part of the \enquote{developed} world) weren't being forced to use and live \emph{through} information processing and communications technology. Call it Informatics; call it telematics.

The components of telematics are mainframes, minis and personal computers, cathode ray tubes, printers, copiers, automated bankteller machines, point-of-sale sensors, antennae, copper and fiber optic wire, copiers, remote-sensing devices, robots (remotely run or otherwise), calculators, integrated chips, software, mass-data-storages, tapes, discs, diagnostic equipment, a babble of \enquote{appropriate} languages, telephones, modems, telexes, terminals, microwave relays, radio, cable, satellites, switching and routing systems\ldots\ Alongside of this, one has to consider the social communication systems and all the transcieving and routing operations there. Even the simplest of conversations are seperated, reconfigured, sent and priced. And those who live in this new world are losing their grip on an older reality. As for those who have no access to, no participation in, this newly imposed world, they are out of the world's new information economy, doomed to obsolescence and death.

A glorious, transcendant and radiant future is promised us. Efficiency will increase, productivity will rise, the office and factory of the future will be automated, we will be able to work at home, teleconference, we will have hoards of instantly retrievable knowledge at our disposal, record-keeping will be easier, we will be freed from work and the burdens of memory. That, or an enormous disaster is in the making as parts of the world become metaphysical. For it's time for Demiurge II. The Year 2,000 is coming. Apocalypse and creation in one.

Whole nations, their economies, their peoples, their resources, their land, can be simulated and displayed on some electronic input/output device. But worse, taken for the real thing. National boundaries become porous and erode. America is no more as trans\-national data-flows penetrate borders. Nations become illusions as foreign enterprises buy pieces of many lands. The informational process has concrete results. True, this is nothing new. International cartels, merchants in past ages accomplished the same thing. But as long as any enterprise becomes translated more and more into its essences --- money and near-money, an all-purpose information, the blood and hormones of business --- those essences cannot be held in containers called nations any longer.

The technology can be likened to a nervous system, one external to humans yet connected to their internal nervous systems by a variety of devices, becoming more fused, joined. For example, with the onset of medical data-bases, monitoring, diagnostic and treatment machines, ancient dreams of being directly connected, the world \enquote{wired} to the brain-nerve complex, leads to the hope that thought alone will move reality.

With the invention of new sensing devices, new perceptual systems come on line. All beings are some function of their information intake, no matter how indirectly the information is recieved. What was done in the mind must now be done through computers \ldots\ programs begin to become quasisolidified thought. New procedures for action and behavior take the form of a ritual, requiring the playing of an excruciating game called programming. People resist? The languages are too hard, the steps to long and complicated? Money is now poured into developing computers that \enquote{talk English,} are touch-responsive or voice-activated. Computers for dummies.

But above all, price is attached to these mediational meditations. Price is a seasonally adjusted, value-added medium in this invented medium, a carrier of values standing for the signs of things sent along a carrier wave. The computer, and its languages, represent a frozen and hard-wired \emph{habituation of thought.} The programs are a way of trying to introduce flexibility, variety and reference into the relative intractability of the machine. However, by itself, and with its operators, and its languages, it is impossible to truly metaphorize --- an essence of human brain activity and thought --- that is to say, fuse into one homgeneity any two or more disparate sensation-terms.

Each \enquote{new age} rewrites the history of the past (while thinking it has discarded the obsolescent past). The last great age of reinvention and rationalization of past and future took place, more or less, from the 15th to the 19th centuries. New world views were created. But does the process of rethinking and reorganizing the past really free any age from that past? Has modern rationalization taken a secret rider, an incubus along in its intellectual and institutional baggage?

New institutions advertise themselves, using the old images of domination to promote the transition. They draw their sales-imagery out of a central bank of symbolic forms. Knowledge of the past is simplified. Epochs are erased (perhaps there was too much that was embarressing in the past). New pasts, whole aeons are invented. Complex existence is simplified, and then recomplexified in another way. Forgetfullness follows. Scramble and resequence; but, in the process of borrowing symbolic energy from the past, new simultaneities and odd juxtapositions, like dreams, emerge.

To look up, to see the stars, the galaxies (in their past and glowing glories) with new kinds of lenses is to have recourse to addresses in data banks where long runs (projected in a short time) of computer-modeled, cosmological statistics are stored (with certain assumptions built in). Look closely at these computer-simulated, eons-long histories of distant stellar objects projected on the cathode-ray tube. Watch them appear to recede. What are we \enquote{seeing}? Are the simulations guided by an underlying compulsion to \emph{aesthetics,} and does this become the ultimate gravitational lens? And those great galactic streamers of stars, and the great gouts of gas jetting off into the blackness\ldots\ how like the monetary jet-streams that banks draw off into the black holes of their balance-sheets from once luminous nations, entropizing and then rematerializing as investments elsewhere. Transubstantiation? Is that what underlies the very concept of the preservation of matter and energy?

Celestial bookkeeping? But see the flaw; the images are seen in squared-off pixels, reconstructions based on a relatively few observations, structured by certain recurring theories. All observational technology is, within limits, the concretization of a speculation. And what we see is all based on some initializing, mythic event: The Beginning.

Troubles in paradise. While trumpeting the imminent emergence of the grand, unifying theory, the unifying theories fall apart. Fundamental forces and particles proliferate. The original central dogma of genetics is riddled with heresies. And even forms of credit go off and multiply. They become desperate to unify and simplify (an ancient compulsion). Unification also implies structuring, measuring, concentration, monopolization, a center, a central intelligence, heirarchies of knowledge, a control room. However, without general acceptance, credibility, faith in new forms of knowledge, these become mere scholastic games. They turn away from observation to their animated projections, assuming them to have been the fruits of experiment, since they cannot journey to the heart of a star, visit a black hole, or distant galaxies, except in imagination. Nor will they be able to make journeys to putative planets without a complete transformation of the body. But they must journey on: after all, their reputations and belief systems, their funding, is at stake. Scientists seem to have reached the end of the line. They have decided that the observing Mind plays an \emph{active instrumental role} in the cosmos (indeterminacy), perhaps once played by god or demiurge.

Modern observational machinery resurrects ancient epistomological problems and incorporates them into itself. Ether, having once failed as a concept, is in the process of being reinvented. Information is the ultimate mediational ether. Light doesn't travel through space; it is information that travels through information as information\ldots\ at a heavy price. The scientists, reacting, are now on their knees, abjectly populating the cosmological and sub-atomic realms with \enquote{god,} \enquote{purpose,} \enquote{design,} \enquote{progress,} \enquote{ascent,} \enquote{transcendance,} "cosmic frequency dances"\ldots\ And if quantum physics tells us that observation intervenes in the observed, and becomes part of it, this also holds true for pure theories of finance, or for that matter, evolutionary genetics, at least according to the sociobiologists. The atom and credit court secrecy. Higher and higher levels of indeterminacy fill every aspect of life.

Scientists cannot seem to live with the possibility that there might be anamolies in the universe. For instance, that: 
\begin{enumerate}
\item this might be the only planet in the universe with life on it; 
\item the universe is infinite and unbounded and, what's more, its contents are not uniformly distributed;
\item that there was, consequently, no Big Bang; 
\item there are no black holes; 
\item the universe neither contracts nor expands\ldots\ it was always this way and always will be; 
\item there are no quarks or pre-quarks, other than as a function of the activity in reactors\ldots
\end{enumerate}
After all, cosmological evidence is a sparse series of stats, stills taken in a short period of time, arranged into an aesthetic, evolutionary, dynamic sequence. Like the ancient mystics, the scientists are projecting themselves into a space they cannot hope to reach, at least in human form. They are colonizing the void with a concept. Like the world's telebanking system with its computer-assisted bankers, they are now inflating space like some financial bubble. Indeed, there is now an inflation model of the universe\ldots\ finance similized into space. Well, if all is one, why not?

Almost every day, new arrays of satellites are hurled into the air: switchboards in the sky, hovering in geosynchronous orbit, remitting messages, insensitive to earthly distances. Other satellites circle the earth rapidly, surveying weather, land, minerals, people. New lines are laid down; optical fiberglass replaces wire (mourn for the slaves of the old copper mines of Cyprus, or for those who died in Chile for Kennecott and think a while of hands cut off for Union Miniere in the old days in the Congo). Every year new computers come on line and are re-interconnected as manufacturers try to configure incompatible computer architectures and languages, requiring the manufacture of new networking devices, plug-compatible hard and software as new models render old ones obsolete.

Data bases for every conceivable aspect of
life are created. However to gather data, to
even think data, requires a vast language
project. For every item must be distilled,
rendered and specified before the computer
can handle it; the old, fuzzier specifications
must be translated. Given the costs of translating 
old knowledge (the price of intellectuals 
and their thought) into new forms (what
room for ambiguity is there?), costs of storage,
heavy doses of electricity to run and
chill the machine, much must be abandoned.
Paper libraries, for instance (of course, promising
the paperless society, more paper than
ever is generated). Certain central markets
(like the New York Stock Exchange, which
depends on personal interaction, 
much secret knowledge and \enquote{tribal} networks) die
hard and slow. Stocks and commodities, the
securities markets, banking, currency, options, 
futures... all these markets must now
be rethought and restructured. Banks become
stock brokers. Brokerage houses sell
insurance. Shopping centers sell securities.
Where once precious metals were carried,
where once paper was exchanged, new electronic
signs and signals verify and celebrate
the exchanges. And if the age of the counterfeit
cosmos has come, it is also the age of
easily counterfeitable currency.

Informational essences become more real
than tangible humans. The very body itself
begins to evanesce, just as in those folk tales
where the shaman's body-parts were scattered 
to the winds and reassembled. Medicine
promises to be delivered from a distance.
Experimentors consult case histories
on tapes or discs thousands of miles away.
Once-free knowledge becomes priced. At
the same time the electromagnetic spectrum
for transmission becomes used up, gets
scarce, high-priced.

As for ordinary life, we are exhorted to
become compatible with all this. We are
urged to fill our homes with personal computers;
to become computer-literate; to write,
learn, do office work; get medical examinations 
and treatment at a distance; do accounting, 
banking, law; buy and sell equity;
shop with computers; do mathematics (without
understanding), logic, all with computers; 
do factory work with robots; confer, and
even eat with computer assistance. A vast
proliferation of books are produced to explain
how to use our computers, to rectify
the outpouring of incomprehensible manuals
written by and for technicians. We play
games, draw story elements out of storages
and arrange new entertainments with the
computer; we have them watch our fuel,
shop, have the lights turned on or off for us,
and even in time, fall in love with far-distant
strangers, perhaps even mate with the gods...
as the ancients were reputed to have
done.

People seated at their terminals forget
time, sitting mesmerized, their fingers jerking 
spastically at the keys, their eyes blearing.
In time, so the sales pitch goes, computers 
will achieve artificial intelligence (and
perhaps even use \enquote{biochips} and so live)
and what's more, they will be more rational,
organize knowledge more neatly than our
poor brain can, once certain problems of
miniaturization, heat, switching speeds, and
the development of sophisticated, \enquote{humanlike,} 
or artificial intelligence languages
have been solved. Perhaps then they will be
able to attain that kind of \enquote{randomness} and
\enquote{intuitive} leap humans make so easily without 
having to scan and compare lists. Having
been talked into surrendering our spirit, our
knowledge, our bodies will become useless.
We will, like Jesus, like Faust, like Dante,
achieve immortality and \enquote{evolve} into computer-compatible 
and re-programmed history, one with Babylon, Nineveh, Rome.
Our essences will be preserved in that great
memory bank in the sky.

But in the meantime, on the peasant land
(what's left of it), in the jungles (what's left
of those), in the world's ghettos (which
proliferate), in the poisoned seas, rivers,
and lakes, the contaminated land, sky and
earth, a lot of humans must be phased out.
Prices decline for technology, but overall,
costs rise because of mundane deregulatory
decisions, questions of intellectual property
(the pricing of ideas), ironic anti-trust decisions (AT\&T, IBM), national information-conflict policies, the classified (or priestly, if
you want to look at it that way) intelligence
approach to knowledge... all contribute to
the expansion of ignorance. Knowledge purveyors block the sun's rays and the rain's
fall, offering to sell sunlight and rainfall...
as signs. Where once you might look up and
see the clouds passing by, you can --- but you
don't have the eyes for it --- look up and see
the spy satellites, the earth/weather/sea/resources sensing satellites hovering, or count
the streams of invisible electronic gold flowing by. Perhaps
you can sense the meta-weather, almost as natural as monsoons. For
a deluge of money --- inflation --- is a \enquote{natural} disaster, creating floods, leaving some
lands sodden and others a desert.

% 2
\chapter{}

How did this development come to be?
Surely more forces were at work than \enquote{Progress?} 
This essay is not a history of the
information
revolution, but some
mention
must be made in passing. At some point
during the Second
World
War, a series of
decisions to computerize were reached. The
overriding concerns were military and intelligence applications. It should be noted that
private industry would never have invested
in this, or any other development. Without
government investment, bankers are paragons of timidity.

The founders of the information, or cybernetic age, were Alan Turing, John Von
Neuman, Norbert Wiener, Claude Shannon
and later, Noam Chomsky. Hordes of electrical engineers --- whether they understood
what they were doing or not --- reworked almost every philosophical problem known to
humans in terms of circuitry and programming languages. These problems began, of
course, centuries ago. For instance the epistimological question: what is knowledge,
how do we know, how do we know we
know, how does it relate to the world outside, who controls knowledge, who has it
and who does not, what is it worth, how do
we talk about it (which is the question of
what language we shall use and how shall
we talk), and what instrumentalities we perceive through.

Questions of the technology of knowing
must be interwoven with political and economic
considerations (within the confines
of what is scientifically and technologically
possible), which is to say knowledge systems are structured like intelligence and
counter-intelligence systems. There is to be
written a whole history of secret and coded
knowledges... priestly systems, rites, hierarchies and ceremonies of learning and
passage, memory systems, networks of initiates. . . In addition, one should ask: why did
one set of systems triumph—that is to say,
why were they preserved, and rrmembered—
and others fail? There is room for a history of
the politics of the promotion, funding and
triumph of intellectual knowledge systems
and this includes the rememberance of the
major streams of philosophy. Philosophy is
one of the atmospheric backgrounds which
provides for a general and unified state of
perception against which day to day knowledge is learned.

The original choices for computers, binary, Boolean (Leibnizian, as Wiener would
have it) logic, reflected a dialectical, even a
Manichean approach and was an unfortunate decision. Why these choices? It was
easier to design electrical circuits that could
carry out the logic operations.

The system began slowly, went on line
massively with mainframes and minis in the
fifties, mostly
in defense
and intelligence
applications, followed closely by banking
and business.

In the seventies, a massive campaign was
mounted to \enquote{democratize} the computer.
The micro was developed by small, innovative businessmen-technicians.
Sales propaganda was disseminated
in the name of enlightenment, efficiency, transcendance
and power. Every possible sales technique
known to public relations, advertising and
mythology was employed to sell the computer.
Not only were ancient and modern
symbols deployed, but also fear. It became
possible, we were told, to have a computer in
the home that was once as large as a building
... and did the same work.

One notes the parallel developments and
\enquote{needs}: The committment to the Great
Theater of perpetual war as the pressure
system out of which innovation and invention and progress came. This generated a
need for a vast corps of mind-workers. Cheap
education produced intellectuals. This led
not only to the further proliferation of mindworkers, but of mediators and mediational
systems. Intelligence and police (and their
surveillance systems); psychologists and
their theories; many schools of psychotherapy; sociologists; anthropologists; analysts;
coders and decoders; cryptographers and decryption experts; 
disinformation-propagating operatives; advertisers; public-relations
flacks; consultants; historians in fifty modes;
economists, both practical and theoretical;
financial manipulators, and the buyers of
their services (bankers, securities dealers,
brokers, currency dialecticians); new critics;
hermeneuticists; structuralists; semioticians;
deconstructionists; quantifiers; metricians;
statisticians; propagandists; accountants and
auditors; lawyers and proliferators of law;
interactivists (and their connecting machineries); cosmic and microcosmic theoreticians;
agronomists; doctors; philosophical logicians
and inventors of yet newer and newer mathematics; salesmen; priests and ministers and
inventors of yet-new religions; logical and
scientific astrologers... And now, in the
present age, all this to be machined.

They sought both unity and fragmentation. Now one must admit that there is a
propensity in some humans to generate new
unifying theories and technologies while
at the same time inventing and proliferating new explanatory systems and new subtheories . . . all of which promise to explain
everything. This seems to be a function of
the density of intellectuals, in terms of availability of jobs and competition, both relative
and absolute, to a general and non-theorizing
population. This insures that a fair percentage of those theories will be nonsensical, if
not fraudulent... which is no impediment to
their triumph.

In addition, general systems theory took
hold, and every aspect of the universe was
designated a sub-system of some larger system and the largest --- and unknown --- system
of all was a function of these bureaucratically
minded spinners of holisms.

The early cyberneticians thought that this
development would add to—if not exponentially, then at least incrementally—the sum of
human knowledge. Accompanying this development was an ancient agenda: the compulsion to impose order, predictability, to
eliminate risk and uncertainty. But as far as
this ancient agenda was concerned, the commitment should be shared, paid for by some
part of the public. New processes would in
turn create still newer knowledge. And, as all
things happen in this modern society, the
\enquote{system,}
with
all of its attendant
confusions, complexities and corruptions, with its
intense conflicts among the different programs, systems and equipment 
manufacturers, with its political and business battles, has
been laid on in the most haphazard, ridiculous, expensive, inefficient and disorganized
way (repeating our earlier history of canals,
railroads, highways, transit systems, communications and technology in general). We
now have a conflict of computer, communicating and language-conversion systems
with many fundamental problems still unsolved.


(And here, lest we forget that the problem
is not merely \enquote{intellectual,} we must remember concrete institutions with which
intellectuals are connected,
and who provide their funding. How, and to whom, ideas
are sold: we must think about AT\&T, Sperry-Rand, IT\&T, IBM, Citicorp and Chase... We
must also not forget that there are unwritten
and true histories to be done of the Department of Defense, the National Security Agency, the CIA, all intelligence agencies of the
world, and how the intellectual thought of
these agencies permeates every aspect of
everyday life. We must think about the politics of international and national communications policy and how these issues are
fought out in corporations, legislative bodies
and regulatory agencies. We must think of
pricing, advertising, marketing, promotion,
generations of faulty computers, paper computers, imbecilic
competiton, suppression of innovation, influence-peddling, lobbying,
bribes, kickbacks and the rest of the common
paraphenalia of business ... especially at a
time when business becomes ever-more
\enquote{intellectualized.})

There was a nescessity to translate all
living and non-living forms, to simulate
events and natural processes, to chart their
interactions and simulate thse interrelations
and to begin to fill the memory and data
banks. This growing assemblage gradually
becomes the total environment ... at least
for a few. These developments are new but
are also, at the same time, the fulfillment of
an ancient desire: to control the material
world by the manipulation of secret know]ledge (secret, in modern times, by being
priced, being made into intellectual property, being classified). How does this differ
from the practices of ancient priests, shamans, magicians?

Ancient magicians thought they could
control the environment. How did information 
control the material world in the past?
By assuming a connection between the 
internal system of intellectual order and the 
\emph{external} system of \emph{material} order. One 
controlled the cosmos by the uses of resonances
and dissonances, rhythms compatible with
the true natural rhythm of the spheres, by the
use of a chant, an incantation, a dance, a 
ritual; or one could apply sacred geometry,
controlling shapes that were analogous to
the shape of the worlds one wanted to dominate\ldots\ magic. Magic embodies a primitive
theory of electromagnetism and telecommunication. 
Magic desires to achieve telepathy
and teleportation. Voodoo, for instance, contains 
the notion of a communicating medium
and the communicants who believe in it.
The Catholic Church is a communicating
organism with an apparatus of switches and
relays and a communicating language for
the input of prayers through a churchly
switchboard up to Heaven, and outputs returned 
to the supplicant. And above all, all
ancient and primitive systems implicitly
propose the notion of an ordered, coherant
universe, expressible in a certain set of languages,
the manipulation of which manipulates 
the universe. The question is: do these
systems manipulate the universe or a simulation 
of the universe? What certain intellectuals 
in modern society propose is electromagic.

\chapter{}

Beginning, perhaps, in the 17\ts{th} century, a
few had embarked on a program of "modernizing' society; shattering old categories
and languages while inventing new ones.
Leibniz, for instance, dreamed of a logical\slash mathematical-based universal language. One
of the great agendas of the 18\ts{th} and 19\ts{th}
centuries was a vast program of reclassification. 
There was also an attempt to trace back
all modern languages to a primal Indo-European tongue. Past and present humans,
societies, languages, plants and animals were
arranged on a progressive scale (and this was
a continuation of the Renaissance, which
had introjected the past, ancient Greece, into
its program of liberation from medieval
thinking). New theories and new disciplines
emerged: economics, politics, psychology,
sociology, history, the physical sciences,
mythology, anthropology\ldots\ all split off
from philosophy. These new disciplines began to atomize and reconstruct, emphasizing quantification. They were partial fictions
and suffered all the difficulties of translation; each developed their own jargons, hard
and soft tools, aesthetics, formal modes of
organizing the perception of the world,
creating new mediating lenses between humans, and between humans and the natural
world. In time, each one of these disciplines
claimed to be a total world view \ldots\ as did
each mitotic sub-discipline. General systems
theory and interdisciplinary studies began
to emerge in the early twentieth century.
Now, in the tail end of the 20\ts{th} century, are
remelted into the general category of information and communication theory.

The information age required a vast new
enterprise: an enormous translation or conversion project; a reduction of all disciplines
into a kind of symbolic, quantified representation --- a new universal language which
would translate the languages, dialects and jargons of all languages and disciplines ---
appropriate to the basic circuit logics in the computers. Bit by bit the differences between
disciplines and disparate bodies of knowledge (as well as living and non-living bodies
considered as language) are becoming eroded. This endeavor implied a perhaps fictional
notion; that the universe and everything in it is logico-mathematical. It also implied
that all things and forces in the universe could be treated as a cryptogram, a code, a
text that could be \emph{read}, sooner or later. Another and muted implication was that all
things in the universe were in some sense \emph{perceptually} simultaneous.

The general computer-compatible\slash general systems-schema runs something like this:
\begin{enumerate}
	\item Anything (or anyone) that can be exactly specified can be automated.
	\item Inferential, judgemental, learned or adaptive behavior can be specified (which raises the problem of translation or conversion of knowledge to information).
	\item Intuitional and creative activity can be indistinguishably simulated by machine (the drive for artificial intelligence).
	\item All this can be communicated from machine to machine, for the speeds of transmission means that messages are distance-insensitive (relatively speaking).
	\item Which means that one has to deal with complexes of social sets and the way in which they, or the information they have, or that represents them (not the same thing) can be communicated.
	\item Information is passed among (or taken from, or imposed on) the sets (but they also frequently resist this passage or appropriation of knowledge about themselves: this implies hierarchy of information systems).
	\item The forces which produce stability inside these social sets create instability among the sets.
	\item From the point of view of the general systematizers, an improvement between and among all social sets (and the way they interpret themselves and the world\ldots\ or the way in which they are interpretable) leads to a better management of the metasystem's information.
\end{enumerate}

But from whose point of view?

By the \enquote{social set} we mean a population
which has a language, a mode of discourse
and a set of customs (by which the language
it uses is processed) existing in a variety of
domains or environments, using sets and
subsets of natural and artificial languages;
bureacracies, corporations, secret societies,
individuals, professional societies, classes
(in the social sense), ethnicities and races,
disciplines, nations, regions, hierarchies\ldots\ and so forth\ldots\ in whatever ways society has
been split, conceptually and actually. These,
of course, overlap. It is apparent that for all
these groupings, the means for universal
discourse hasn't been invented yet and
what's more, many resist translation actively.
All the propositions point directly at the
problem of translation, or the generation of a
universal language.

Systems-building has gone on since the beginning of the appearance of humans.
Even the most \enquote{primitive} of groupings builds all-encompassing (and complexly
muddled) systems. Underlying this newest global climacteric, this vast re-writing program,
was a not particularly new set of assumptions: that any set of things, events,
forces linking people and events could be represented by some language, or set of
languages, logics, numbers, letters, symbols, signs\ldots\ That there is an ultimate and fundamental language, a deep structure in the
universe \ldots\ and that it is mathematicological and is discoverable and translatable\ldots\ These representations could be
linked in several ways: language to language and language to the world represented by
these languages ... into interactive and mobile structures that in some way match,
dance in time to the underlying and fundamental language of the universe (automated
natural language translation is a disaster). When things and people move, the signs
representing their existence are communicated to this informational technosphere.
Conversely, when signs, symbols, language elements, variables of all sorts are moved,
people, things, whole economies, the universe and all that is in it, should move. This
manipulator's dream is possible only if information is connected to the universe in
some concrete way, requiring sensors, languages, translators, categories and levers.

The sensors (eyes, ears, skin, writers of books, typists, telescopes, microscopes, 
electronic sensors of all kinds ... and so forth)
must \enquote{read}, transmit and input these signs of movement into some kind of storage
where language could work on them (meaning the incredible complex of miniscule and
high speed movement in the circuits, in and out of the various logical devices and timers
and storages ... ). Contrariwise, a set of language-motivated output levers could,
theoretically, energize and change the configuration of the universe. (In quantum,
operationally-oriented physics, this interventionary notion, that mere thought and its
instrumentalities affect the universe---in yet unmeasurable ways---is implicit. 
By extension, mere thought affects the universe, but in as yet unmeasurable ways.) This desire
reflects an ancient obsession; the Archimedean dream of minimal expenditures of energy 
moving great masses, for example shifting the great nebula in Andromeda into a
better orbit.

All of these desires occasions the search for the universal system-langauge which is,
at the same time, the \emph{real} language of the universe, the ultimate \enquote{machine code.}
A recourse to what can be considered gnostic wisdom, Pythagoreanism, or Kaballism:
these are used as key words to exemplify a way of thinking. Pythagoreanism was both a
mathematical and a magical system. Number translates into space and converely; all is
number and geometry. Kaballism and gnosticism are fundamentally literary. Cartesian
thinking carried this obsession further, turning space into a vast, suburban real-estate
development. Kaballa views the universe as \enquote{word} (although \enquote{word} translates into
number games: \emph{Gematria}). Considered from the perspective of these ancient magical
systems: gnosticism, hermeticism, the religion of the Jains, the I-Ching, Rosicrucianism,
alchemy and astrology, all the material universe is translatable. But this is to throw a
net of language out into the universe, and is the precursor, perhaps, to quantum physics
and operational indeterminacy.

None of this denies the need for the creation of language but points toward a recurrant obsession with language as the ultimate reality. What drives this obsession?

The hunt for the ultimate, sacred, or secular, usable, transmissible knowledge or information is like a vision of, a penetration to a sacred realm where total, instantaneous,
universal and all-purpose code, exists. This kind of thinking assumes an underlying,
unified universe. To match this universe, somewhere, somehow,
there exists, and is decodeable (if only in visions and dreams)
an underlying language, an ultimate metalanguage, a deep structure of grammar, a
boss language of all boss languages to match that reality. 
And that meta-language is basically mathematical, logical, rational. 
\enquote{In The Beginning was The Word and The Word was made Flesh.}

(In anticipation, let's propose several such languages:
the language of genetics, the language of quantum-relativistic 
particle physics, the language of finance, the language of
mathematical logic, the language of literature \ldots
which includes the psychoanalytical disciplines. There are more.)

However out of this uniform ur-language has come Babel. That is to say pure Word,
pure light (ultimate information), being made into Flesh, yielded corruption, decay,
dialects, death, a plethora of languages. Or, from the evolutionary geneticist's point of
view, diversity, uniqueness, adaptability to material conditions, non-repeatability \ldots
\emph{quality}. Diversity is the way to disorder, chaos, entropy, 
a confusion of languages. Specialized knowledges divide into languages,
sub-languages, jargons (even putting it this way assumes primal unity);
specializations fragment futher the possible wholeness into cultures, sexes, nations,
races\ldots Truly unlike languages have unlike assumptions behind them;
they cannot translate. This arises out of observing the bewildering array of languages, proliferated
into mutually incomprehensible dialects, a diversity of natural languages,
in a short period of time. But worse, the presence of,
the possibility of incompatible languages intimates that the underlying 
\emph{physical} universe is not one. Again, chaos. These \enquote{mutations} 
puzzle and anger the unifiers. They take steps to correct the process of decay and
dissolution. (As the first Rockefeller put it:
\enquote{Combination is the wave of the future,} in more ways than one.)

If there is a fundamental oneness in the
universe---particles seen one way, waves
another---and all things, people and events,
all singular events are manifested out of this
fundament, then all change and process and
people are mere aspects, illusion, perhaps
developed over time. This \emph{ur}ness should
have its own language, shouldn't it? Why
this diversity? Or perhaps this \emph{ur}-language
existed before the beginning of time? What
happened to fragment pre-temporal paradise
(which will become balanced at the other
end by post-temporal paradise), this primal
oneness, this \emph{quality}, leading to the 
shattering of Eden into cosmic Babel? The Fall?
Sin? Imbalance?

It is to offset the effects of a modern Babel that languages are constantly converted into
one another, hopefully without losing anything. Work into COBOL; COBOL into Aramaic;
Hebrew into digital \ldots and so forth.
The universe and all that is in it is assumed to contain a secret code or cryptogram; the
new language project, this drive toward fusion is designed to unlock the code (or
perhaps its purpose is to \emph{invent} the code).
Strings of biochemicals, DNA and RNA, are
called a code or cryptogram. The code expresses and replicates to duplicate itself
(with mutational and combinatorial variations) and becomes a body which eats food,
converts it into energy to give growth to
another body-shell in order to perpetuate a
code, a language, a message. The body as a
message transmitter.

The wisdom of the east handles this problem another way: it announces that diversity
is an illusion. The west, at present, holds that the clue to the ultimate bottom language is
supposed to lie in the human mind: in some way this primal unity can be \emph{remembered}.

(Remember Lucifer's tale before Eden was built? Consider the role of knowledge in the
tale of Eden. There is a struggle over the possession of information and thus a fight to
control the sacred language, or Lucifer's tale before Eden was built. Satan becomes the
Lord of Diversity, the Lord of disinformation, disunity, chaos, entropy: Prometheus
becomes the lord of stolen knowledge.)

Word-obsessed Kaballistic or Gnostic lore anticipates the Big Bang. Many religions
anticipate the Fall. The Big Bang leads to entropy: entropy leads to diversity. Humans,
we are told, are themselves the creators of
negentropy (holding the center together conceptually, if not physically) 
in the expanding universe, and so they invent unification and then start on the project\ldots The
mental act of unification has evolved into a technological and informational endeavour.
Diffusion is death; unification is eternal life. 

The division of knowledge into disciplines to pre-conceptually \enquote{observe} society was
problematical to begin with. What assumptions were brought to this task? The newest
synthesis raises new problems without solving the old ones. In order to achieve this
translation, one should look at some underlying assumptions of western, perhaps human, 
thought. One fundamental tenet of this kind of thought is that one can take wholes,
break them down into fundamental units
and rebuild all up from those units, providing the structural operations, the \enquote{grammar}
to string the units, complexes of units, into
whole, new languages. In this newest approach, wholes are broken down into a language of irreducible particles (which are
easy to account for, and match up to units
that are either measureable and countable,
modeleable, mappable, comparable: specifiable) and are built up again.

Now, structuralist thought (which allows
one to reorganize the positioning and sequencing of any text and relate it in new
ways to any other text \ldots an exercise in
simultaneity) and semiotics begins to treat
life---including literary and media artifacts---
like a complex cryptogram, a treasure, always
oblique, to be disinterred. It should be noted,
however, that while this can be done with
artifacts, with fictions, with records of the
dead, it cannot be done in real life.

One of the first rules of this game of
interdictions is that almost nothing is allowed to mean what it first seemed to be.
Novelists, priests, poets, mythmakers, magicians, have practiced this combinatorial and
sequencing, this matching-up and conversionary pythagoreanism for centuries. Novelists, tied to certain traditions, were permitted
to only see a limited set of realities and not
others. In much the same way poverty is
invisible in the board room, suffering is not a
category to be found in an annual report.

But the new constructs in the present
contain an accretion from the past (a sort of
memory) which is then used to rewrite and
reconsider the past. (The act of \emph{primal} creation---and its time, or the \enquote{beginning} of
time, and timing---didn't happen \emph{then} but is
reinvented again and again, and happens \emph{now}, just as history is rewritten again and
again to justify the present in order to assert that all events could only lead up to the
inevitable present.)

The newest instance of breakdown and
buildup has led to several crises: atomic
theory is in trouble. The breakdowns threaten
to become endless. \enquote{Fundamental} particles
proliferate; gravitons and chronons are invented 
alongside quarks which require prequarks. Every sub-atomic particle must be
specified and recorded, creating firestorms of indeterminacy between all boundaries of
thought (and reality), which allows for certain 
excesses of the imagination, the possibility of new transformations, shapeshifting
and chimeras, creating arcane juxtapositions
in the life of things, operations which once

belonged to the realm of dreams. Unit-quantum thinking contains a history of obsessional perception since Democritus: the universe particularized ... wholes fragmented
into quanta... This difficulty is apparent in psychosocial, statistical disciplines which
study groups and individuals... atoms and wholes. And yet, at the same time, the 
universe is percieved as a unified and contiguous whole in which the most distant parts
affect one another \ldots sooner or later.

Another complex contradiction to be considered: all could be viewed as the agglomeration of force-fields, electromagnetic waves, gravitational waves, frequencies, which,
when they reach some critical density, change into some other \enquote{quality.} (Or geometries,
numbers, values, dimensions, symbols, images, gods, spirits, phantoms, cash, talk,
drawings, dances \ldots ) One can look at humans as\slash of\slash in these fields in many ways,
from many angles, through a variety of disciplinary lenses. Humans, for instance, could
be considered as manifestations of the cosmological/astrophysical (and astrological)
whole ... brothers, sisters, spawn of the stars;
as biological manifestations concreting out of Fourier processes, complex waves
inventing complexes of waves in order to explain the self. If so, then medicine based
on physics, chemistry, molecular biology
should, in time, be replaced by electromagnetic wave therapy ... which is what voodoo is based on.

When the universe is waved, or when a
universal language is discovered or invented,
the boundaries between objects and objects,
and between objects and languages blur. As
boundaries were blurred, the disciplineseperated currents of the past dissolved. It
became desireable to create the logical links
that united the now-unbounded contents of
once artificially seperated areas of thought.
Indeed, the newest developments in computer thought demand this unity... but in a 
special way. Language domains can, in
principle, be interpolated into any form of
discourse, past and present, spatially seperated, including literary and psychological
discourse (we consider those psychologies
that don't take account of the nervous system to be literary; word games). Now a
re-reading and re-critique of all \enquote{great traditions} becomes possible. But, as in all translation, much has to be left out, cast aside as
irrelevant or dangeous dross, basically untranslatable (or not desireable to translate).
New problems of classification are raised,
for one cannot say \enquote{OM} and have it stand
for the \emph{All}.

The long and corrective project that some
humans are in the process of inventing and
reinventing leads to reunification and reconcentration. In the legend, humans are
created to replace fallen angels, but they
must go through aeons of ‘‘development."'
Through a long-range process, this concept
mutates into \enquote{evolution.} All of history is a
trip toward ascendance and transfiguration,
or transsubstantiation: in modern times this
reads as a recombinant genetics project for
the manufacture of immortal angels. (And in
the Golem myth, information placed into
dust brings the dust to life.)

How is this fragmentation of languages, of
civilization, of energy to be cured? Perhaps
by creating the appropriate thesauri, slide
rules, categories, classes, conversion matrices
for comparative mapping of realm onto
realm. A mental act should make it possible
to describe n-dimensional hyperspaces in
which the languages of, say, poetry, finance,
or relativity theory are seen to be one. Recreate a post-ur-language at one of the timeends of the universe? New vision? Not
really. Kaballa anticipates these alternate
spaces as language-manipulation. Primitive
religions describe journeys in folded and
short-circuited spaces which can be matched
up to the hoard-spaces and passages through
which bankers hurl their money around.

If fragmentation is the way to death, then
the parts of the person can still be united.
How? Resurrection: by the parts being conjoined (keeping track of them, memorializing the whole and its subsequent parts
in a file), \emph{communicating} in a magical
medium... If not a mystical medium, then
possibly an \emph{externalized} nervous system, a
great, artificial brain. Who does the connection? In the old days, the witch, the shaman,
the priest, The Church, the observor, the
remembrancer, the tale-teller. The information establishment plays the modern role.
Modern medical information and telecommunication
systems may memorialize representations of whole, or parts of people in
different and seperate data repositories even
in different and distant countries (since the
speed of communication is distance-insensitive, and time-insensitive, relatively speaking), ready to join the parts together in the
twinkling of an eye. By maintaining communications and preserving a joining-together algorithm, one can create a modern
version of the Resurrection through wire or
wave transmission, reuniting the parts. The
presence maintained after death. Ghosts, of a
sort.

Let's consider a practical problem: the operation of the multi-purpose computer in
the defense sectors: the Situation-Room.
There are various defense and intelligence situation-rooms, and presumably the ultimate and best one in the White House. The
purpose of the situation room is to recieve data from all over the world, so that a response to a political or military problem is
instantaneously possible. It is a vast intelligence input/output, monitoring device,  % TODO fix
presumably in a form understandable to nonexperts. (After all, for the crisis-data to arrive
in machine language, or any other primal computer languages, would be meaningless.)
Information arrives from all over the world. Given a change of any variable in the 
political, military, economic orders, creates
changes in the computer: it's a sort of electronic spread-sheet.

In order for this to work a number of conditions must be met. Various sensing
systems must be recieving and sending data at all times: a world network of National
Security Agency stations monitoring all electronic traffic, decoding and interpreting
it, the Central Intelligence Agency recieving a stream of reports, Internal Revenue
System, Federal Bureau of Investigation, Political
analysis, economic reportage, inputs
from the civilian sector \ldots and so forth.

This incredible influx of data must not only be translated, compressed and graded
in terms of importance, and matched up to
already existant data (to determine significant change), but must be arranged according
to some overriding set of scenarios for ready response\ldots scenarios into which go various
modes of analysis which had, at some point or other, to be automated, must match up, in
computer languages, to those original systems which generated the scenarios in the
first place. That is to say, military, economic, agricultural,
trade, political, sociological, psychological, anthropological, even medical data, country studies, all, at one time or
another, have to be translated into machine-readable forms. And, as much as possible,
this influx must be in real time. Of course
when one considers the variables, the incredible proliferation of disciplines and
their attendant languages which developed
before the computer arrived, we see that
what constitutes actionable facts are hard to
deal with and are imperfectly specifiable,
translatable or programmable without enormous distortions.

But another consideration is more practical and that has to do with the question of
whether or not individuals, classes, social
sets share, or don't share their data. While on
the one hand there is a striving for grand unification,
at the same time centrifugal forces --- competition,
protective secrecy, a proliferation of sub-disciplines --- work in
the opposite direction. After all, to take one example, given an age of high taxation,
many groupings and individuals have a vested interest in concealing their data.

But, in practical terms, when projected
production of oil and its consumption do not
conform to electronic wishes and statistical
projections of energy companies, then the
instrumental Archimedean levers to correct
this deplorable situation becomes the Marines, the CIA, the torturers, moving in to fit
a preconcieved notion of immediate long-term gain and growth.

Of course, beneath all of this there are
larger agendas, whole world-views which
are, by nature, metaphysical. Views of human
nature (psychological theories), views of nature itself, progressive --- military-assisted ---
Hegelian-Calvinistic destiny.

\chapter{}

The limitations of the new information
languages, the limitations of the machines,
storages, operating systems, circuitry, machine-compatible logic, programs diminish
what was once far richer. The old words
were broader; they packed complexes of
implication within them; their ambiguities
allowed for richness and latitude, for rethinking, redefinition from time to time.
They contain treasuries of implication within them; the \enquote{amounts} of information they
contain are staggering. Consider Eliot's \booktitle{The Wasteland}:

How do you represent, in terms of specification, and thus bits of information (if that
is even the way to put it) the endless galaxy of implications contained in Eliot's poem?
In the first place the poem is a rhythmic index, a memory-system. The allusive and
apparently self-contained word or phrase opens up into other poems, and histories.
That is to say they are references to memory
storages. The first line reads \textquote{April is the cruelest month...} 
\emph{All} Aprils included;
the April of Chaucer, the April/Easter of the
crucifixion, \booktitle{The Divine Comedy}, all the
Aprils contained in \booktitle{The Golden Bough}, and
all the rites of spring, sacrifices and renewals
to insure fertility, of dying and being planted
in the earth to spring up in a new form. Are
we to mix our terms here? For example,
political, social, economic conditions, to say
nothing of genetic evolutionary and \enquote{adaptive strategies} 
and continuities to be considered as the stuff of literary concerns\ldots\ at
least not \emph{directly}.\footnote{Although the naturalists had tried to meld science into literature.}
Nor had passion, hatred, character, conflict,
ceremony been allowed to be part of science.
But once we consider the universe to be
language, information, then fiction and
magic permeates it all. When experience and
reality are processed by computer, the usual
domains and disciplines are mixable. Once
discrete realms shatter again, their languages
melt, float, interface. Multi-lateral Pythagoreanism. But then, hadn't this modern fusion
of realms been anticipated by dreams and surrealism,
but in a non-quantified, anti-particleized way?

Agriculture itself as a metaphor for death,
all resurrection, and conversely. In ancient
thought, of course, the spoken and acted-out
rite is a ceremony pre-operational to planting: it is a planning and management 
system. It \emph{precedes} the actuality of material life, as the modern rite of manipulating the memory of past and future wheat crops inherant
in money \emph{precedes} ali spring plantings and
growths.

There is the question of metaphor and
simile, which the computer \emph{cannot} handle
unless given specific instructions, and then
within a limited set of circumstances (which
may grow, but becomes unwieldy). Metaphor
is a form of \emph{fused association}, sometimes of
completely \emph{unrelated} terms (except in the
mind ofa poet, novelist, or advertising copywriter), to create a third term, but in a pecu-
liar way. In the world of program-driven
computers, one list of items may be matched
to another list. What is compared are two or
more strings of \emph{stored-up impulses} (given in
one or another set of computer languages).
Beneath, on the level of circuitry and
machine-language, something different happens than thought. \emph{Items are not being
compared.}

All items in natural language are not
bounded by the compartmentalization required by a computer: they have no true
boundaries. Truly different sets of items
\emph{cannot} be compared unless a tedious and
endless program is written on the order of "if
[\![ this \ldots\ ]\!] then [\![ that \ldots\ ]\!]" 
An algorithm describing the way metaphors are generated can
be easily written; it cannot be implemented
or generated by a computer language. The
instructions could be written such that
\enquote{whenever \enquote{April,} then \enquote{cruel}}, but only
applies to these terms whenever \enquote{April} and
\enquote{cruel} appear. But supposing that another
poet appears, working out of another framework, to speak of April in conjunction with
the autumns of the southern hemisphere?

In order to unlock the poem's meaning,
one already has to have access to a vast
knowledge of literary, religious, anthropological, political, historic, mythological and
psychological compilations (and those edited) in order to summon up the full text of
references. The way the fragments are allusively juxtaposed may be analogized to the
way information is stored, organized and
sequenced on a disc ... in non-sequential
fragments with memory addresses. Well and
good.

It may be said that the retrieval process of
the brain-body is somewhat like the retrieval
process in a computer, but there are significant differences. (That is if anyone knows
what goes on in the brain). The computer was
orginally likened to the brain. The terms
were reversed and the brain was likened to
the computer, leading to ridiculous assesments—tried out experimentally—that the
brain \enquote{processed} information in an on and
off way. Later this was modifed and it was
said that the brain \enquote{parallel-processed.} The
mental processes are associational and quasi-random, and frequently get confused, yielding felicitous mixes. The computer processes
are much more rigid and limited. It is when,
in the poem, the fragments are fused that the
difference becomes more apparent. The
computer cannot fuse associations into a
seamless whole.

Fiction, drama, poetry, non-quantifiable
psychology and other traditional modes of
discourse are partial but stand for wholes;
they have long incorporated complex modes
of organization of people, events, matter into
dramatic sequences ... novels, plays, epics,
poems, psychological theories ... Creating
indices, hierarchies, queues, maps, models,
simulations, translations, sets, classes are
some of the problems raised by information
handling. Each reorganization raises these
problems again and again in new ways.

All literary works contain, among other
things, indices and every such work solves
the problem of hierarchy, or of queueing
without specifically delineating these modes
of sorting as problems. Literaure does not
accept polarities as absolute oppositions.
Dialectics is the emanation of crippled and
self-constrained minds ... the realm, really,
of accountants. Hierarchy? It's all in Dante.
Sets? Borges deals with them wittily. Indexes? See Eliot. In literature (and literary
psychology ... such as Freud's or Jung's) all
these modes are dynamic, allusive, \emph{multireferential.} 
As for set-theory, this is, of
course, the mathematician's and logician's
whimsey. As in life, literature shatters sets.

In real life, all sets are fuzzy. For example;
in a complex, transnational, transtemporal
holding company which owns other companies and parts 
of companies and constantly seeks to conceal itself behind a thousand
portals represented by the shell-names of
companies distributed in a lot of countries,
which is the set of all sets? The subsidiary or
subset may contain (by control) the set of all
sets. In fact, taken all in all, the contiguity of
economic (and political) activity eventually
links and sends the representations of each
part of itself at incredible speeds to every
other economic and political body \emph{in the
world} these days, especially in the age of
advanced telematics. In the modern world,
America, France, the USSR
(since transnationals overlap their boundaries) ... are indeed fuzzy sets, better expressed in literary
language... indeed, the language of \booktitle{Finnegan's Wake}. (And one may say that a set,
called for convenience Mafia, is conjoined
and interpenetrates those sets called corporations and those sets called governments
through other interpretive realms called
politics, by using a set-violating, economic
activity called bribing... It is, to be sure,
primarily and economic organization, but
on the other hand, requires primitive rituals
as part of its sustaining power.)

The way of the fiction writer and mythmaker is a function of long stores of knowledge, arrayed in certain ways, drawing from
a taken-for-granted memory bank. The world of information processing is a world of partial faiths and fictions. Tests for truth that
match reality are meaningless in this world: inner consistency is what counts.

In literature (its application to the new information world will soon emerge) one
converts the experiences of the self and
others into words. (What sensors does one
use to acquire the experience of others?
Through what set of filters --- ideas, technology, legends, myths, psychological and social theories, artificial memory/intelligence --- 
does one sift one's own personal experience?) The writer uses imagery, similes,
symbols, signs, translations, conversions, comparisons,
metaphors, tropes, compact representation, character, emotion, conflict,
drama in certain limited situations. All
dictated by a body of traditions. Experience,
real people, furniture, space, action, geometry, geology, geography are converted into
evocative words and are arranged into some new structure, perhaps
more neatly --- or more amusingly, or startlingly --- than in life.
Fiction, poetry, drama are programs designed
to transmit energy which \emph{amplifies} as it goes
through its \enquote{circuits.} There is selection.
One cannot write about everything.

The compilers and refiners, the preservers and organizers, the abridgers who assembled and trimmed the treasure, the great
canon of literature (surely there was always editing and censorship involved), assert that
their Great creations represent whole populations ... Man and Woman writ large. But
in fact these are \emph{sampling} techniques, which
can provisionally be said to have been invented by the Greek Tragedians. But what is
left out? Among other things, the living stuff
of humans, which involves neuronic, hormonal, enzymatic, chemical, metabolic, genetic, electromagnetic activity had not been
fit for literary language. Yet one could come
up with a prototypical set of biological reactions accompanying (indeed, initiating and
sustaining) mythological, literary and religious prototypes and archetypes.

When Proust bites into the tea-soaked
Madeleine, the taste is a stimulus to releasing memory which then pours out and is
recorded as a printout. We have a report of
Proust's brain operating, but in another language. We could also write a biochemical
essay which replicates the event. We could
also say that Proust wrote a treatise on information storage and retrieval, on the long and
short-term memory, in which the tea-soaked
Madeleine is the key word, and the instruction, that begins the memory dump.

When a marketing study is prepared,
psychology and sociology is used to create
the stimuli, the association of ideas that will
generate the \emph{feeling} which reaches the memory to release biochemicals, electrical charges,
symbols. All emotions, behaviors, dramas
and tastes are tallied, sampled and compared with standardized and representative
beings: prototypes, concurrent archetypes:
the hypostatic buyer. Frequently life-signs are monitored by biological telemetering
equipment. But, at the same time, the marketers must try to turn diverse populations
into Standard Consumers; infecting the archetypal consumer's memory with prototypical
hunger. Persuasion.

\emph{True} archetypes, \emph{original} archetypes, prototypical figures, their deeds and emotions
are communicated down through the ages.
From time to time their attributes are altered
to adjust to newer ages. These are used as
templates to force people into those imageforms. Consider that message called \emph{Oedipus}
and how it has been used again and again.
But were such recorded, aboriginal events
prototypical? Who, what set of people made
them that way and why? If Jung talks about
the collective unconcious, one must not only
ask, \enquote{who collected it?}, but who keeps
reminding us? And what are we not reminded of? Who speaks of the archetypal
revolutionary, the primal guerrilla?

Some set of \enquote{orginal,} transcribed events
and characters are made into standards,
broadcast and rebroadcast. Those who follow
are made to resonate to the original
signal. People through the ages are made
into transcievers. The emotion of a perhaps
once-lived life in turn gains power to motivate people across time and space. In fact,
the people, the transcievers, the relays, are
frequently more emotionally moved by these
compulsive fictions than by their own life or
the people they live among. They are taught
to screen their own experience through
these long-transmitted stereotypes, reassessing their own lives, comparing, matching,
referencing, denying what they are in fact
living out, viewing what is around them
through mediational scrims.

Some humans, fearing death, watch the
butterfly emerge from its chrysalis, or they
see the seed planted, going into the winterdead ground, and grow up as grain, and
dream of the time when they can metamorphose, transubstantiate into angels and gods.
The whole algorithm of agriculture is described in Christian thought, represented by
a human\slash divine figure in its metamorphoses
into divinity. So they invent varieties of
immortality. The way to immortality lies
through death; it begins with a dissolution, a
liquidation and ends with a reconstruction,
recombination, resurrection.

But the progrssive series of deaths and
resurrections lead somewhere. (Not \enquote{true}
death, for energy and matter are conserved.)
If we view the world, the universe in a
quantum-operational sense,
in which the observor intervenes in the observed, then, as
we have said, \enquote{Mind} permeates the universe and must constantly be sustained and
reproduced or the universe will cease to
exist. Thus, \emph{all} the instruments of perception, of past and present, place mind in the
univese and the universe in\slash as\slash of mind. True, the referential frameworks change and
the rules change. Thus dreams permeate through skins, flow through all boundaries
and are shuttled back and forth in history;
empty space becomes a plenum which manifests itself into sometimes virtual entities,
fields of particles and waves, sometimes real
and permanent entities which are viewed
with astonishment. Life is an illusion; the
flow, the dance, the perpetuation and evolution of language is all, and bodies lyse into
language, symbols, matter, space, velocity, energy, bodies, money... (Perhaps humans
resist this liquidity. When humans are converted into economic symbols, one form of
capital, the being is liquified, transsubstantiated into capital. What is capital but a set of
numbers which will metamorphose again into factories or other humans?) The first law
of thermodynamics tells us that matter and energy are conserved. Then the instruments
that view these events are also conserved. That is to say mind is eternal. (Really?) And
perhaps, so is time. And clearly we can see the whole program of the birth, life and
death of the universe speeded up for us in a reconstructive program.
But more, we can now populate our computer with a plethora
of virtual, interconnecting and concurrent spaces and minds, as the Kaballists had proposed so long ago. Perhaps we will see
them, since these fictions will have themselves become instruments of perception.

In the pre-Judeo\slash Christian past, the metamorphoses were circular or cyclical, 
happening without any purposive, long-range
strategy. The Jews and Christians introduced
two complementary --- yet opposing --- long-range strategies; non-repeating, goal-oriented
history, which led to collective immortality,
out of which sprung metaphysical evolution. Events in time were arranged in an
ascending sequence leading to collective
death, transfiguration, resurrection and escape into paradise ... a reuniting with that
from which they had originally been seperated from.

With the advent of the industrial-scientific-technological-capital
revolutions, the transformation programs \emph{appeared} to be secular. Events were arranged into a \emph{non-}Judeo\slash Christian, long range strategy of transcendance. \emph{Qualitative}, metaphysical transubstantiation was replaced by an accretive, recombinatorial, technology-assisted march
toward transubstantiation (since everything
can be prised apart into units, numbered and
rearranged). This invention was called progress, though still incorporating the old notions of immortality. At certain stages, the
accumulation would reach a critical mass
and a quantum leap into a new period would
take place. Permanent revolution. This was a
reaction to that perpetuated, obsessive dream
of a faded paradise, the Roman Empire, and
the descent into Feudal chaos. How many
expulsions from timeless and non-progressive paradisos haunt our \emph{collected memory}?
This kind of thinking led to concepis of
rational forecasting, retrocasting, planning,
management.

But since events had to be scheduled in a
practical, materialistic, exponentially rising
line, in order to gain escape-velocity,
in order to escape entropic fate, it was clear that
man's ascendant journey required a series
of engineered metamorphoses. The mechanisms of natural evolution seemed to be
gone. And anyway, Man wouldn't let any
new form that emerged from him, challenge
him. This meant that the journey toward
immortality required tapping the earth's and
the universe's energy. And if one were to
harness the universe's energy, that would
cost a lot, and anyway, most people, while
fearing death, were not nescessarily interested in immortality: the enterprise had to be
sold. Enter \booktitle{Faust}.

\booktitle{Faust} stands for a key word, one of many
in a directory of transcendance strategies.
\booktitle{The Divine Comedy} is another. \booktitle{Faust} is
dynamic and action-oriented: Dante is static.
Dante (and The Church) encouraged saving
of energy and cutting down of indulgence-expenditure --- called sin --- while Faust encouraged profligacy as a means toward
progress.

One general form of the transcendance
algorithm runs this way: Hero seeks, or is
impelled to seek knowledge; hero has dream,
or dies, or passes into another realm; the
new, or next world is revealed; hero comes
back, or leaves directions in the form of a
book. Note the role of knowledge or information.

\booktitle{Faust} is the great poetic myth representing
the transition from the medieval to the modern age, from medievalism to capitalism,
from agricultural\slash feudal society to industrial
(and then to the information society), from
one kind of magic to another. (And it is
Eliot's lament that we had taken the wrong
path.) \booktitle{Faust}'s ascent is built around one
concrete and one mystic project. The concrete project involves a dam and land-reclaimation. 
\booktitle{Faust} is in every marketing strategy
the computer and software manufacturers
generate. The mystic project is the transubstantiation of, the rejection of the body and
the earthly life, earthly events and time,
history, and mundane procreation, utilizing
a meta-sexual image. \booktitle{Faust}
becomes, as it were, one of the programs of modern society.
The dependance on mystic knowledge, information, is very strong.

\chapter{}

Even the most primitive of tribal entities,
those most materially deprived, seem to
have extraordinarily complicated and sophisticated intellectual systems. And, along with
generating wonderous classification systems,
they stack up tales of underlying, fundamental orders, and the creation of these orders,
into layered versions. In contrast, modern, western thought is relatively simple and
simplifying. 

\emph{Formally} speaking, all these \enquote{pre-modern}
systems are very alike: in practice, quite different. 
The introduction of modern formal analysis may be a particularly western
mode for making unlikes alike, a strategy for
destroying singularity and quality, still lurking in ultra-modern,
western civilization. We see the imposition of this western 
formalism on the underdeveloped world. On a
formal level, these older systems seem to
bear a curious resemblance to modern ones.
Is this a function of modernist and leveling
perception? And all, of course, address
themselves to origins and endings.

A cosmic or godly event begins the series.
New versions are invented. Beginnings and
endings proliferate. Origins are projected
and analogized to inconsistent historical
and mythological structures, which are then
rationalized and united. The Kaballists ask;
what happened \emph{before} the Biblical version of
creation? The Hindus retroactively add geological strata of explanation and, \emph{literally},
concretized them out of mountains of stone,
chiseling the \emph{Kamasutra} of creations:
a sexual version. The Mayans and Aztecs choked
their cosmos with Gods of retroactive explanation.
But, as far as the Judeo\slash Christians
were concerned, nothing less than the hunt
for \emph{The One} would do, worshipping at the
shrine of The Great Unifier and Explicator.
But, yet, in entropic time, the cathedrals of
nuclear particle and information priesthoods
abound with sacred Fundamental Particles
and Forces (or operators). Looking backward,
the god or culture hero, the calculator-supreme, the order-bringer who will, after a
hunt for a mystic vision, will enter the
spaces of inspiration and blinding insight:
there he will view an inscape, not of earth.
After which he will return to deliver the
Message of the Oneness of all things.

All information theorists erect their knowledge-processing gods: Prometheus, Lucifer,
Thoth, Metatron, Hermes, Hermes Trismegistus, Simon Magus, Giordano Bruno, Jesus
with his tidings of great Joy. Now we can
look at the latest culture hero in a new light:
as \enquote{truth-bringer.} Under the hero's mythic
appearance are subsumed physicists, astronomers, 
molecular biologists, financiers, geneticists, neurologists ... is the five-faced
Cyberneticist: Alan Turing, John Von Neuman, Norbert Wiener, Claude Shannon and
Noam Chomsky (it should be understood that these names are mere variables for
which other names may be substituted).

The great cosmic battle that calls them
into being are the first two world wars. The
civil war in western civilization is matched
up to that great pre-primal civil war in
heaven, the aboriginal revolution. This great
Initial\slash Final struggle provides the impetus
to machine the universe and go into business. Before earth existed, God and Lucifer
battled. Earth was created (a rearrangement
of space, matter, energy and time) as a tool
and a battlefield in the Grand Struggle. The
Grand Alliance vs. the Comecon legions of
Pandemonium. Before the earth and the
galaxies were invented (or reinvented), the
Communists sinned against The Light and
had their Great Fall from Grace. Or conversely, the Capitalists sinned against the
light of Primitive and Paradisical Socialism.

The Delphi where their cybernetic or information thought was born are the Macy
Foundation, the Rand, IBM, Bell Telephone, bank wire rooms, the coding, cryptography
and signal intelligence systems of government,
the military and the universities.
Rather than being anything new, this mode
of thinking was designed to reorganize and
incorporate the long tradition of western
rationalizing and simplifying: Greek and
Judeo/Christian thought. This scattered and
compartmentalized Delphi strove to organize all divisions under its aegis, devouring
and engulfing all before it, metabolizing
diversity into this new, computational-assisted, gray pap. This assumption, this reclassification of diversity into fundamental order and unity in the universe, all things in
it and all interrelationships among them, may, in the long run, be an act of faith ... the
propagation of a magic spell to colonize all minds.

Magic depends on a community of belief.
The more who believe, the stronger the
dominating resonance of the vision, which
is then broadcast out into space like an
incantation. If, indeed, as the quantum
theorists would have it, the observing mind
and its prosthetics intrudes into the cosmos,
then wouldn't a cacaphony of visions and
systems lead to a plethora of spaces? To
make sure that chaos is averted, peer review,
and ceremonialized, hierarchical orders of
permissible discourse allow only for glacial
change. Experiments which might prove incontrovertable
diversity, or lead to a final
skepticism are not funded. Goodness: what
if Ernst Mach, who disputed atomic theory,
was right?

In modern thought, each version of universal order, developed in history, subsumes
and tries to erase past or contradictory versions, negotiating away genuine differences
or, at best, converting them into polar opposites or contradictions. But to call something
a contradiction is to subsume it. The early
efforts of unification included the pantheonizing activities of the Greeks, Romans,
Christians, Mohammedans, Hebrews, Buddhists ... who attempted to compress, resolve and rationalize the many gods (who had many attributes and stood for many things), spirits, realms and cultures. Early
unified-field theorizing. (And at the same time, intractable organic trees were converted
into rows of stone pillars with stone leaves on top to control the wild proliferation of
nature and to house this riot of now-domesticated gods.) Not only did the arcane formations of the past lurk on, visible to the eye,
palpable to the touch, but in dream-states (since brain activity generates a wild, a surreal associationism, generating metaphors and conceits); the ever-present archeology of
past formations collected in social thought (the true artificial intelligence/memory)
which was the matrix through which education is sifted. These re-emerge when unifying attempts run into trouble in their encounter with intractable reality.

When order, universality, oneness,
falls apart, when classification fails,
fallback positions are prepared. Contradictions are
invented, the perniciousness of dialectical
thought. Human thought (or at least the
thought of some subset of humans) seems
obsessed with the use of polarities to explain
what will not fit; similarities percieved as
counter-identities. (Is a lobster the opposite
of a human?) All this finds its way into
computer thought, based, simply enough,
on addition ... mechanistic thinking built
on the limited operations that, first logic,
then switching devices and logic gates can
perform (input-output and feedback devices).
For all its complexity, any computer has to
use a symbolic logic, which is limited by the
control of the flows of electricity. The speed
of a vast amount of miniscule operations is
mistaken for complexity. The messiness outside this logical world --- 
whole living ecosystems in wild and wonderous irregular
shapes, plants, marine shells, animals,
microorganisms, a memory of jungles, sea
bottoms, a casual distribution of galaxies --- 
must be reduced to binaries, Cartesian\slash
Leibnizian pixels on an image-processor's
screen, or a printout. And this presumably
matches the world outside this hermeneutical cave 
of transistors.\footnote{Computerized image-recognition depends on building up a reference library of simplified, sensor-apprehensible images, compatible with computer recognition.} 
All this is another way of, as
Aristotle put it, holding up the mirror to
nature. We see what a few of us, who have
designed the sensors, expect us to see; the
designers have their preconceptions reflected.

Economic behavior is also obsessed with
input-output polarities; systems which account for-and-of capital flows, the \emph{yin} of
debts and the \emph{yang} of credits; male gain and
female loss; dark and light; the dialectic in
which money in a bank is a liability and
money out on loan is an asset; the hurling of
a profit in Frankfurt to a loss-column stationed in Panama (but both in an electronic
balance sheet in, say, New York), or anywhere. All quite Hegelian. Hegelian thought
is particularly applicable to accounting systems. In the first place, it is ideal, which is to
say that it deals with representations, not actualities. In the second place, implicitly
shows progressive and inevitable, even divine growth --- an organic metaphor --- which,
when tied to evolutionism, fulfills itself, or God's Purpose, in Time.

Now, the question is: was Hegel the father of the modern, automated balance sheet, or
does his thinking derive from double-entry accounting practices? We have been trapped
since double-entry bookkeeping and unit-pricing was invented by the ancients. Another
question is, how to deal with the unexpected,
random, \emph{risk}, and uncertainty?
If no one knows what happens \enquote{out there,} but projects: if no one knows what happened 
\enquote{back then,} but retrojects (doing a long run ... for
which a computer is ideally constructed ...
if the events can be specified), the past,
present and future can be procrusted into a
prophet's or risk-analyst's dream. (Consider
Joseph in Egypt, whom we will mention again.)

In physics, this long range dialectic concerns itself with entropification, for which
quite Manichean Maxwellian Demons were
invented to reconcentrate dissolving matter and overcome long term loss. Maxwellian Demons
dealt with particle matter, but couldn't deal with the quality of matter; it
was a Statistical conceit. Norbert Wiener
derided the very notion of \enquote{quality,} which
he considered a Medieval hangover. How
can you measure or chart a quality? Humans,
inventing particle physics (physical biologists) see humans as agglomerations of
quanta.

As for business, the non-quantifiable aspects, compulsive behavior, fundamental irrationality, the rites attending successes or
failure, the ceremonies of contacts and connections, the accompanying trade in prestige and rank (and frequently women), panoply, display, to say nothing of individual obsessions, trade in contracts, stupidity, shortsightedness, favoritism, structural theivery,
fraudulent accounting, altered and destroyed
records, computer glitches (and bad design),
invented numbers, nepotism, bribery, kickbacks, the giving and receiving of presents,
acts of abject faith, all are ever-present and
must be factored in ... however indirectly.

These sorts of thinking (particlization, accretion,
mutational variation within combinatorial limits, dialectical contradiction,
fleshly pythagoreanism, accretive historicism
on the pilgrim's progress to transubstantiation and resubstantiation), when applied to
the replication of organisms, give rise to the
notion of genetic structures as information.
This living, quivering biology, this jellyware,
comes to be seen as mysterious codes, cryptograms, instructional algorithms for the development of bodies. In sociobiological
thought, bodies are mere totality shells, packets, envelopes for transmitting genetic
messages along a carrier wave, that becomes embodied from time to time, down the ages.
Genetics is not only information, but it is memory.
(And, as an instrumentality that
interferes in the universe's workings, generates mind which then generates it.)\
Perhaps, at time's end, the messages will reach
such a level of accretion and recombination
that humans will evolve and transmogrify
into angels. (Hence, one burden of this essay is time itself.)

\chapter{}

At this point in history, the conceptual
and theoretical constructs are distilling and
summarizing the past into programs that
mimic natural and human activities. And
conversely, the rich paper records are being
concealed, secreted away in caves like the
treasure of the Niebelungs. The distillation
remains in databases,
hoarded by large corporations and governments.

More and more the world is seen in terms
of information no matter what the reality is.
Just look at the account books, the numbers,
the projections, the returns. But computerized account books tend toward a sort of
semi-autonomy --- market-to-market, interactively linked --- and drive this outer reality
before it. The investment in computer-compatible thought is so great that more and
more we become trapped in this new culture
and they cannot admit that we have been led
down the wrong fork in history's decision
tree.

If all is fundamentally the same, it follows
that a data base in one language should have
the power to talk to data bases of other
disciplines in other languages (mediated, of
course, by programmers, protocols, translators, modems, computers, networkings...).
One might have to descend into the primal
language and then, choosing the right fork in
the decision tree, emerge into the proper
language. If only one can design the right
protocols, ones that will not only link among
unlike, competetive machines with unlike,
competetive architectures --- IBM's,
Control Datas, Apples, Crays, DEC's --- but also unlike
transmittal systems run by competetive companies. Languages can be united because
each field and domain, each way of looking at things,
should be a subset of the one,
universal, primal language. Perhaps what is
expressed in one domain should be considered as an encryption of what is expressed
in another domain.

However, not only do computers in different disciplines not translate into one
another well, but different manufacturers
and communicating companies (to say nothing of nations), while proclaiming one world,
one language, falling prices, one global village, and universal compatibility, fight one
another tooth and nail. They erect a maze of priced mediations and product differentiation,
countering speed and directness of transmission with profitable labyrinths, in
different time-zones, each turn and gate
tolled and tariffed, competing and maintaining secrecy, organizing those to whom they
sell services on a need-to-know-and-pay
basis, playing the differentials among different states of being, business and knowledge. (Citicorp, for instance, computerizing
and gaining speed, places its headquarters in South Dakota in order to --- taking 
advantage of the laws --- gain advantage which allows it to keep checks for a certain time and
thus enjoy a float in the empyrean.)

There are certain laws to be deduced from the observation of business practice. Information
management, traffic control and
pricing follows the timeless strategy of railroads in the past: which is to say, given a
certain limited distance, the problem becomes to increase distance by increasing price.
Economies of scale are developed, need for certain volumes regardless of content, development risk to be paid for by the consumer.
Tesseracts of tax shelters spring up. An incredible maze of contradictory laws emerge
requiring incredible expenditures of intellectual energy and computing time.
Information theorists always leave out the costs.
Claude Shannon quantified information;
AT\&T and IBM priced it. Shannon's theory did not develop in a vacuum; he did his work
for Defense and Bell. Where did the money come from? What did the funders want and
what did they not want? What other enterprises cross-subsidized these developments?
What solids were melted down, who was liquidated to fund the Great Enterprise? No
different than the practices of the ancient Phoenecians, Babylonians, Greeks, Romans,
Venetians, Fuggers, or any other merchants in history. (In addition, of course, the amounts
of energy, in terms of electricity, required to run and cool computers is staggering.)

If we take into account the human, informal, anti-organizational, shadow-organizational networks,
the person-to-person
contacts, those who emerge to resist this development,
those who have an interest in not
sharing information, we see vast, centrifugal
forces at work. On the one hand, the emergence of a unified system, a sort of electronic
Catholic Church: on the other, a sort of
electrofeudalism.

Given all this potential convertibility,
how can money talk to nuclear particles,
pension funds speak recombinant genetics,
prime numbers retrieve fictional heroes\ldots ?
Can we really create a translation program,
which is to say a unified field theory? Or
should we, not having been invited to the
initial feast of reason, create a \emph{disunified field theory}?

The primal-language business, like the
origins business, is highly competetive (since
the costs of computer runs is much more
than paper experiments). One of our many
ultimate transformational and alchemical
media, a primal liquidity in which all life is
dissolved, reconstituted and redisolved is
genetics. What is the market value of bioengineering as expressed in some form, with
purchases involved, with manufactured products and processes at the end, investible
end-products and investors screaming for
their dividends, trying to hurry time up?
Will it cost the world's savings to transform
humans and will we be left with one creature
at the end?

We raise the same questions about particle-wave physics and its ruinously expensive paraphenalia.
Finance, literature, genetics, nuclear physics: four (of many) primal languages; three
media in which translations from realm to realm can be seen as new versions of progressive metamorphoses.

\chapter{}

So we begin again from another angle,
using one of the most highly computerized
of modern entities: the corporation. We will
talk for a while about entities, borders (or
skins), sets, organisms, time-series, populations and genetics.

We live in a time when corporate operations have become speeded up. They have
become mobile: they waltz across the sea:
they take wing and fly into the night: they
grow slender and slip through 30 gigabyte-wide needle's eyes. They attack and pillage
one another. They split into pieces and recoalesce. They live in this age and other
ages. Metanational entities gobble up chunks
of their national hosts. Nothing new about
that: what is new is the informational paraphenalia and the high-speed, distance-insensitive, time-devouring equipment which
pressages a vast political change.

Corporations are --- at least legally, and
metaphorically --- seperate entities. At the
same time they are not. They are linked into
para/meta/supra/sub-networks by investors,
interlocking consortia,
joint ventures, cartels... They defy the notion of discrete sets.
A corporation has a boundary only for purposes of classification and identification.
(By the same token, one can also say that
conceptual imprisonment denies the human
their individuality.)

A corporation is defined as a living being
in the \emph{contemplation} of the Law. A corporation is diverse. The contemplating Law is
also not only diverse, but dynamic and
changeable, requiring lawyers, plaintiffs,
litigants, defendants, judges and legal memory banks (precedents) assembled through
the ages. (Law too is being placed in data
banks, where into a system called Lexis.)
a corporation spans nations,
And it is contemplated by several systems of law at
the same time. Are we expected to believe
that the emanation of a complex of people,
events and memories, the law, can \emph{contemplate} the abstracted emanation of a complex
of people, events and memorigs, a corporation? That is to say a fiction, or anthology, or
novel, contemplates another fiction, another
anthology, another novel. A sort of organic
\emph{character}? We need hunt no further. Here,
truly, is artificial intelligence. Why spend
any more R\&D money?

The corporation
can be represented as
information in a pure, financial form, which
is a slice of the composite life of lives and works
in progress, taken at some point
in time. (Financial institutions, especially
banks, are considered here because of all
business, the financial entity is the most
heavily intellectualized, the most heavily
dependent on computers and communications. The technology is inextricably bound
up with the value-flows, the calculations
being fundamentally simple.) While appearing abstract,
ideal, it is neither ideal nor
platonic, nor is it static. Its positions can be
caught in an account sheet, but not in its
motion. Its motion can be caught, but the
positions are lost. Sound familiar? Its existence now requires the constant intervention
of humans and machines, managing, evaluating, working, integrating it into a market
structure. It includes several methodological histories; the evolution of the notion of
evaluation and the history of its accretion of
value. This is to say humans and machines
producing, trading, buying and selling money
or near-money. If it is to have life, it must
have lives to keep on working. Its abstract
operations have concrete results that drive
the lives of the people inside of it and
outside of it.

With the advent of the high-speed calculator, robots, and programs (based on long
statistical runs of past performances, accretions of admonitory history-scenarios and
tortured equations) which mimic some financial transactions,
such trading and instantaneous as automated communication
programs to link buyers and sellers into an
electronic market, it becomes possible to
concieve of a pure, automated and constantly adjusted financial corporation, one totally
devoid of humans, territories, factories...It
would do all the things financial organizations do: move money around, trade, arbitrage, take deposits, account, merge, acquire, make long and short term loans, divest, invest, liquidate, grow, collect, lobby
for laws, pay or dodge taxes, contemplate
risk and probability, now and then order an
assasination, all while living in exotic, anaerobic climes.

If there is no plant in the physical sense,
there have to be virtual \emph{plants}, as-if factories,
represented on paper or in computers, stored
in data banks, to guide bankers in their
moves, driven by modes of evaluation, lists
of people and institutions to borrow from or
loan to, risk studies about good and bad investments (prophetic programs; Joseph scenarios), market-switching-and-routing programs for heaving pools of money this way
and that.

But to be meaningful, this fictional being must be connected to other markets and
other forms of endeavor in real time, a something, somewhere in the universe to connect
to, someplace to enter inputs, a someone, or
set of someones or the simulation of someones to credit these messages from this totally automated financial institution. Most of
all, such an ideal financial being, in order to
exist, must be credible: that is, accepted as a
motivating act of faith, by other institutions,
other beings, and Law. This has not happened yet. As we talk about corporations, we
still detect humans ... somewhere, if only
living in palaces. It still needs humans to
interact. Humans to be affected by these
interactions. Although one can see a time
when some automated complex of corporations consume
what some automated
complex of corporations produce.

We talk about the life of the corporation in
several ways, as we talk about the life of an
organism. Humans are stockholders, directors, traders, officers, consumers,
workers. All who contribute to the life and existence
of this contemplated being have histories, both social and biological. They may also be
defined as a gene-pool, although differing from a race. This population is the result of
diversified reproduction as against non-diversified reproduction (a tribe, clan, race,
ethnicity, nation; those with subsets of shared genes). There is, to be sure, such a
thing as a family corporation, a form that
bridges the gap between a dynasty and
shareholder \enquote{democracy.}) Thus one can
say that the corporation has a sort of genetic
structure, one composed of sets of genetic
structures, each set composed of parts of
other sets, defined or ensetted not so much
by family or race consanguinity as by their
participation in the enterprise. Indirectly,
this corporate organism has, so to speak, a
genome, which is to say a complete genetic
constitution.
At the same time its participants invest, give life to, guide the destinies
of other such entities which have different
genomes, for one must spread risk and diversify.

Another way to think about the corporation is from the perspective of investments
(stock, bonds, etc.), plus other rules for life,
governance and growth (even replication).
This combination of investments and rules
can be considered as the genes of the corporate, fictional organism. These metagenes
are said to express themselves into living
organisms. Genes, information of living beings express themselves into other living
beings and also into metagentic forms: capital, which has its own rules of continuity and
metamorphosis.

Investments shapeshift into material life.
Here we have begun to introduce the concept of modern significant demographies in
which contiguity, as well as continuity is
provided by investment, participation and
modern communication. Replication does
not require face-to-face existence. And so the
question of space and time is raised again.
Offshore receptacles with names, await;
Panama is said to have at least one hundred
thousand corporations of all sorts. Some are
real. Some are shells, mere names
of fabulous beasts that are inspirited and informed
by a shower of electronic gold.

It is possible to perform a mind experiment. Consider the genes, in their informa-
tional aspect, of all who interact with the
corporation, and search for something in
common between genetics and corporate
life, belongingness, in turn related to the
equity\slash debt\slash assets expressed as stock, directorship, management, etc. This new set of
translations relates to the organism's two
totalities and is exemplified in the annual
report, the balance sheet, which is a slice of
life, frozen in time, but as transtemporally
allusive to past and future histories as, say,
Eliot's time-meditations in \booktitle{The Four Quartets}.

This being's genome can be seen as the
result of a long stretch of accreted historic
information (the story of the buildup of
equity and credit) which can, through a
series of reconstitutions (as one reads past
organisms, arranged in a historic sequence,
in the genetic memory of any being), be
remodeled backwards on to life, production
and reproduction of gods and humans. The
\emph{names} of those past and present humans, the
lives as they live, replicate in parallel, can be
mapped to the corporation's replication pattern. Or we can go forward again from these
humans
to dissolve
them,
their essences,
into this legal fiction, this corporate being.
It's a simple mapping problem; genetic information encrypting corporate information
and conversely. Once again; we should not
forget that both sets of information involve
human activity, or at least the \emph{still-living
memory} of human activity. Considering the
age-old continuity of some dynastic fortunes,
this is not too arcane to believe. After all, the
corporation is a metaphor embodying real
human existence. It is a chimera.

Corporate mergers are referred to in sexual
terms; marriage, even rape. Are these mere
words, anthropomorphisms used to describe
a phenomenon too complex for words? Shorthand? Key words, which when decrypted,
open up a vision of vast legal and credit data
banks containing huge record depositories
of swiftly shifting law, money, reports,
memos,
accountings, as well as histories of
the mixing of complex social groupings,
populations, evolving, revolving through a
variety of forms? An analogy?

There is a problem here: are we to take a
metaphor seriously? Is it truly descriptive of
a complex reality? No, but... The keyword
always becomes an intrinsic contentual part
of the phenomenon descibed; it is wrenched
loose with the greatest difficulty. What do
lawyers and accountants argue about? They
argue about \emph{words} and \emph{semantics}. They
structure, deconstruct, semiotisize, mine the
law-bodies for hermeneutic nuggets: they do
\emph{literary} criticism and linguistic analysis.
They invent new, post-dated critical theory.

And if we are talking about fictions, try to
think of what the Oedipus Complex describes
without the word \enquote{Oedipus} to unlock the
memory bank. Try to think of the merging,
through a marriage, involving two individuals, representative of two complex dynasties
(corporate entities) and fortunes (bloodlines
or genes; property and treasure); royal families. Whenever we are talking of corporate or
dynastic marriage, in terms of informational
essences, we are describing an alchemical
wedding, a marriage of heroes, gods, archeiypal figures, and enterprise. But what is it
that is mated? People? Yes. But also information:
bonds, representations, money, deeds, abstractions symbols (stock, for land)
which are generalized information, not specific to any corporations, resolvable into
things --- people, factories --- as the genetic
information is potentially resolvable into
bodies of living beings, races, once properly
expressed.

A vast practical and ceremonial apparatus
is required to bring alchemical essences together. A vast, practical, biological and cere-
monial apparatus is required to bring the
genetic essences of any two people together.
When it comes to the marriage of dynasties
or fortunes, ceremonial behavior increases
(to say nothing about a great to-do about
contracts). Mating dances. (Stock in \emph{this}
company or \emph{that} company allows for certain
kinds celebratory rites and participations.
When stock is loaned against currency, that
currency is potentially part of \emph{any} corporation, but only when traded for new securities
with their particular, limited set of behavioral instructions.)

If a human is a combination of two halved
gene-sets, a kind of information-bearing and
organism-producing program, then a corporate merger is a multi-sexual, abstract orgy. It
requires dozens, thousands of essence-sets
to conjoin, to participate, to be transferred in
order to materialize into another kind of
existence. Certainly as long as human activity goes on, as long as computerized and
abstraction-activity goes on, as long as it is
recognized in the contemplation of law and
the faith of people, this wonderous being
lives. Go tell someone that IBM does not live.

Do analogies of corporate marriage incorrectly define this entity? While we wait --- 
probably forever\footnote{Sorry, Sol. --- S.W. Editorial} --- for pure, artificial intelligence to come on line and carry on the purified and transcendent sum of knowledge, making
new, autonomous decisions, we can say: no humans, no corporations.
No human reproduction, birth, death? No corporations. The corporations may not
actually mate, but mating and reproduction must go on somewhere. Is this any sillier
than saying, as the sociobiologists do, that the body is nature's way of producing more
DNA? The exchanges, assets, liabilities, accretions, all the other signs of operations
in the corporation's informational sphere
can be said to be determinants of social
behavior inside and outside (wherever its
influence reaches) this being. It's individuals are just as caught up in a sort of fate as
poor Oedipus (a dynastic drama) was. Considering
the rhetoric; invisible religions,
acts of abject faith, superstition, lurk beneath
the most rational, mathematical and scientific works. Magic continues to shape human
behavior.

In ancient mythology there were organisms that had lion's heads, wings, human
bodies, snake's tails, the heads of hawks and
owls... We could in principle make a genetic map
of such a being. A modern, diversified transnational being is, of course,
infinitely more complex and amorphous, for
have an organism composed of living we
things, dead things, and the rememberances
of things dead and past, no less phantasmagoric than those ancient sphinxes, chimeras,
minotaurs and hydras.

\chapter{}

Let's backtrack to wherever it is that
the Origins of Western Civilization are reinvented and then go forward to try a little
progressive meta-history of this most modern of organisms. Greece, a tiny time-slice
inserted far down, towards the beginning of
the ascent-trajectory of western Man. Our
art, philosophy, logic are said to begin here.
What if all memory of Greece were erased?

All countries, cultures and religions secrete, value and store up selective histories
and mythologies to legitimate themselves.
What do Europe and America consider themselves to be without this passing nod to Greek
origins? Even Hitler's and Mussolini's regimes featured Greek-derived iconography.
To make these histories, an amalgam of
events, myths, legends, folk tales, sagas,
religious deeds, epics (to say nothing of vast
storehouses of paper, as well as appropriate
mnemonic architecture and monuments)
were collated. Thus a quasi-arbitrary sequence is created, one linking events into
causal chains in order to create the feel of
mystical inevitability, celebrating the everconstant triumph of the present. Mistakes or
horrors are leached out or are considered to
be the price one pays for progress. Continuity, progress, history is a smoothing out
of the sudden, the disruptive, the violent, the
random and unaccounted-for. Even the most
rationalistic sequences must include the cultural, for logic too is a \emph{culture}.

The building up of Western Civilization is
a story of grand strides toward unity, yet
requiring breakdowns and fragmentation of
old formations: that is to say, re-feudalization before new reorganization (note the
re-feudalization of AT\&T). This totalizing
sequence constitutes a grand myth --- given
an ascendant trajectory which will avoid
breakdowns --- called progress or sometimes
evolution --- beginning in the past few hundred years when some percieved that the
industrial revolution had to rewrite the old
myths into new ones, forever ending circles
and cycles and introducing exponential
curves reaching into the empyrean (sometimes called space). To question this accretion at any point is to disrupt the spell the Grand Ascent has over us.

It may be argued that certain events were
chosen arbitrarily; an association of ideas
about event sequences, which happen to
have --- we have not accounted for pure invention --- operated in a certain temporal sequence, not nescessarily in a cause-effect
line. A politico-mythico-Lockian mnemonic.
Ultimately G\"{o}del makes the point that the
linking of all logical steps depends on an act
of faith, just as the assignment of meaning to
a pool of securities is an act of faith.

We can follow some of the many arbitrary
feeder streams that empty into this ocean of
Western Culture. Oedipus, for example, is
not merely a Greek play about a hubris-ridden, stubborn figure, one prototypical
individual, blinded and suffering. It is the
story of the restoration of balanced ethical
and divine budgets. The tale is also about a
power struggle, a plot in the face of an
environmental disaster: medical (plague),
economic
(starvation), genetic (dynastic legitimation: though it is not specifically mentioned as such, genetic information as fate --- 
as explained by the retroactive prophecy
delivered by Delphi --- are intertwined), and
demographic catastrophe (sterility in women
and potential population decline). The tale
is also about an information-search (among
other things for genetic origins). Who passed
ownership and title on to whom and how,
based on breeding lines of descent, a story of
false and contaminated claimants. It is about
the relationship of genetic purity to rulership, property, the laws that define those
relationships, the rules of mating. The health
of the nation is at stake.

Usually this play is interpreted for us as a
moral, psychological and sexual tragedy (incest). Freud led us in the wrong direction,
positing the erotic rather than the reproductive consequence, falling into a trap set by
the old Hebrews and Christians. Lust, especially incest, leads to death. But then, we
may ask, what did Freud know about life?
What little he did know, he lied about.

Genes, contaminated by \emph{deeds} (acquired
characteristics, or fate?), the inadvertant sin
of Oedipus and Jocasta have brought together
the wrong genes, illegitimately. The sins of
the rulers, or putative founders of this kingdom of Thebes (the information) contaminates the body politic, its well being, its
health, and affects all nature. Did the Greeks
know about genetics? No, but they were
obsessed with plant and animal breeding
and dynastic breeding strategies. The forced
comparison between human activity and
nature's response is interpreted in a magical
way and skews human behavior. (That the
play is also Athenian propaganda against
Thebes and Delphi is another story in itself.)
If the thought of the pre-Socratics, Socrates,
Plato and Aristotle go into modern rational
thought, why not these tales, which are
\emph{logical} constructs disguised as dreamlike
tales of scandals.

Another take: Helen of Troy gives birth to
the Hellenes. We can now understand one of
the reasons for the furor over Troy. Only
Helen can give birth to the Hellenes and
she---or the symbolic and real reproductive
apparatus inside her beautiful body---has
been stolen. Helen, mythic figure, like a
queen ant or bee, is a collective emanation.
As a carrier of a certain nation or race-spawning mechanism,
she is, from an information-perspective, also collective. \booktitle{The
Iliad} concerns a conflict of ethnicities, a
polarity called Asia vs. Europe, an East-West
struggle for the control of trading routes to
the profitable Scythian hinterland. The war
is used as a means of uniting diverse tribes
into one mega-ethnicity. The first Grand
Alliance, the first NATO-like Allies. These
themes are summoned up again during the
East-West war against the Persians. The Iliad
includes a brilliant, albeit indirect, essay on
set-theory disguised as a list of tribes. The
Hellenes are the set of all Hellenic sets: a
genome of a corporate body called a race.

The Old Testament can be seen as a set of
linked stories and myths about the Hebraic
relationship to God's Design (covenant, contract, law ... rules of the game, more honored in the breach than the observation), a
history of dynastic continuities; rules for the
preservation of the Hebrew genepool and its
royal subsets: the living propagation of
God's word, or perhaps the Word made into
code...God provides the operating system;
the Hebrews write the software). The Design
has rules for mating, accretion of power, and
how to transmit \emph{informational} treasure (\emph{The
Talmud}). The Old Testament is also a meditation on the rules of economic behavior,
negotiation of conflicts, law, contracts, politics, nations.

These notions of a nation encompasses a
genetic mystique. An ethnicity contains the
idea of birth from a set of primal founders, all
of whose descendants have shares in a commonality of genes and are related.

Implicit in the Old Testament and Judaic
law (and fulfilled in Kaballa) is a sort of
evolutionary trajectory along a linear path.
The history of Jews, unlike the history of
\enquote{primitives} is not cyclical except in the
longest sense. Gratification and fulfillment
are deferred. The Hebrews introduce the
notion of the long-range trajectory, completed
by the coming of the Grand Recombinatorial
and debt-redeeming wizard, The Messiah.
This Messiah will reunite the scattered
Hebrews (for how can they mate if they are
far apart?). And \emph{this} Messiah is only for the
Jews, only for \emph{one} genepool, no one else.
Kaballistic lore, a heretical meditation
on the same long-range cycle, is a wave with
only one fall and rise, spread throughout the
universe. The Messiah becomes a way of
reuniting, reconcentrating the power that
the disaspora fragmented, a kind of ultimate
meta-history of mergers.

Still another event: What strategy did
Joseph use to monopolize Egypt's grain production for the Pharoah? Prophecy, dream
interpretation, fear, land-reform, expropriation of commodities, rationalization of production (a kind of early, political, state-run
agribusiness), storage facilities, new gathering techniques. Clearly Joseph must have
used some form of accounting and econometric projection to affect a grand, structural
and political change. His interpretation of
Pharoah's dream is risk-analysis. Not only
did Joseph change the environment around
him, but his thought radiates down through
the ages and affects agribusiness today. The
short-range, forced march collectivization in
the Soviet Union, the long range \enquote{collectivization} and concentration into agribusiness
in the U.S. (using genetic cropping strategies), use the same pattern. Maybe we should
call Jung's idea the collectivized memory.

Up to the New Testament, the Hebrews are
heterosexual. The Jews \enquote{give birth} to the
Christians, or that's the way we tell the story.
God uses Mary as a child-producing\slash nurturing vehicle (expropriating variant tales out
of the past). Here is a departure into androgeny: self-replication, an ancient, sinful
theme, for this is what Lucifer did too. Incest? Autocest? Autogamy? Or is God a
wierd insect? New rules for sexual mating
enter the picture and implicitly alters the relationship of dynasty to property. By proslytyzing \emph{all}, not only Jews, the nation, the
ethnicity dissolves and becomes a sort of
corporation: shares in redemption are open
to all. The Messiah will redeem \emph{all} debt. The
Old Testament is endogamic: the New Testament is exogamic.

Historiography is expressed as reproduction, human continuity with attendant \enquote{qualities}; genetics as myth. But does the myth enter into the genetics of the present,
as many scientists turn back to what was religion and magic in the past? (One may pour
enormous resources into a project and use
scientific techniques in pursuit of magical
aims.) The Christian God's experiment enthralls the moderns and they try to replicate
it, using the Holy Spirit to fertilize a laboratory vagina.

Genetics and the right to rule, to own
property, come for the Romans at a time
when both myth and documentation exist
side by side inside the dynasty-obsessed
empire. The Empire funds the history of its
own origins, a vast, dynastic myth called
\booktitle{The Aeniad}, which traces the ancestry of the
Romans to the Trojans, not the Greeks.

For the Christians and the Romans, these
tales explain discontinuity while maintaining continuity. The new birth stories insure
discontinuity between the Romans and the
Greeks. In the same sense, Jesus is not only
divine, but a Jew and not a Jew. Aeneas is a
Greek, but not a Greek.

At time's end, the Christians promise -- 
theme and variation -- Resurrection and redemption of all believers (investors). They
will bring to material life the dead and
dissolved who remain as memories. But how
is this Ressurection to be accomplished?
Will there be an information-search for the
dissolved and scattered, the transformed,
decayed human material of the world, and
then a sort of reconstruction, a retro-combination? Clearly, the parts, the very atoms,
have memory addresses. Or are the \emph{memory}
elements to be reinfused into dust, and that
dust brought to life, as the Kaballists claimed
to be able to do? Or are the memory elements
inserted in machines? No matter. The point
is that the \emph{thinking} behind these tales descends to us. (Of another transubstantiation
myth which has to do with capital itself, more later.)

These political and social materials, this
kind of thought, woven back into the whole
traditional memory-corpus of dramatic, fictional, religious works, permits us to see that
a trans-disciplinary
whole operated even
in the ancient past, albeit using coded languages. These were preserved, by some, with
consequences for the present. On the other
hand, what was selectively forgotten and
buried? We never get to hear the peasant's
side of the story in the Joseph tale. The
oppressed have no dynastic history. The
concreteness, the mundanity of the past, has
been generalized, its particularity eroded,
just like the start-up equity put into a corpoation. The information of the past shapes
the information processing of the present.

The econometric predictions of this present (think of the Russian-American wheat
deal of 1972-73, the decision to corner the
grain market, extensive planning) uses, with
minor variation, the Josephian scenario. In
fact, to tell the story again, the disaster
Joseph predicted was not a disaster of underproduction, but too many years of overproduction, with attendant deressed prices, requiring either a famine to be manufactured,
or at least an \emph{informational} famine created
by cornering the market, leading to real
hunger. Prices? Is this really in \booktitle{The Bible}?
Joseph, after all, as Pharoah's agent, sold the
surplus grain on the world market during a
world famine. New discourses inserted back
into past events disrupt the holiness of the
memory time-series, and question the legitimacy of modern thought-buildup, indeed,
the legitimacy of all present-based\slash obsessed-with-past, the soft cause-effect linkages built
into history.

\chapter{}

Why raise these questions? To challenge
an obsessional mode of thought which annunciates itself as new and seems to become
more rational every day, but which is a
capital intensive, ghost-haunted complex,
stealing thought and memory away to hoard
it. To attack a long and endless, even boring
set of preoccupations; theme and variation
on a few original musical phrases; a casting
out, an obliteration of cacaphonies. But do
the original themes tell us anything of real
social behavior of the message-senders, the
people who generated them? Can we really
seperate passion from ideas? Even the ambition to succeed can distort ideas. After all,
how do the preoccupations and passions (to
say nothing of the confrontation with, or the
bowing to power) enter the idea-stream?
After all, people have killed one another to
preserve their ideas.

What would happen if we altered the
memory-references to, say, the ancient Greeks,
and their culture
(sculpture, philosophy, logic, myth, history)? What if we
included the practical, day-to-day thought
and considerations of power? This disruption of the sequence, described as a progressive and ascendant evolutionary movement
from the simple to the complex would ripple
back up again to the present, causing glitches
and noise in the pipelines.

When Marx said that the philosophers
had only talked, never done anything, he
was wrong. They indeed did something, for
the propagation of messages requires a climate of belief which they generated. Marx
himself refused to let go of any of this past, merely rewriting the perspective.

Consider: Socrates never answers Thrasymachus satisfactorily, in that order-obsessed
schema, \booktitle{The Republic}. If philosophy is a
sequence, each succeeder building on each
preceder, then philosophy never got off the
ground. We still wait for a response to that
question: power is justice. What followed is
nonsense. On the other hand, the merchants
and politicians, the soldiers listened carefully to Thrasymachus.

And does it mean anything at all that the
great themes sounded by the Athenian playwrights and philsophers, and upon which
the great symphony of western thought is
composed, were all homosexuals, but nevertheless required to mate with women and
replicate? Is there a hidden content, a secret
sexual message in philosophy, a movement
toward body-purified thought? This has
bearing on the question of heterosexual reproduction, the desire to escape the tyranny
of Grand Design-serving matings. A homosexual population generally doesn't replicate; it must recruit. Will it put artificial
reproduction on the agenda? Do dreams of
non-heterosexual reproduction lead to designs for immortality and eternal youth...a
longing for transcendance, a covert desire to
escape the decayable body?

In both \booktitle{The Divine Comedy} and \booktitle{Faust}
there is a seeking to find a route to escape the
bodily and heterosexual mode of reproduction, to seperate the erotic from the reproductive, which begins to lead to algeny.
Faust turns his back on earthly marriage and
love, to mate with a \enquote{female} principle in
heaven, seeking and using knowledge and
deeds in his journey. Dante glimpses Paradiso, seeing shining intelligence and bodiless love. As light\slash love\slash intelligence radiates into space, it sinks into the blackness of
body, matter and sin. Divine love and knowledge are a metaphor sent down the ages.
Alchemical wisdom. But in the past, there
were no major idea-and-body-replicating
devices to carry down the notions of hermeticists, gnostics, Kaballists, magicians.
Heterosexuality was necessary. How to escape this trap? Later, John Von Neuman
dreams of machines, gathering knowledge,
becoming autonomous, reproducing.

To understand the present one must look
over this long series of \emph{regularized} events,
saying in each case, \textquote{this time is like that
time} or \textquote{this time is not like that time.} To
regularize events is to force similarities on
these happenings. This is possible only if
time-segments are everywhen the same. It is
disconcerting to think that each individual
in history was unique, singular, unrepeated
and unrepeatable. How can you retrieve
their thought? How can you resurrect?

Modern, rational thought requires even
greater precision, since we think, and replicate thought through this \emph{capital-intensive}
mode, one which cannot handle true singularities. Order, repeatability, similarity, pattern, structure,
identity is introduced into past sequences, otherwise how can there be
such a thing as a series? Consider the search
for missing links (a medieval, logical technique left over from the invention of the
Great Chain of Being) in order to smooth out
the series of evolution, to eliminate great and
cataclysmic jumps. If we allow these ruptures, we leave room for the reintroduction
of the divine, the inexplicable, wild and truly random chance. Evolutionary frauds,
counterfeit artifacts, faked documents are
manufactured and inserted in the attempt to
create legitimizing historical series where
there were none before, or at least no record
of any. How much easier this is to do with
the computer as it projects, constructs, simulates to fill the unfillable gaps. If we cannot
think without the aid of the computer, then
thought itself is capitalized.

There are several kinds of capital (or value) here. One involves the accretion of
knowledge. Another can be likened to the
yet-to-be-valued good will of an ongoing
enterprise. Good will is an intangible, but it
can be quantified and entered into the account books of an enterprise, then bought
and sold. A third kind is more mundane:
energy ingathered and stored up as wealth,
credit, money, which is information. A
fourth kind is the good antecedants,
the precedants, the legitimizing \enquote{genes.} A fifth
kind is structure; the orderly arrangement of
timed events into a sequence of inevitability:
critical-path operations, program evaluation
and review technique, scheduling ... one
kind of chronology as against other and
disruptive chronologies. All together, they
constitute the strategic program for running
the modern enterprise, the extended \booktitle{Talmud}
of western civilization.

Whatever really happened in \booktitle{Oedipus} and
in the Joseph story is suppressed. The making of the sequence requires rewriting the
elements long after the facts, if there were
any. These tales are supposed to be exemplary, instructional, fragments of an algorithm. The enterprise must be saved from
disruptive thought, from noise, and requires,
first and foremost,
the storage of a transmittable message. It is in that sense that this
sampling of literary\slash mythic works mentioned
here are an intrinsic part of the good will of
this capital-intensive enterprise called Western Civilization.

Now, the chief operating executive in
Thebes, a subsidiary of a diversified banking
and religious consortium called Delphi (also
a data bank and intelligence-gathering operation) was called a king. Oedipus. The
king has broken the rules; he's an anamoly.
In order to prevent the enterprise's demise,
Oedipus must be deposed, the management
changed to demonstrate that the enterprise
continues under the aegis of fate (what \enquote{appears} to be an inevitable series) rather than
individuls. The board of directors meets at
the shrine of Delphi; it is they who plan
Oedipus's deposal, implying retroactively
that not only was it fated, but the cause of the
crisis lay in Oedipus's very genes (as later
Eliot will re-sound this theme, using the key
word, \enquote{Tiresias,} to express the barrenness
and sterility of modern life). The board are
kingmakers. At the same time, this ruling
elite, in order to restructure the trajectory,
also plan a meta-demise, a long-range, exemplary message which will be transmitted
down the ages, a program-scenario for the
sacrifice of kings, managers, chief executive
officers to maintain order and sequence.
This is to be celebrated thousands of times.

What if \emph{all} the characters in \booktitle{Oedipus} were
disgusting, then why bother to preserve that
memory? But in fact, that's what they were:
greedy, grasping, selfish, monstrous, in \emph{no}
way noble, and in that sense archetypical.

If we are dealing with no more than a
revisionist, mythic history of a political
struggle, then the classical Freudian interpretation --- indeed the whole Freudian industry, which needs one interpretation and not others --- loses this stored-up good will.
The credibility of one of the foundations of
our enterprise goes down the drain. In one
version, Oedipus is a hero; Jocasta is not his
mother. In another we see the story of a coup
against Oedipus. In a third, it is Jocasta who
ordered Laius slain. In a fourth we see that
indiscriminate sexual behavior, including
casual incest, is everyday behavior in royal
families. In a fifth we see the play as the
celebration of the coup from the point of
view of Athens; the defeat of Delphi and the
assumption of control by Athens as it struggles to establish hegemony. Ina sixth, we see
Oedipus trying to replicate himself through
incest. If Oedipus is not exemplary, then
good will is devalued. Oedipus could just as
soon be Boss Tweed.

One could make a communications flowchart of these paramemories, trace how this
good will was transmitted down through the
ages by humans who acted as recievers, repeaters, relays, enhancers, gates, transformers, noise-eliminators, interpolaters, adders,
switches, coders and decoders, error-correctors. They recieved
and sent these stories
down by voice, in writing, or by use of rites,
chants, liturgies, ceremonies, dramas, storing them in any variety of devices. From
time to time they were retrieved to be used as
guidelines to correct present and future behavior.
At the same time, humans themselves, as biological creatures, also transmitted a different kind of information and
memory: genetic messages. Alongside these two
streams, treasures, credit, good will, capital was sent. The memory of a memory:
one, events; the second, biology; the third,
capital. Built in was an adjustable, timesequencing rescheduler for calender reform.
None of the tracks can exist without the
other.

All forms of knowledge intertwine to pass
down a climate of opinion, a meta-environment, which becomes part of the present
percieved \emph{physical} environment. Indeed, as
futurists --- equipment-sellers all --- talk about
the next evolutionary step into the information age --- which is also a whole environment --- and annunciate this adaptation to this new climate, \emph{property} seems to become
less physical and more ephemeral. The
burden of our argument becomes clearer:
inheritance.

(For example: the one subject, the true,
corporate --- or embodied --- \enquote{hero} of all of
Dickens's works is Inheritance. His characters are always involved with claims on
Inheritance. Inheritance as meta-genetics,
manifests itself into shells called humans, or
characters. The role of Dickensian characters
is to move Inheritance through history. Each
one of Dickens's novels involves a search for
a programming error, which, when corrected,
allows for the continuity of inheritance outside of the fates of the characters involved.
To use inheritance self-indulgently is to
descend into sin.)

The newest version of these old inheritance stories is sociobiology, a kind of biodeterministic Calvinism. Into the observation of nature are inserted these birth-mythologies and breeding-and-replication logics
of ancient Israel, Greece, Egypt, Rome, and
the consanguinity-obsessed Middle Ages.
The carrying down of these treasures, saved
from the ruins of shattered civilizations,
finds its way into modern myths of adaptability and evolution, even within the present
twenty year span. To this theme is added
inevitable causality.

As an asset to Western Civilization, what
kinds of valuation can be placed on these
long gone events? How can they be calculated into the asset picture? One must begin
by reviewing, assessing, quantifying and
valuing these intangibles, these pools of
good will. How do these carry-forwards contribute to the development of rational calculation in its newest, computer-assisted modes,
translated into assets and liabilities? What
distortions, mythologies, religious superstitions have crept in and how did they get
there? Or, since everything can be informationalized (if specified) and assigned a currency value, does it matter?

\chapter{}

Enter cliometrics. If legend, folktale, myth
is a form of information and meta-history,
featuring un-individualized heros as variables, then cliometrics is meta-meta-history --- 
the kind that makes it easier to match events
up to number-series and can incorporate
genetics-as-coded-logic --- which itself can
be used as a kind of history (as history can be
used as a kind of genetics). Each particular
person, each event generated becomes a variable in a statistical series. Chosen events in
a time series are matched up to chosen sets of
self-transmitting individuals. Cliometrics is
a sub-branch of those statistical time-series
that computer archeologists are so found of
retrojecting back to creation in the attempt to
make history compatible with logical operations. Cliometrics is quantified history in
which people and events are graded along
an importance-scale and matches up to a
time-scale established by archeology and
evolution. Cliometrics, more than any other
kind of historiography, is computer-compatible history.

But then, ascent-mythology reenters. All
event-clusters on an evolutionary series are
in fact fungible. The billions-years-trajectory
is mapped onto a short stretch of human
history, just as an infinity of natural whole
numbers can be, say, mapped onto all the
possible fractions between one and two. But
like any series, such a construct depends on
the measurable regularity of events. Catastrophe, the unexpected, must be integrated,
smoothed out. The catastrophe is cut apart into small segments, the compression stretched
out and we see that event in regularized, even
fragments. Why evolution --- which is supposed to move in glacially incremental
steps --- is matched up to history --- which
takes place in only a short time --- is an
obsession --- progress, growth, incremental
gain --- that the 19th century foisted on us.
Humans have not changed, not evolved, in
all the time they have been on earth.

Then the proposition that evolution, as
applied to humans and their history, is the
movement from simple to complex, from
less to more, from worse to better, from chaos
to order, from primal virus to sub-human to
human to angelic to god, doesn't work.
Whereas the latest studies are purported to
demonstrate that more than fifty percent of
the economy is information work, a case can
be made that whatever the conditions, peasant, primitive,
factory worker, more than
fifty percent of the work was \emph{always} informational. (Which intrudes a touch of the
miraculous, for when, how and with what
blinding speed, did humans become the way
they are?)

The very act of selection, grading and
valuation, in order to constitute part of the
latest national rememberance-treasury of
our enterprise, to make it fit to be good will,
it must now be said that these events are
convertible into these numbers, these increments. For it is part of the equity, the capitalization placed retroactively into the start
up of the enterprise; a set of past events
converted into numbered, stored and convertable values. After all, even eventless,
invested capital must have some history. In
any pool of money, there lurks an implicit
collage of time series.

Now if cliometrics is to have any usefulness, not only must religious history and
mythology be quantified and arranged into
an accretive time series, but it must be valued. Once quantified they may be translated
into other metrical disciplines through a
complex series of conversions shuttling
back and forth in time. (Although, before
this can happen, a rewriting and re-evaluation of the meaning oftime itself is required.)

\chapter{}

Starting from an other end, we can look again at the hypothetical annual report and
balance sheet of our enterprise, Western Civilization. If we analyze, concretize, qualify, re-view, individualize or personify the
bottom line, representing pools of capital, we unfold it to see a variety of sub-enterprises: factories, farms, plantations, mining
operations, shipping, workers, slaves, wars, thievery, pieces of drama, the addition of last
year's, last century's, last millenium's profits or losses, the whole ensemble of economic
activity. These pools of money and nearmoney (an incredible array of quasi-liquid
exchangables) have different key words, titles, identifiers, instructions and names attached to them, indicating by what rules
they may be converted from one form into another: numbers
to numbers, numbers to people, people to events, events to a varied range of goods...

But to repersonify, re-invent, decrypt
events out of our bottom line puts limits on
their exchange potentials by pointing to sets
of \emph{particular} people, their lives, energy expenditures, products. They are not given to
us in the proper language. It is hard to tradea
human being (or a ghost) \emph{directly} in our
society, with its putative committment to
humanist values (each and every human in
the world is supposed to be of importance
... but not their work). Yet, the abstracted,
refined, distilled, symbolized value of part
of a human, their work --- with the rest cast
off --- can be stored, calculated and traded
every day. Modern purification ceremonies.

The coming of high-speed communications, computers and trading programs allows for the movement of keyworded, encrypted bundles of capital across barriers
and borders, to be entered as a profit here ina
tax haven (a souls bereft of bodies ascend to
heaven), a loss in a taxed sub-world: expensive signals of expenses. \emph{All} forms of credit
are highly abstract, general-purpose information. Otherwise, how could they be moved
and interfused? (Citibank, Chase, the Bank
for International Settlements, Banco Ambrosiano, the IMF, the Federal Reserve System,
etc., all banks and securities houses are
information-handling, transactions and communications companies.) Information cannot be too particularized in order to make
entries in account books. Generalized, it can
be teleported easily through the most complex
spaces (where humans cannot be moved),
some of which are not so much spaces but
representations of spaces, magnetic perturbations, addresses standing for a country,
another enterprise, files on a tape or disc, or
in flight, entries in an account book. I can
transmit six million dollars (how many
lives; how many things?) froma U.S. bank to
a Japanese bank, which has a branch (a
presence, but not a nescessarily a facility; a
designation) in Panama --- move it from a
ledger marked \enquote{U.S.\slash Chase} to a ledger
marked \enquote{Panama\slash Bank of Tokyo.} Has the
money traveled? What does the question
even mean? Yet the U.S., Panama, and Bank
of Tokyo behave as if it has.

If we intone \enquote{Russia,} six letters, we have
some idea of what that means: the letters, or
the sounded word, stands for the space in the
file and that space, on tape or disc, matches
up to the whole of Russia. (The code-name,
the identifier, the recognition-signal, legitimizes the transaction. It must be precise. We
cannot say we are sending \enquote{Oedipus,} for
what does that mean? Or can we?

At bottom, these forms represent the activities of people in the present, the past, and
the future (when capital was/is/will be
stored, built up, accreted). Money contains a
past: it is a memory and energy-storage. The
labor of old machines (animals, plants, and
humans) which once converted sun, soil and
seed into food, raw materials into things; all
these and more are implicitly represented.
The credits are a form of miraculous, nonenergetic, energy storage.

We have reached a fascinating metaphysicality; energy storages which are not contained in material objects... Wood, coal, oil... these we can understand. With electricity, things become problematical. Electricity
cannot be stored (except in limited ways;
batteries). We must produce heat in order to
drive the generators. But yet there are abstract,
informational, non-heat, non-energized, \emph{symbolic} storage devices which nevertheless
have the power to drive the energizer of machinery,
the human. Money, securities, paper instruments of all kinds,
electronic signals... all these (among others) have this power.
But in order to energize they must recieve this power, which is purely a symbolizing activity which must take place in the context of a whole social climate which trains people to respond to the energy-stimulating information associated (not in) these devices, these instruments, these fictions. We are talking, in the long run, about belief-systems.

History resides, embedded in each item we touch. A set of ghosts... remnants and memories of people who worked to produce money, wealth. Looking at a pool of credit, we are not permitted to infer the concrete existences, the lives, the sufferings, the pleasures now, of those pasts, this \emph{unspecified} stuff of phantoms. This analysis holds if we believe that capital is a residue, expropriated and stored work-energy, value and time (representing the expenditure of human energy, lives spent in forced labor). Electronic gold is the newest minimalist form. Capital represents a compressed work-and-time series omnipresent in credit and every processed or owned thing. A continuity of real and fictional people. (Why fictional? Information doesn't have to be about real things or people; it only has to be \emph{accepted} as real.) A revived theory of spirits.

But when we come to the creation of
credit --- as the material representation of embedded
value, gold, gems, declines in importance --- by banks (real banks, or fraudulent shell-game banks, and all the interest-leveraging games they play, lending five,
ten, twenty, a hundred times their assets; national treasuries with their printing presses
running full-time, account-book manipulations), we have arrived at the manufacture of
value out of nothing. If credit, value are
linked to people (genetic series: or their
residues, for they continue to work when
dead) and the lives they once lived, or might
have lived, then inflationary cycles create
even more fictional people, fictional lives
which impinge upon and crowd the living.
These humans and their energies seem to be
drawn out of the \emph{future} (looking at interest as
a price based on usage and future realization, as expressed in money and time, which
is to say energy-expenditure --- labor --- to be
realized) and become another way of birthing fictional people and enterprises.

Hence, in this world all the forms of credit
(information), if convertable to the lives of
populations, create \emph{population pressures} on
a limited environment, recrowding the past,
or emptying the future. They, these \emph{prespirits}, impinge on the economy and on
daily life and the psyche. Fantastical?

If dreams and myths have an effect on the
lives of people, it is clear that metadreams
and metamyths also have an effect. (What's a % make this a footnote
metadream? The purest form is capital, information without specification. To take
another example from other realms of discourse; the score of a psychological test,
which converts psychic states into numbers.
This involves the pricing of the tests plus the
start-up costs for developing the tests, getting people to believe in its validity, and the
investment in the formation of some psyche
or intelligence-testing company, and the
sales of such test, and so forth. None of this
mentions cohorts of theoreticians who develop the background theories of human
nature. What are we left with? A disembodied and quantified psyche.) Such manipulations violate the first law of thermodynamics, creating energy out of nothing or
the future. You can violate the laws of nature, at least for a while, if you put enough
money into it... So said the physicist, I.I.
Rabi, clearly identifying money and energy as one.

These labor\slash life\slash time\slash suffering residues,
these generalized phantoms are peculiar
ones. They have no meaning until the underwriters of an enterprise
assign a value, a meaning to them. Assigning a value (and raising money against it) requires an act of
confidence and the general acceptance of
that assignation (which itself must be sold)
is what constitutes the general act of faith in
certain people's notions of present, past, the
invisible, the future. Credit, \emph{credo}. You can
invest in it, you can bank on it, you can buy a
piece of the faith and people will be moved
by it. By what? By this long series of conversions,
metamorphoses, transubstantiations
and appearances out of pure space, dyings
and rebirths, incorporating a kind of serial
cannibalism among its many charms. These
acts of faith in the modern, rational age, are
not merely cool and detached. They are
attended by great excitations, curious passions and lusts, unseemly fights for status,
murders, massacres, tortures, annihilations of
populations, pomp, circumstance, exchanges
of gifts, drink, drugs, bribes, kickbacks, display, ostentation, treachery, thievery, and so
forth.

Massed and abstracted capital represents
not only a history but populations and their
biographies. Economic archeology: capitalist historians, pushing progress, are counter-
intelligence archeologists. They rewrite their
history to deny that the suffering of populations ever took place, just as we deny the spirit
world of \enquote{primitives} and their reverence
for ancestors. We don't revere ancestors; we
revere ancestry: inheritance, not inheritors.

If you averaged out capital buildup per
person per life, established historians, rewriting history, tell us that things were becoming better and better during the industrial revolution. Concentration of wealth is
ignored. The invention of a class perspective created an oppositional classification,
revealing one population's misery. But even
Marx accepted these constructs; the evolutionary inevitable. He only wanted to change
direction once these \enquote{necessary} sacrifices
had been made. Both Marxists and capitalists have played the same kind of game.
They didn't challenge the basic, agreed-upon, evolutionary-schemata, the concept of
capital itself.

\chapter{}

If one could invent (or draw out of some
infinite account) values created by populations yet to be born --- fictional people, which
might include robots and simulations --- those
populations have a real effect on our present.
What \emph{space} do they live in? If they're fictional, how can we feel their presence here
and now? Or if they are far away, even in the future, how can we feel their impingement,
as if they were next to us? They exert mysterious influences through myths and interest rates, like the gravitational perturbations
coming at us from past and future,
annunciating the presence of an unseen
planet or star. We search for them. They can
be found by indirect methods; relating population to space, converting the numbers of
those-yet-to-suffer into space time and people. This is done by the art of studying
population pressures in the present, for new
populations are being unleashed upon our
planet, teleported to us, backward
in time, emerging through the central processing
units of financial institutions.

Segmented demographics (studies of populations
in abstracted, discontiguous home territories, stacked up, perceived, recorded,
retrievable, quite as if they were in some mass
memory; memorialized particle-clusters, when networked, make their territories contiguous, while lack of communications-access makes the land of the poor quite seperated) allow us to play new games with the
location of populations, to move them from place to place, even into invented countries.
For example the Nielsen rating's \cA country, \cB country, \cC country. (There is a \cD country
but that is discounted. Dante also had an \cA, \cB, and \cC country and, although he didn't
write about it, a \cD country; for Dante never wrote about the poor and the ordinary.)
People in \cA country can be scattered all over
the world; they don't even have to live in
proximity.

How were these populations get into these demographies? If we have totalled up real people --- hungering, sweating, copulating --- then we need real geography, at least at some point in time... perhaps in the past. With the advent of the need for targeted, reachable populations, demographic regions were invented, related, classified, archtypalized along buying\slash class\slash income\slash taste\slash interest\slash communication lines. We see continents dissolve into an archipelago. (Dante preferred to stack them in cones and concentric, high-velocity circles.) All nations have --- at least as far as some of their populations who live in transnational space --- dissolved.

Corporations transcend boundaries. Financial institutions hurl money across timespace. Pools of wealth flee the constraints of national taxation. Russian and Americans
bank in off-shore havens. The French and Germans invest in Russia and the U.S. I.G.
Farben, Dupont, Imperial Chemical and
Mitsui had an explosives cartel throughout
the Second World War. What does nationspace mean to them? They live in Paradiso.
They absorb the energies of dead Germans,
Americans, British, Japanese. This subset of
French, German, etc., constitute a cultural
homogeneity.

Such an approach generates endless lists
for direct-mail and political polling, possibly electro-politics. One could even poll
\emph{the compilations of taste-lists} (which would
contain significant sociological, psychological and bio-medical data simulating humans),
a thing the computer does very well. The
information age becomes more scholastic
and magical.

Given these geographies --- these weird
topologies inhabited by real and fictional
populations --- questions are raised: are we
merely playing an intellectual game, or do
these fictional people have some substance?
Do they have a history? How were these
list-populations generated?

In his extremely information-oriented,
mechanistic, Laplacian \emph{Sociobiology} (genetics percieved not only as information, but as
the motivators of every human act), E.O. Wilson proposes the notion of the feeding
capacity of an environment, a region, a land, a territory, in relation to the population of
animals, or humans. The relationship is called the density-dependence
ratio. The amount of grass to cows, cows to humans, for
example, in a set amount of geography. This
limits the feeding capacity of cows and humans. If one factors \emph{percieved need} the ratio
changes in a non-rational way. When class, and all that term entails (culture, for instance), one must erect a set of densitydependence equations and ratios for hierarchies of populations, relating masses of
fictional capital to fictional populations to
real populations, since capital needs to feed
and be fed... and all of these to assigned to certain confined
spaces, which constitutes mass. We are beginning to talk about frame-references of space, energy, time, money, genetics (both real and invented).

Grass converts sunlight; cows convert grass; humans convert cows by ingesting
and transforming them. Commodity brokers, grain companies,
agribusinesses, food-processors, chemical companies (for preservatives and fertilizers), refrigeration specialists, distributers, railroads, truckers and
unions, politicians and bankers... intervene beween humans and their eating, seperating mouth from food, by inventing multiplexed and involuted distances which are a function of pricing through two kinds of
intersecting transportation systems: physical transportation of goods and transportation of money, invoices... The complexities of credit and time, in all of their forms --- commodities, options, future, indices --- creates
conversionary mediation-complexes and lengthens distance. The need of a loan in some distant place creates a shortage of liquidity on the local land, a sort of drought,
symbolically and effectively equivalent to a plague of grasshoppers.

The more capital there is, the more population. A small, capital-intensive poulation consuming vast amounts of capital-intensive food with the aid of technology becomes equivalent to a huge population, crowding out the living from earthly space by the reproduction of hungering ghosts. No wonder the ancients fed ghosts, spirits, gods. And given cliometrics, demographics and the density-dependence equations, considerations of capital, telematic acceleration, surely we can find these antecedant considerations somewhere in our ancient texts.

We could estimate the amount of food,
perhaps in calories, required to feed each
individual, averaging consumption, and set
a provisional standard for existence. But this
is mere munch-democracy. It's not the way
things work. Given real considerations of
power and concentration, much capital never
even reaches the earth, but remains in
perpetual transit in the empyrean, going
from countryless account to countryless account,
money being invested in money (which embodies time) for high interest rates
and other profitable instrumentalities, passing through odd logic gates and mythic
addresses. Some populations not only eat for themselves, but for whole hordes. The bankers have, in fact, invented hyperspace concurrently with the physicists and have their
own white and black holes. In short, relatively, we are speaking of gods. All they lack
is actual, as against relative, immortality.

Wanting to know what his population is all about (consider a heaven or hell, full of
ghosts, phantoms, spirits; the dead and their claims upon the living; the exploitation of
the dead) we can, by decoding, assign characteristics to this population. We can ressurect the dead, or the unborn, though not, unfortunately, with their bodies intact. We
can write histories. Suppose the recorders of life, the gossip columnists, the journalists, sat down and looked, along with the poet, novelist and biographer, over the bottom
line of our enterprise's balance sheet, instead of looking at the illusory world around
them. If our premise is right, why couldn't they construct (or deconstruct)
a novel, a history, a biography from the numbers? How?

For example, one could use psychological, medical testing and diagnosis, to record

and store up the signs of the living. We can
get readings on medical machines: lie-detectors, EEG's, EKG's, stress-analyzers, voice
prints, PET scanners, CAT scanners, NMR
machines, varieties of brain scanners, and
other kinds of instrumentation. We can
create electronic simulations that fluctuate,
as if alive. We can store those records electronically (at a price). We can establish a
constant link between body and record (itself fragile, almost \enquote{alive}) so that as the
body changes, the record changes. Each change alerts the medical monitoring machine to take action. The record becomes an
analogue of the body. If the development of
medicine has been correct (problematical)
and the translations of bodily qualities into
signs, numbers and waves is correct, and the
instrumentation is correct, the simulation
begins to approach autonomy (as long as the
electricity is on; doubtful in these days).
Now presumably this body-to-record-and-recorder can be reversed, so that when the
records change (the simulation is given a
disease and then transmits it to the body),
the state of the body changes. We can invent
whole library-demographics of electronically
simulated bodies. We can also put them in
synch with the state of the world's economy.
So that a change in the numbers in a complex
of account sheets would change the health of
the simulated population and the health of
the simulated population would affect the
health of the world... And, indeed, isn't that
what happens in the case of the sacred king?
Oedipus?

We can abstract and average out the readings; we can even feed back pre-recorded
physiological data into this ghost population and give it life. We can appropriate the
stored-up work-effort of the sports athlete,
the assembly-line athlete, the drug-consuming athlete, the neurosis-athlete (who struggles to produce new records in psychosis:
anguish indices). Why, pain itself can be
telemetered on different scales, even stored
electronically in these computerized telemedical systems and transmitted, when
needed, through wire or satellite, broadcast
to far distant places. After all, the electrical
\emph{instruction} to a prod applied to the genitalia
doesn't have to be administered directly by a
finger pushing the button that turns on the
juice, button connected to finger connected
to person in the same room with the tortured.
The amounts of current can be calculated
from far away (assessed by previous testings)
and be transmitted by satellite to some
banana republic, just as money (compressed
and massed lives, sufferings) is transmitted
every day through SWIFT.

We can even poll the indicators to find out
the wishes of this phantom population. In
fact no one has to poll real people, just query
the taste and market-cluster data-banks containing the repository of psychosocial desires and physiological indicators. It is already clear that voting electronically can be
like ballot-box stuffing, using votes of the
dead, since the powerful make the powerless
a transmitter and talk only to themselves. If a
big stockholder votes his shares, why can't
he vote his ghosts?

When we move into these realms of abstraction, we can play any game we want
with the indicators, the concepts, the \enquote{stand- fors,} just as long as there is consensual
agreement to credit, honor, have faith in
these acts. And this is what's happening.

In Kaballistic thought, the elements of a
language, ordered one way, reveals one
world. Ordered another way, a completely
different world with different laws emerges.
Language precedes things and humans in
enostic thought. Even history itself can be
restructured (along with geography) in interesting ways. Events in \emph{Oedipus} can be
placed next to the events in Joseph's Egypt
(since they are about agricultural disaster),
even made concurrent with a cash-flow
crisis in the U.S., and related to the collectivization period in the Soviet Union, and
to link them up we have the marriage of IBM,
Comsat General and Aetna into Satellite
Business
Systems linked by Systems Network Architecture. After all, wasn't it Delphi's and Joseph's monopolization of knowledge that caused all the trouble?

\chapter{}% 13

We have been moving from time-series to
simultaneities. Serial and synchronous time
threaten to become surreal time.

Speed and distance are functions of time.
In the world of linked up computers, messages move faster at the center than at the
peripheries where messages move an entirely different way. What's the center? One
can propose a model: a set of rings. Messages
in the inner ring move fastest: less distance
to travel. Messages to and from the outer ring
move slower. Dante's model. This is a conceptual device that expresses the state of
communications today. However the center
is in fact spread and networked all over the
world. It is faster, for, say, Citicorp to get a
message to Hong Kong from Lexington Avenue, than it is to deliver a message across
Manhattan walking, riding a bicycle or taking a taxi. Citicorp-Hongkong is a center:
9,000 miles. Lexington Avenue-Eighth Avenue is a periphery: 1/2 mile.

When we take into account pricing and
power, the the problem becomes even more
complicated when the message traffic has to
go through some center or complex of centers. It is asserted that if everyone is linked
up by interactive terminals and microcomputers, this blazing center of knowledge will
be available to all. This is nonsense. In the
real world competetive advantage depends
on your opponent's being relatively ignorant. We're not even beginning to talk about
price and the horrendous effects, in the U.S.,
of the AT\&T divestiture. Prices of computers
go down: this is true. But prices of communications not only go up, but will be unavailable to a large group of people. And anyway,
one has to reeducate oneself to use these
clumsy machines.

If we are to make a transition to the information economy, in which information is a
certain kind of currency, certain steps must
be taken. Treasure is meaningless if everyone has it. Treasure, and every good, has
built into it a political and business version
of the second law of thermodynamics. Maxwellian demons concentrate treasure, energy
and information. These are shrunk, massed,
concentrated into smaller and smaller class-spaces. When knowledge becomes treasure,
the value of it is meaningless if everyone has
it. But there's a problem. The spread of
information is limitless. If we tell a number
of people something, then they all have it. So
the purpose of the information revolution is
to put a value, a price on information and
add to the rituals of learning by technologizing it so that few may have it. In the context
of the present attempt to make the grand
transition to this new era, we have come to
see what this means. It is a way of recapitalizing the past and to undo what Lucifer or
Prometheus did. Think of the whole complex of modern telematics as one, gigantic,
central, country-spanning intelligence and
counter-intelligence agency. It also means
that everyone outside this informtion economy is doomed, and that, perhaps, is half the
world's population. This is important to remember.

It is said that the speed of generating and
processing messages inside of a computer
may be faster than in the human brain. That's
one way of looking at it. But, in fact, the
permissible messages, their content and
form, in a computer are enormously different than the message traffic inside of a
brain, especially if one considers the development costs (which are in their way a
function of time and energy).

The application of abstraction to things or
people creates problems. One can say two,
four, six...: obviously the next number
should be eight. But we can also pick any
number at all and make that the next step
after six, and invent a logical proof for that
choice. A logical proof can be invented to
justify \emph{any} arrangement. (We are moving
toward a consideration of time-series in a
modern, quantumized, relativized, financial,
informationalized context.)

There are values, variables, with a multiplicity of identifiers, from different yet convergant frameworks, assigned to the stored-up residues of past, present and future human activity. It may be a genetic identifier, a
financial identifier, a cliometrical identifier,
a literary identifier, a physical identifier.
The arrangements of history and the sequence
of the buildup of capital of all sorts (taking
into account the falsified and adjustive historiography as common practice: for instance,
CIA or Church historiography) is somewhat
like a problem in scheduling information
traffic in a computer. It must be controlled by
timers managing the sub-routines, moving
and saving bytes, using loops, querying the
memory, all contributing to the flow of traffic,
done as events happen, after events happen,
before events happen; a sort of time-travel.
Given something abstracted, but accepted as
an act of faith and so lived-by, as a pool of
credit, one can fill in any history one wants.

But in order to do so requires that one
overcome deviant memories and histories.
One has to fight to control the history, its
event, its passions, its humans, its meaning.
This we surely know: people died miserable
to contribute to that pool. Defining the
meaning of that pool becomes a political and
ideological fight over good will. The winner
writes history.

The derivation or invention of any series
takes place both in historical contexts and
according to \enquote{deeper needs.} But these
\enquote{deeper needs} are not to be found in nature,
or \enquote{Man,} but are the shared desires of a
small part of the world's population who
constantly fine-tunes the ancient methodologies of series\slash simultaneity-making. The
\enquote{facts} --- whatever those are --- or processable
specifications, establishes a background
theory for those \enquote{facts.} The accumulation
of many forms of capital is required, each as
a contribution to the information economy,
for we are no longer in that age when the
wishes, ceremonies, sacrifices and incantations of priests and shamans seemed to control the universe: although the sacrifices still continue.

For capital to be accreted and stored, there
must have been sets of people arrayed in
some time-sequence, laboring to build it up
(and also wasting it) during the historic
process of production, circulation, consumption, storage and reproduction for that subset of humans who are series-makers and rememberancers. Certain goods may have decayed, but they can still be stored eternally, retrieved, called up, as information.

There's a limit to how long actual grain
can be stored but there's no limit to how long
we can store the abstractions standing for the
grain. It is possible to sell a ton of grain
harvested in Pharoahonic times now. The
only thing is that it cannot be \emph{eaten}, only
bought and sold perpetually. If the buyer
and seller agree, one can sell the Pharoahonic grain and use the money to buy real
grain. Perhaps it is only the designator,
\enquote{Pharoahonic grain} which throws us. Can't
we sell a cargo of grain a thousand times,
symbolically moving it from port to port
without that cargo actually moving?

At issue is the relation of symbols, information to the non-informational world. What
happens if the informational world collapses?
Panics, depressions, bubbles, inflation are
all \emph{informational} collapses. The non-existant
crowds out the living.

If we have a pool of symbolic capital,
which stands for, and is used for, stored
energy, stored value, stored time, stored
space, dreams and aspirations, then we implicitly have an accompanying population-continuity and \emph{population-simultaneity}. It
may be fictional but can also be considered
a storage of real and fictional genetic sequences. We may consider how real people
adapt to their changing environments, but
we must also think about how fictional
populations adapt to material environments
and how real populations adapt to fictional
environments. For if they are valued, their
fictional lives impinge on the lives of the
truly living.

What sort of time-sequence-storage does a
genetic sequence in any one human represent? What we are supposed to have is life,
enormously compressed, a serial simultanized, represented by pools of credit. The
pools of credit are as folded up as any
crumpled helix of genestrings. And if the
production of engineered humans becomes
possible --- given enough money (taken from
where) to suspend the laws of nature --- capital and genetics can be compared, even
equated. A look at the bio-engineering markets is in order. Where do these fictional
populations \enquote{live}? On everted globes, on
satellites and space colonies, or ribbon
planets, in chip architecture, on paradisical
islands before, beyond or at the end of time
itself? What operations must we do with
these time series\slash simultaneities, these lives,
real and false? But what's time?

We have been bound by several perceptions of time, subject to various revisions.
We have been tied to the tyrannous cycle of
ageing, risings and settings of suns, rounds
of seasons (and seen the priests control those
rounds, inserting themselves between us
and the sky), birth, growth, death: \emph{felt} duration. Our biological clocks can be fooled.

The perception of time became industrial
gradually, introduced in the 15th century or
so. Time's continuity was fragmented into
equal lengths, matched up against factory
and production ties; unit time, unit goods,
unit prices, unit consumption, units of exchange, but all arranged into the cheerful,
progressive, accumultory one-way-up trajectory. This vision was introjected into the
conciousness of those inhabiting the industrializing world. It is being introduced now
into the consciousness of those inhabiting
the underdeveloped world.

Time zones were created in relation to the
sun's passage, marking the business day and
year: market time. But all renegotiated timeschema retained this long range trajectory,
the primal beginning and the ultimate end.

Enter, just before the industrial revolution, 
the modern magicians. First wave: mathematicians, scientists, logicians, topologists
(and technicians), the Founding Fathers of the New Age, \emph{circa} the 17th and 18th century... followed quickly by accountants and business topologists, the time and money managers.

But Leibniz and Descartes were primarily
mystics: Galileo faked the results of experiments. As for Newton, the evidence is that he
was more interested in gnostic\slash astrological\slash alchemical\slash hermetic thought than science.
In astrological thought, for the stars to affect
life, and conversely, \emph{instantaneous} transmission of forces are required. Perhaps for Newton the enterprise of regularizing the universe was required to give a sound and
calculable foundation to astrology. The astrological requires order and regularity as
well as an orderly medium for the transmission of heavenly signals affecting human
life, thought and destiny. Newton tried to
formulate a precise scientific methodology
for dating events, using Scripture and Greek
myths. For Newton, time was teleological.
He related time to a history of royal, Hebraic
dynasties. He matched up time, considered
abstractly, to a special kind of ethnic\slash dynastic genetics (although he didn't use those
words). He felt that the ancient Jews had
secret knowledge which filtered down to the
Pythagoreans. He considered the music of
the spheres a metaphor for the law of gravity.
He believed that the dimensions and configuration of Solomon's Temple concealed alchemical formulae which corresponded to a
divine unity in nature. He explored sacred
geometry, practised alchemy (along with
Robert Boyle), and was of course that perfect
kind of compulsive dualist in all things.
Newton was also alchemically and financially involved with gold; he was Master of
The Mint. Given this, Newton's \enquote{beginning}
is religious, extrapolated to Nature.

Or maybe he wanted regularity and predictability because he lost money in speculation.

How much better than Velikovsky was Newton?

It was this complex of thought upon
which the reconstructions in relativity and
quantum physics are based.

With the introduction of artificial light, divisions into day and night begin to end. With
sealed, climate-controlled
environments,
the seasons begin to become irrelevant. the
conversion of the natural world into the artificial world, from the raw
to the processed, continues. For some the world is
already the atemporal control room of a
space ship where the ever-chilled, perpetually running, energy-consuming computers,
spinning out their fantasies, are attended.

\chapter{}

Einstein's thoughts (building on and reconstructing Newton, the alchemist's thought)
on simultaneity become instructive, as do
the thoughts of modern bankers as they scan
their electronic spread-sheets. For bankers
and physicists meditate on states of simultaneity, relative to those who do not, \emph{cannot}
have access to this distance-insensitive equipment. This is to say bankers enjoy abstract
immortality in relation to the banked, along
with quantum physicists and metaphysicians
and magicians.

For Einstein a number of events can be
considered to be taking place at the same
time, but only in relation to an observor
recording different signals from those events
and timing them with coordinated clocks.
(Compare the delicate timing of an international
arbitrage operation which requires relative speed, or relative simultaneity, and
at the same time requires relative ignorance ---
or relative distance --- on the part of one's opponents.) Everything depends on the observor and the recording instruments and
relative accumulations of knowledge (the instruments being a manifestation of assumptions built into them) and speeds of transmission to determine what looks like simultaneity.

Simultaneity has nothing to do with where
those events are happening (unless you are
being shot at from two directions), but rather
when they are percieved. Perception is a
function of distance and the tools required to
transmit knowledge. A corporation, our
chimera, in the \emph{contemplation} of law, this
society of the anonymous and hidden, an
organic being (more than the sum of activities of individuals and groups), with
stored-up misery-and-energy, lives converted
to credit, may be scattered over a wide
geography and over time. All things in it and
of it and about it may be considered simultaneous from the point of view of an auditor
contemplating this balance sheet.
When the accountant's sheet is computerized, constantly recieving information from all over
the world and interconnected with the product of statistical projections from all over
space and time, then all operations are happening simultaneously, including invested-in future events, since they are handled in the present.

Similarly, it is said that any complex of
genetic material contains the complete history of the organism and all preceding organisms (a genetic archeology, but still
alive). By the proscess of combination and
recombination of its memory elements, amino
acids, etc., since it contains all past, probable, potential and possible organisms, even
pre-organisms; and like the elements of, the
combinatorial rules of, say English (rules
which are derived after the languge is mature), is subject to the same operations, containing all past and future literary works,
even those that Shakespeare forgot to write.
Implicit in all these observational operations
is the notion of the simultaneity-observor of
all genetic and\slash or linguistic, and\slash or monetary
possibility. And thus genetics, if seen as
information, is the rememberance of things
past (all the organic --- and inorganic --- universe, all the realized and unrealized beings,
objects and forces) which generate the organisms which invent various schema of
rembererance, which then remember the complex that recollects it.

But for Einstein there was an upper limit:
the speed of light. The quantum physicists,
introducing indeterminacy, and the intervention of the observor and his instrumentation,
implied that in a certain sense, all
events were in fact simultaneous, regardless
of clocks, for they had distributed Mind into
the universe, a notion Einstein rejected.

Let's take a side trip to paradise and consider time there. What happens before the
beginning, or after the end, is a question to
which Kaballists and gnostics address themselves. This problem has been transmitted
right down into modern times in a new form:
what happened \emph{before} The Beginning, the
Big Bang (itself a construct open to doubt)?

\emph{All} matter, energy, space and time (and
thus all possibility) was contracted into a
dimensionless point (or nothingness) ... so
the mythic tale goes. Infinite mass, for if the
point was suseptible to measurement,
even of the most miniscule kind, then its mass
was less than infinite. If all matter, space,
motion, energy, time --- and potentially, perhaps inevitably, all life --- in the universe was
massed into this dimensionless point, then
there could be no time or space \emph{outside} this
dimensionless, infinite-massed point. Infinitely compressed matter and energy (which
included, potentiality, all life) meant that
there was no one to measure it, not even any
automated measuring tools. So all Mind was
there too. Perfect simultaneity, \emph{but not objective simultaneity}, which, being involved
with signals, distance and time, couldn't be
measured. It was therefore eternal inside the
dimensionless point, a feature of all paradises.

Time is not a term that stands by itself, nor
does any other term: all terms are multireferential, bootstrapping every other term
into the air where it hovers like some plasma, contained away from the apprehension
of most people. But the multi-referentiality
(as well as the breakdown into terms of a
complex of things) indicates the intervention of mind inside the universe. If Mind is
an emanation of that pre-moment --- physical, chemical, biological, monetary, literary,
religious --- it might be possible to \emph{remember}
this paradisical past now, as we \emph{remember},
say, Eden, a lower-level and later paradise. It
is out of logic, the mathematics, the equations --- a form of metaphorical activity --- the
statistical retrojection --- a specialized and
leached-out form of memory --- and the observation of the distribution of matter in space,
and the Red Shift, that we might infer, and remember, the Big Bang.

It is over this Big Bang that modern physicists, ancient priests and shamans, magicians, caballists and communications artists come together.

Given this compulsive activity, this associative, similizing, metaphoric, organizing, disorganizing, ordering, concentrating,
distributing, interventionary, aesthetic activity of humans, and the compulsive nescessity
to lay on one hundred and forty four interpolations where there appears the slightest gap
(indeed to invent gaps) we can also say that
all economic observations contain the purified metaphysical and stored-up simultaneous record of all activity. And we can add
that the genetic material is not only an
information-analogue, but a factory and
clock analogue: the gamete's growing becomes an analogy to the ever repeated evolution, a biological mini-Big Bang.

Since all credit is meaningless unless
linked to an active, perpetually moving
market, linked to people to believe in it and
to the institutions they inhabit, given these
massive flows of \emph{perceptions}, symbols and
signs standing for life, space, factories, and
given the increasing velocities, we have
reached the age of Einsteinianism in business, approaching (as in Dante's Paradiso),
simultaneity \emph{inside} this system and serial,
labrorious time and ageing (the post-paradisical universe) \emph{outside} the system. It is
with the arrival of the computers and highspeed communications, with perpetually
operating, around-the-world-all-time-on-line-markets, time and time-zones,
for some speculators, mean less and less, but are
needed more and more.

The zones, after all, are merely a hangover
of local dawns and sun-settings, a way to
start and end the business day (but at the
same time have reference to eternity). For
those, time becomes a manipulable commensurable, a function of price to be adjusted
seasonally, daily, hourly, minutely to the financial needs of different credit\slash time\slash zone-spanning topologies. Linked-up, high-speed
computers, telexes, with their internal and
communicating velocities, working all the
time to coordinate long and short term messages become like the matching up of different infinities.

Now, it is said, \emph{all} can be linked: financial
markets, factories, laboratory work, electrical grid systems, voice conversations, graphic
displays, on-line-in-real-time accounting
systems, tax structures, banking operations,
brokering, trading programs, games, military and political scenarios, telemedical
diagnostic and treatment-delivery networks,
point-of-sale processors, home banking and
trading, data-retrieval... All change the notion of time and timing. The restlessness of
these unsleeping telematic devices negates
out older senses of time. The movements of
humans, matched up to the \emph{movement of the records of humans and their endeavors}
change the notion of time, timeing and human behavior. Real time begins to dance in
time to telecommunicated computer time;
capital-containing time-movements\slash time-containing capital movements. The human
cycle of production and consumption falls
out of line with the informational cycle of
production and consumption. Life is driven
by these abstractions and fictional populations are more suited to survive at these
speeds than humans. It becomes bizzare
when time becomes not only mensurable,
but something that can be corelated to a set
of logic games.

After the reconstruction, or rememberance
of the Big Bang, time was said to move only
in one direction: \enquote{forward.} An anomoly.
This asymetricality was bothersome. After
all, the universe is electromanichean. Then
it was found, possibly, that for sub-atomic
particles time may be bi-directional, an analogy deduced from the assumed bi-polar nature of charged particles. A question: why
should time only go \enquote{forward,} along a
\enquote{line,} as if trying to move away from, or
perhaps forward into an enormous pool of
pre- or post-existent, paradisical pre-creation.
If there are chronons, positive, forward-moving time-particles, then there should be anti-chronons, negative, backward-moving timeparticles. (Or on the other hand, maybe the
observational equipment, the human, subject to time, could not perceive time any
other way --- except in dreams --- and thus projected this time, operationally, imposing ---
as mind imposes --- a certain order in the universe.)

But, if time is moving forward, and matter
is moving outward, expanding, attenuating,
inflating, then entropic disaster faces us and
preoccupies some small, but influential set
of thinkers. It is astonishing that some far-off
heat-death of the universe should affect this
subset with despair, as if they were faced
with a black and dusty cosmos a mere ten or
twenty years from now. This hints at a religious or at least an ideological sensibility.
The crisis demands reconcentration and reunifiction, these new forms of calculational
and accumulative ideology which permeates
all forms of thought. There is also a peculiar
aspect to this thought, a sort of despair. If the
universe perpetually expands, or if it is
steady-state, or expands and contracts in
cycles, all seems \enquote{purposeless.} Now we are
not merely talking about religious thinking,
but scientists have voiced these concerns.
Purpose, as well as the imposition of order in
all things, is negantropy.

This crisis, this terror of conceptual, informational,
ideological inflation, is seen in finance, physics, cosmology, genetics...
The universe-picture collapses for the physicists. All the functions of esoteric calculation-magic to keep the universe alive emerges
in their logic-compared-to-the-universe. The
banker's loans default, their world system is
threatened with collapse, just as a star, using
its energy too freely, burns out too quickly,
collapsing back on itself into a black hole.
The banker must reschedule or re-time his
loans, or at least reform the calenders of his
debtors ... although he cannot retime his
debtor's lives. For the banker and the physicist, the universe must balance, all things in
it, thus time itself: they must hold their
two-aspected world together. For the banker
it may be nescessary to reschedule or slow
down time.

Can this be accepted by those living
in a debtor --- low-energy, low-mass --- nation,
since the tyranny of their bodies may not
repond to this new schema? They cannot
suspend their bodily functions and await the
paradise of debt payoff or redemption in a
hundred years. Therefore, the bankers must
play games with time, population (genetics)
and space. The physicists and cosmologists
also reschedule time, making statistical projections and retrojections, equating (like the
banker) all time with that contained in
a massified and concentrated microworld
which they can then manipulate with ease.

Operations can be performed with timequanta that distort our sense of what time is.
We can add it; we can subtract it; we can
make it go sideways, crowd centuries into
minutes. How much time was spent by
Dante, going through hell, purgatory and
heaven on that Easter triad of days, 1300:
subjective time inside these three realms as
against objective time outside of it? If time
can be accumulated, can it be sold? Truly
sold as a commodity? Can it be consumed,
metabolized? In some sense, yes. How? Be % TODO be -> by
making it into a commodity; the businessman's trick. Commodity means \enquote{the measured against.} Is time, like other commodities, deliverable? We sell time-sharing. But those are, after all, metaphors. We can sell it
as interest. We sell money, we loan money,
and if we are striving for a profit, time
becomes expressed in interest rates already
embedded in money. All money, all that is
valued, contains time, both the time of its
existence and the time incorporated into it.
But, to buy it is still not to \emph{live} it.

Debt redemption is the redemption of
time-price. (A famous work on slave cliometrics is called \booktitle{Time on The Cross}. After the original time on the cross, came the
journey into another space and the ressurrection.) It is performance in production,
events-to-come-treated as if they had already

happened. Time is delivered from then to
now. Ridiculous? If quarks are confined in
larger particles and cannot be separated, but
nevertheless calculated with as if separated,
why not time? No one has ever seen a free
quark; why not confined chronons? Matching time to value, we monetarize it, but in a
\enquote{confined} manner.

There can be no such thing in our current
financial system as a static pool of money, or
near-money. Since it is restless, it has velocity, and if it has velocity, then it traverses
space --- what is velocity without space --- in a
variety of ways. A meter is now defined in
terms of light travel. It was discovered that a
rigid measuring rod shrunk, and so gained
mass, in the direction of the movement relative to a measuring rod in a slower timeframe. Speeding clocks also slow up in the
direction of velocity. We may also say that money,
in perpetual motion, if massed,
slows up time relative to slower moving
money, or conversely, time inside a massed
and concentrated pool of money slows up as
the velocity of money increases. So now
so many fraction-seconds of light traveled
equals a meter, based on the assumption that
light-travel is a constant in any time-frame.
The same operation applies to money, which
can define space (if not pure space, then at
least real-estate). Increase the velocity (requiring what amounts of energy and investments?) to the speed of light and mass will
begin to become compressed into smaller
and smaller spaces approaching nothingness,
infinite mass, and at the same time will be
relatively eternal. Time, from the point of
view of someone's going slower, appears
longer. Infinity, but relative infinity. Immortality, but relative immortality. And when
the speed of light is transcended (or possibly
when there is enough treasure-energy piled
up) then time might go backwards. We have
arrived at the conditions of the black hole.

If we can talk about dollars per time-unit,
we can now also talk about time-units per
dollar. If we change our rates, we can talk
about more dollars per time-unit and more
time-units per dollar. If inflation takes place,
value drops or dollars per item increases,
and therefore rates increase to compensate.
Or conversely, velocity also contributes to
inflation. By the same token, chronons per
unit can be inflated. Inflation creates time.
We approach immortality.

If this seems like an intellectual game,
without consequence in the real world, we
must consider the effects on people of short
and long-term debt, defaults, accelerated
payments and production, in which the
whole cycle must be speeded up in order to
repay. Either people work faster (and live for
shorter periods of time) for less pay, or fewer
people work at faster rates (aided by the
ghost army embedded in robotics and computers).

The speed of light is the ultimate standard,
the limit, Einstein's \emph{primum mobile}. Since
everything is defined in terms of everything
else, there can be no such thing as a \emph{primum
mobile} other than the one those who set the
standards impose on us. While time is
closely linked to light and the traversal of
space, if we link time to compound iterest
formulae, the parameters change. While we
have been reticent, resistant to play with
time because cycles of hunger, fatigue and
death drain us, it is happening nevertheless,
in our practice. But biological time still lurks
somewhere in our perceptions; yielding to
an artificial immortality terrifies us,
like submitting ourselves to heaven and hell.
It should be remembered that an infinite
amount of money must be spent in order to
become equivalent to an infinite amount of
mass and energy, to an infinite amount of
space, lives, energy, history, time... High
technology, capital intensivity compresses
the mass of commodities. Capitech-intensivity increases its mass-energy-time-velocity
in relation to a slower-moving, starvation world.

All operations that can be performed with
credit also indicates a relative immortality,
for the possessor of the accumulation of
credit possesses a huge accumulation of
stored energy-lives, time, in a very small
space: a great mass, which, because it still
can't be metabolized, must manifest itself in
certain expenditures; cars, houses, military
potlatches, estates, hotel rooms, airplanes,
pomp. This, of course, is reckoning backwards; we are deriving certain laws to explain the insane behavior of the rich. And
yet, a dream persists: of one could only store
enough mass-accumulation, then you can
store, perhaps, enough time in a small enough
space to transcend, or at least reach the
speed of light. The question here is, of
course, \emph{real}, \emph{actual} immortality (for whom
and at what price to everyone else?)

The difficulty lies in any human's being able to \emph{metabolize} an extreme amount of
time, or mass, or energy, or velocity, or convert information (which is also a function of all these terms) to something like flesh ...
convert energy, etc., into usable
energy in an assimilable form. (As Ahab, an
insatiable hunter, desiring a sort of immortality, dreamed of swallowing the power of
the sun concretized in the symbol of the
White Whale, Moby Dick.) Thus, in the abstract and therefore in the real world, the
rich create a vacuum around them by sucking up the abstractions, the information
standing for the energies of the world, which
then siphon off the actual energies of the
universe.

But there is something saving after all. The
real, ultimate constant may not be the speed
of light at all, but \emph{felt duration}, after all, give
or take a little of the average lived life. The
felt duration of people ina different time and
speed frame seems subjectively to be the
same (although no one knows) even though
the person in the fast lane is said to live
aeons longer. It is such calculations that
produce vast masses in time and timer-space,
creating a blind impasse, the existence of the
concept of black holes, out of which no light
or energy escapes... Which yield amusing
stories to frighten and amuse the young, but
does not yield any Magellan's passage to an
India-Paradiso.

\chapter{}

If the \emph{mythology} of the past is \emph{everpresent},
at least in the \emph{memory} (distant signals presently felt, but not in any order) then it too is
simultaneous even though the event is long
gone. Seen this way, event, and thus time,
and thus life, become merely informational;
that is to say an abstraction that can be
handled in a non-real way with real effects.
This is, after all, what dreams and surrealism
are all about. A financial institution is not
only an information-handling and communicating company these days, but a clock, an
observatory with \enquote{eyes} that see electronic
impulses in which all the events it scans are
simultaneous and fused. It is a dream state.

As we continue to struggle on toward
demonstrating this unified field theory as it
is in practice, we see we have arrived
at a dangerous, earth-devouring system of
thought and illusion. We take the whole
development of modern cosmological and
physical theory to be, at the present time, a
function of the \emph{culture} of modern transnational capital. When a certain level of interconnectedness is reached, what will happen is that the universe's symbolic energy
(affecting real energy systems) will move so
fast that it will burn itself out, explode into a
nova (a cosmological financial-informational
bubble), collapse and compress back into a
black hole... leaving the universe outside
quite the way it was for billions of years.

\emph{It is the logic of this mode of thinking that
produces these wierd and surreal visions.}
Welcome to Laputaland.

In the past, magical systems held the notion that there was a link between the microcosm (what happened in people's minds)
and the macrocosm (what happened in the
universe). The macro affected the micro. If
one had the magical tools, then the micro
could affect the macro. Electrical levers were
substituted for magical levers. But the dream
of \emph{magical} control has never been exorcised.
We are still caught in a quasi-metaphysical
system; one that expends enormous resources
and money. It is this dream of intellectual
magic that drives the present information
revolution: the grasp for power and control.
The thought of a small sub-set of the world's
population devastates the earth. Perhaps,
after all, modern capitalism is a great factory
for the production of angels.

Illusion dominates. It doesn't matter yet
since the information doesn't even have to
be true.

The ground is eaten out from beneath us.
We will take Berkeley, Kant and the whole
rout of idealists along to the stars, lured by
worlds we have pre-populated but cannot
reach.

The dream time is upon us.

% \chapter{The Other Side (1984)}

% \chapter{Faust's Stages of Spiritual/Economic Growth and the Takeoff into Transcendence (1987)}

% \chapter{The Destiny Algorithm (1988)}

% \chapter{How the Athenians Planned to Colonize the West and Immortalize Themselves (1989)}

%$ \chapter{Some Notes on Iraq (1990)}

% \clearpage \begin{quotation} The old philosopher's stone could convert base metals into gold. Now, humans, real estate, social relations are converted into electronic signs carried in an electronic plasma. The dream of magical control has never been exorcised. Perhaps, after all, modern capitalism is a great factory for the production of angels. \end{quotation}

\end{document}
