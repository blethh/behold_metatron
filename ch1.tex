\chapter{}

We are in the middle of that Great Transformation into what is called the Information Age, or Post-Industrial Society. As in all Grand Transitions, \emph{fin du siecle's} and climacterics, perceptions of reality are once again being redistorted by the insertion of a vast new mediational system into an already multiplexed, historically accreted maze of mediations. In the context of this forced march, the relationship of information to society and nature has to be rethought.

Call information capital-intensive knowedge, a mechanelectronistic metaphor made to dominate more and more of life. All knowledge is in the process of being converted to computer-compatibility. The old philosopher's stone could convert base metals into gold. Now humans, real estate, social relations \ldots\ are converted into electronic signs carried in an electronic plasma. This would merely be an amusing game if people (in fact only a small subset of the world's population: 90\% of all information processing is controlled by a small part of the \enquote{developed} world) weren't being forced to use and live \emph{through} information processing and communications technology. Call it Informatics; call it telematics.

The components of telematics are mainframes, minis and personal computers, cathode ray tubes, printers, copiers, automated bankteller machines, point-of-sale sensors, antennae, copper and fiber optic wire, copiers, remote-sensing devices, robots (remotely run or otherwise), calculators, integrated chips, software, mass-data-storages, tapes, discs, diagnostic equipment, a babble of \enquote{appropriate} languages, telephones, modems, telexes, terminals, microwave relays, radio, cable, satellites, switching and routing systems\ldots\ Alongside of this, one has to consider the social communication systems and all the transcieving and routing operations there. Even the simplest of conversations are seperated, reconfigured, sent and priced. And those who live in this new world are losing their grip on an older reality. As for those who have no access to, no participation in, this newly imposed world, they are out of the world's new information economy, doomed to obsolescence and death.

A glorious, transcendant and radiant future is promised us. Efficiency will increase, productivity will rise, the office and factory of the future will be automated, we will be able to work at home, teleconference, we will have hoards of instantly retrievable knowledge at our disposal, record-keeping will be easier, we will be freed from work and the burdens of memory. That, or an enormous disaster is in the making as parts of the world become metaphysical. For it's time for Demiurge II. The Year 2,000 is coming. Apocalypse and creation in one.

Whole nations, their economies, their peoples, their resources, their land, can be simulated and displayed on some electronic input/output device. But worse, taken for the real thing. National boundaries become porous and erode. America is no more as trans\-national data-flows penetrate borders. Nations become illusions as foreign enterprises buy pieces of many lands. The informational process has concrete results. True, this is nothing new. International cartels, merchants in past ages accomplished the same thing. But as long as any enterprise becomes translated more and more into its essences --- money and near-money, an all-purpose information, the blood and hormones of business --- those essences cannot be held in containers called nations any longer.

The technology can be likened to a nervous system, one external to humans yet connected to their internal nervous systems by a variety of devices, becoming more fused, joined. For example, with the onset of medical data-bases, monitoring, diagnostic and treatment machines, ancient dreams of being directly connected, the world \enquote{wired} to the brain-nerve complex, leads to the hope that thought alone will move reality.

With the invention of new sensing devices, new perceptual systems come on line. All beings are some function of their information intake, no matter how indirectly the information is recieved. What was done in the mind must now be done through computers \ldots\ programs begin to become quasisolidified thought. New procedures for action and behavior take the form of a ritual, requiring the playing of an excruciating game called programming. People resist? The languages are too hard, the steps to long and complicated? Money is now poured into developing computers that \enquote{talk English,} are touch-responsive or voice-activated. Computers for dummies.

But above all, price is attached to these mediational meditations. Price is a seasonally adjusted, value-added medium in this invented medium, a carrier of values standing for the signs of things sent along a carrier wave. The computer, and its languages, represent a frozen and hard-wired \emph{habituation of thought.} The programs are a way of trying to introduce flexibility, variety and reference into the relative intractability of the machine. However, by itself, and with its operators, and its languages, it is impossible to truly metaphorize --- an essence of human brain activity and thought --- that is to say, fuse into one homgeneity any two or more disparate sensation-terms.

Each \enquote{new age} rewrites the history of the past (while thinking it has discarded the obsolescent past). The last great age of reinvention and rationalization of past and future took place, more or less, from the 15th to the 19th centuries. New world views were created. But does the process of rethinking and reorganizing the past really free any age from that past? Has modern rationalization taken a secret rider, an incubus along in its intellectual and institutional baggage?

New institutions advertise themselves, using the old images of domination to promote the transition. They draw their sales-imagery out of a central bank of symbolic forms. Knowledge of the past is simplified. Epochs are erased (perhaps there was too much that was embarressing in the past). New pasts, whole aeons are invented. Complex existence is simplified, and then recomplexified in another way. Forgetfullness follows. Scramble and resequence; but, in the process of borrowing symbolic energy from the past, new simultaneities and odd juxtapositions, like dreams, emerge.

To look up, to see the stars, the galaxies (in their past and glowing glories) with new kinds of lenses is to have recourse to addresses in data banks where long runs (projected in a short time) of computer-modeled, cosmological statistics are stored (with certain assumptions built in). Look closely at these computer-simulated, eons-long histories of distant stellar objects projected on the cathode-ray tube. Watch them appear to recede. What are we \enquote{seeing}? Are the simulations guided by an underlying compulsion to \emph{aesthetics,} and does this become the ultimate gravitational lens? And those great galactic streamers of stars, and the great gouts of gas jetting off into the blackness\ldots\ how like the monetary jet-streams that banks draw off into the black holes of their balance-sheets from once luminous nations, entropizing and then rematerializing as investments elsewhere. Transubstantiation? Is that what underlies the very concept of the preservation of matter and energy?

Celestial bookkeeping? But see the flaw; the images are seen in squared-off pixels, reconstructions based on a relatively few observations, structured by certain recurring theories. All observational technology is, within limits, the concretization of a speculation. And what we see is all based on some initializing, mythic event: The Beginning.

Troubles in paradise. While trumpeting the imminent emergence of the grand, unifying theory, the unifying theories fall apart. Fundamental forces and particles proliferate. The original central dogma of genetics is riddled with heresies. And even forms of credit go off and multiply. They become desperate to unify and simplify (an ancient compulsion). Unification also implies structuring, measuring, concentration, monopolization, a center, a central intelligence, heirarchies of knowledge, a control room. However, without general acceptance, credibility, faith in new forms of knowledge, these become mere scholastic games. They turn away from observation to their animated projections, assuming them to have been the fruits of experiment, since they cannot journey to the heart of a star, visit a black hole, or distant galaxies, except in imagination. Nor will they be able to make journeys to putative planets without a complete transformation of the body. But they must journey on: after all, their reputations and belief systems, their funding, is at stake. Scientists seem to have reached the end of the line. They have decided that the observing Mind plays an \emph{active instrumental role} in the cosmos (indeterminacy), perhaps once played by god or demiurge.

Modern observational machinery resurrects ancient epistomological problems and incorporates them into itself. Ether, having once failed as a concept, is in the process of being reinvented. Information is the ultimate mediational ether. Light doesn't travel through space; it is information that travels through information as information\ldots\ at a heavy price. The scientists, reacting, are now on their knees, abjectly populating the cosmological and sub-atomic realms with \enquote{god,} \enquote{purpose,} \enquote{design,} \enquote{progress,} \enquote{ascent,} \enquote{transcendance,} "cosmic frequency dances"\ldots\ And if quantum physics tells us that observation intervenes in the observed, and becomes part of it, this also holds true for pure theories of finance, or for that matter, evolutionary genetics, at least according to the sociobiologists. The atom and credit court secrecy. Higher and higher levels of indeterminacy fill every aspect of life.

Scientists cannot seem to live with the possibility that there might be anamolies in the universe. For instance, that: 
\begin{enumerate}
\item this might be the only planet in the universe with life on it; 
\item the universe is infinite and unbounded and, what's more, its contents are not uniformly distributed;
\item that there was, consequently, no Big Bang; 
\item there are no black holes; 
\item the universe neither contracts nor expands\ldots\ it was always this way and always will be; 
\item there are no quarks or pre-quarks, other than as a function of the activity in reactors\ldots
\end{enumerate}
After all, cosmological evidence is a sparse series of stats, stills taken in a short period of time, arranged into an aesthetic, evolutionary, dynamic sequence. Like the ancient mystics, the scientists are projecting themselves into a space they cannot hope to reach, at least in human form. They are colonizing the void with a concept. Like the world's telebanking system with its computer-assisted bankers, they are now inflating space like some financial bubble. Indeed, there is now an inflation model of the universe\ldots\ finance similized into space. Well, if all is one, why not?

Almost every day, new arrays of satellites are hurled into the air: switchboards in the sky, hovering in geosynchronous orbit, remitting messages, insensitive to earthly distances. Other satellites circle the earth rapidly, surveying weather, land, minerals, people. New lines are laid down; optical fiberglass replaces wire (mourn for the slaves of the old copper mines of Cyprus, or for those who died in Chile for Kennecott and think a while of hands cut off for Union Miniere in the old days in the Congo). Every year new computers come on line and are re-interconnected as manufacturers try to configure incompatible computer architectures and languages, requiring the manufacture of new networking devices, plug-compatible hard and software as new models render old ones obsolete.

Data bases for every conceivable aspect of
life are created. However to gather data, to
even think data, requires a vast language
project. For every item must be distilled,
rendered and specified before the computer
can handle it; the old, fuzzier specifications
must be translated. Given the costs of translating 
old knowledge (the price of intellectuals 
and their thought) into new forms (what
room for ambiguity is there?), costs of storage,
heavy doses of electricity to run and
chill the machine, much must be abandoned.
Paper libraries, for instance (of course, promising
the paperless society, more paper than
ever is generated). Certain central markets
(like the New York Stock Exchange, which
depends on personal interaction, 
much secret knowledge and \enquote{tribal} networks) die
hard and slow. Stocks and commodities, the
securities markets, banking, currency, options, 
futures... all these markets must now
be rethought and restructured. Banks become
stock brokers. Brokerage houses sell
insurance. Shopping centers sell securities.
Where once precious metals were carried,
where once paper was exchanged, new electronic
signs and signals verify and celebrate
the exchanges. And if the age of the counterfeit
cosmos has come, it is also the age of
easily counterfeitable currency.

Informational essences become more real
than tangible humans. The very body itself
begins to evanesce, just as in those folk tales
where the shaman's body-parts were scattered 
to the winds and reassembled. Medicine
promises to be delivered from a distance.
Experimentors consult case histories
on tapes or discs thousands of miles away.
Once-free knowledge becomes priced. At
the same time the electromagnetic spectrum
for transmission becomes used up, gets
scarce, high-priced.

As for ordinary life, we are exhorted to
become compatible with all this. We are
urged to fill our homes with personal computers;
to become computer-literate; to write,
learn, do office work; get medical examinations 
and treatment at a distance; do accounting, 
banking, law; buy and sell equity;
shop with computers; do mathematics (without
understanding), logic, all with computers; 
do factory work with robots; confer, and
even eat with computer assistance. A vast
proliferation of books are produced to explain
how to use our computers, to rectify
the outpouring of incomprehensible manuals
written by and for technicians. We play
games, draw story elements out of storages
and arrange new entertainments with the
computer; we have them watch our fuel,
shop, have the lights turned on or off for us,
and even in time, fall in love with far-distant
strangers, perhaps even mate with the gods...
as the ancients were reputed to have
done.

People seated at their terminals forget
time, sitting mesmerized, their fingers jerking 
spastically at the keys, their eyes blearing.
In time, so the sales pitch goes, computers 
will achieve artificial intelligence (and
perhaps even use \enquote{biochips} and so live)
and what's more, they will be more rational,
organize knowledge more neatly than our
poor brain can, once certain problems of
miniaturization, heat, switching speeds, and
the development of sophisticated, \enquote{humanlike,} 
or artificial intelligence languages
have been solved. Perhaps then they will be
able to attain that kind of \enquote{randomness} and
\enquote{intuitive} leap humans make so easily without 
having to scan and compare lists. Having
been talked into surrendering our spirit, our
knowledge, our bodies will become useless.
We will, like Jesus, like Faust, like Dante,
achieve immortality and \enquote{evolve} into computer-compatible 
and re-programmed history, one with Babylon, Nineveh, Rome.
Our essences will be preserved in that great
memory bank in the sky.

But in the meantime, on the peasant land
(what's left of it), in the jungles (what's left
of those), in the world's ghettos (which
proliferate), in the poisoned seas, rivers,
and lakes, the contaminated land, sky and
earth, a lot of humans must be phased out.
Prices decline for technology, but overall,
costs rise because of mundane deregulatory
decisions, questions of intellectual property
(the pricing of ideas), ironic anti-trust decisions (AT\&T, IBM), national information-conflict policies, the classified (or priestly, if
you want to look at it that way) intelligence
approach to knowledge... all contribute to
the expansion of ignorance. Knowledge purveyors block the sun's rays and the rain's
fall, offering to sell sunlight and rainfall...
as signs. Where once you might look up and
see the clouds passing by, you can --- but you
don't have the eyes for it --- look up and see
the spy satellites, the earth/weather/sea/resources sensing satellites hovering, or count
the streams of invisible electronic gold flowing by. Perhaps
you can sense the meta-weather, almost as natural as monsoons. For
a deluge of money --- inflation --- is a \enquote{natural} disaster, creating floods, leaving some
lands sodden and others a desert.

