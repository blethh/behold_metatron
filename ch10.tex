\chapter{}

Enter cliometrics. If legend, folktale, myth is a form of information and meta-history, featuring un-individualized heros as variables, then cliometrics is meta-meta-history --- the kind that makes it easier to match events up to number-series and can incorporate genetics-as-coded-logic --- which itself can be used as a kind of history (as history can be used as a kind of genetics). Each particular person, each event generated becomes a variable in a statistical series. Chosen events in a time series are matched up to chosen sets of self-transmitting individuals. Cliometrics is a sub-branch of those statistical time-series that computer archeologists are so found of retrojecting back to creation in the attempt to make history compatible with logical operations. Cliometrics is quantified history in which people and events are graded along an importance-scale and matches up to a time-scale established by archeology and evolution. Cliometrics, more than any other kind of historiography, is computer-compatible history.

But then, ascent-mythology reenters. All event-clusters on an evolutionary series are in fact fungible. The billions-years-trajectory is mapped onto a short stretch of human history, just as an infinity of natural whole numbers can be, say, mapped onto all the possible fractions between one and two. But like any series, such a construct depends on the measurable regularity of events. Catastrophe, the unexpected, must be integrated, smoothed out. The catastrophe is cut apart into small segments, the compression stretched out and we see that event in regularized, even fragments. Why evolution --- which is supposed to move in glacially incremental steps --- is matched up to history --- which takes place in only a short time --- is an obsession --- progress, growth, incremental gain --- that the 19th century foisted on us. Humans have not changed, not evolved, in all the time they have been on earth.

Then the proposition that evolution, as applied to humans and their history, is the movement from simple to complex, from less to more, from worse to better, from chaos to order, from primal virus to sub-human to human to angelic to god, doesn't work. Whereas the latest studies are purported to demonstrate that more than fifty percent of the economy is information work, a case can be made that whatever the conditions, peasant, primitive, factory worker, more than fifty percent of the work was \emph{always} informational. (Which intrudes a touch of the miraculous, for when, how and with what blinding speed, did humans become the way they are?)

The very act of selection, grading and valuation, in order to constitute part of the latest national rememberance-treasury of our enterprise, to make it fit to be good will, it must now be said that these events are convertible into these numbers, these increments. For it is part of the equity, the capitalization placed retroactively into the start up of the enterprise; a set of past events converted into numbered, stored and convertable values. After all, even eventless, invested capital must have some history. In any pool of money, there lurks an implicit collage of time series.

Now if cliometrics is to have any usefulness, not only must religious history and mythology be quantified and arranged into an accretive time series, but it must be valued. Once quantified they may be translated into other metrical disciplines through a complex series of conversions shuttling back and forth in time. (Although, before this can happen, a rewriting and re-evaluation of the meaning oftime itself is required.)
