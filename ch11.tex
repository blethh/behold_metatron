\chapter{}

Starting from an other end, we can look again at the hypothetical annual report and balance sheet of our enterprise, Western Civilization. If we analyze, concretize, qualify, re-view, individualize or personify the bottom line, representing pools of capital, we unfold it to see a variety of sub-enterprises: factories, farms, plantations, mining operations, shipping, workers, slaves, wars, thievery, pieces of drama, the addition of last year's, last century's, last millenium's profits or losses, the whole ensemble of economic activity. These pools of money and nearmoney (an incredible array of quasi-liquid exchangables) have different key words, titles, identifiers, instructions and names attached to them, indicating by what rules they may be converted from one form into another: numbers to numbers, numbers to people, people to events, events to a varied range of goods...

But to repersonify, re-invent, decrypt events out of our bottom line puts limits on their exchange potentials by pointing to sets of \emph{particular} people, their lives, energy expenditures, products. They are not given to us in the proper language. It is hard to tradea human being (or a ghost) \emph{directly} in our society, with its putative committment to humanist values (each and every human in the world is supposed to be of importance ... but not their work). Yet, the abstracted, refined, distilled, symbolized value of part of a human, their work --- with the rest cast off --- can be stored, calculated and traded every day. Modern purification ceremonies.

The coming of high-speed communications, computers and trading programs allows for the movement of keyworded, encrypted bundles of capital across barriers and borders, to be entered as a profit here ina tax haven (a souls bereft of bodies ascend to heaven), a loss in a taxed sub-world: expensive signals of expenses. \emph{All} forms of credit are highly abstract, general-purpose information. Otherwise, how could they be moved and interfused? (Citibank, Chase, the Bank for International Settlements, Banco Ambrosiano, the IMF, the Federal Reserve System, etc., all banks and securities houses are information-handling, transactions and communications companies.) Information cannot be too particularized in order to make entries in account books. Generalized, it can be teleported easily through the most complex spaces (where humans cannot be moved), some of which are not so much spaces but representations of spaces, magnetic perturbations, addresses standing for a country, another enterprise, files on a tape or disc, or in flight, entries in an account book. I can transmit six million dollars (how many lives; how many things?) froma U.S. bank to a Japanese bank, which has a branch (a presence, but not a nescessarily a facility; a designation) in Panama --- move it from a ledger marked \enquote{U.S.\slash Chase} to a ledger marked \enquote{Panama\slash Bank of Tokyo.} Has the money traveled? What does the question even mean? Yet the U.S., Panama, and Bank of Tokyo behave as if it has.

If we intone \enquote{Russia,} six letters, we have some idea of what that means: the letters, or the sounded word, stands for the space in the file and that space, on tape or disc, matches up to the whole of Russia. (The code-name, the identifier, the recognition-signal, legitimizes the transaction. It must be precise. We cannot say we are sending \enquote{Oedipus,} for what does that mean? Or can we?

At bottom, these forms represent the activities of people in the present, the past, and the future (when capital was/is/will be stored, built up, accreted). Money contains a past: it is a memory and energy-storage. The labor of old machines (animals, plants, and humans) which once converted sun, soil and seed into food, raw materials into things; all these and more are implicitly represented. The credits are a form of miraculous, nonenergetic, energy storage.

We have reached a fascinating metaphysicality; energy storages which are not contained in material objects... Wood, coal, oil... these we can understand. With electricity, things become problematical. Electricity cannot be stored (except in limited ways; batteries). We must produce heat in order to drive the generators. But yet there are abstract, informational, non-heat, non-energized, \emph{symbolic} storage devices which nevertheless have the power to drive the energizer of machinery, the human. Money, securities, paper instruments of all kinds, electronic signals... all these (among others) have this power. But in order to energize they must recieve this power, which is purely a symbolizing activity which must take place in the context of a whole social climate which trains people to respond to the energy-stimulating information associated (not in) these devices, these instruments, these fictions. We are talking, in the long run, about belief-systems.

History resides, embedded in each item we touch. A set of ghosts... remnants and memories of people who worked to produce money, wealth. Looking at a pool of credit, we are not permitted to infer the concrete existences, the lives, the sufferings, the pleasures now, of those pasts, this \emph{unspecified} stuff of phantoms. This analysis holds if we believe that capital is a residue, expropriated and stored work-energy, value and time (representing the expenditure of human energy, lives spent in forced labor). Electronic gold is the newest minimalist form. Capital represents a compressed work-and-time series omnipresent in credit and every processed or owned thing. A continuity of real and fictional people. (Why fictional? Information doesn't have to be about real things or people; it only has to be \emph{accepted} as real.) A revived theory of spirits.

But when we come to the creation of credit --- as the material representation of embedded value, gold, gems, declines in importance --- by banks (real banks, or fraudulent shell-game banks, and all the interest-leveraging games they play, lending five, ten, twenty, a hundred times their assets; national treasuries with their printing presses running full-time, account-book manipulations), we have arrived at the manufacture of value out of nothing. If credit, value are linked to people (genetic series: or their residues, for they continue to work when dead) and the lives they once lived, or might have lived, then inflationary cycles create even more fictional people, fictional lives which impinge upon and crowd the living. These humans and their energies seem to be drawn out of the \emph{future} (looking at interest as a price based on usage and future realization, as expressed in money and time, which is to say energy-expenditure --- labor --- to be realized) and become another way of birthing fictional people and enterprises.

Hence, in this world all the forms of credit (information), if convertable to the lives of populations, create \emph{population pressures} on a limited environment, recrowding the past, or emptying the future. They, these \emph{prespirits}, impinge on the economy and on daily life and the psyche. Fantastical?

If dreams and myths have an effect on the lives of people, it is clear that metadreams and metamyths also have an effect. (What's a % make this a footnote
metadream? The purest form is capital, information without specification. To take another example from other realms of discourse; the score of a psychological test, which converts psychic states into numbers. This involves the pricing of the tests plus the start-up costs for developing the tests, getting people to believe in its validity, and the investment in the formation of some psyche or intelligence-testing company, and the sales of such test, and so forth. None of this mentions cohorts of theoreticians who develop the background theories of human nature. What are we left with? A disembodied and quantified psyche.) Such manipulations violate the first law of thermodynamics, creating energy out of nothing or the future. You can violate the laws of nature, at least for a while, if you put enough money into it... So said the physicist, I.I. Rabi, clearly identifying money and energy as one.

These labor\slash life\slash time\slash suffering residues, these generalized phantoms are peculiar ones. They have no meaning until the underwriters of an enterprise assign a value, a meaning to them. Assigning a value (and raising money against it) requires an act of confidence and the general acceptance of that assignation (which itself must be sold) is what constitutes the general act of faith in certain people's notions of present, past, the invisible, the future. Credit, \emph{credo}. You can invest in it, you can bank on it, you can buy a piece of the faith and people will be moved by it. By what? By this long series of conversions, metamorphoses, transubstantiations and appearances out of pure space, dyings and rebirths, incorporating a kind of serial cannibalism among its many charms. These acts of faith in the modern, rational age, are not merely cool and detached. They are attended by great excitations, curious passions and lusts, unseemly fights for status, murders, massacres, tortures, annihilations of populations, pomp, circumstance, exchanges of gifts, drink, drugs, bribes, kickbacks, display, ostentation, treachery, thievery, and so forth.

Massed and abstracted capital represents not only a history but populations and their biographies. Economic archeology: capitalist historians, pushing progress, are counter- intelligence archeologists. They rewrite their history to deny that the suffering of populations ever took place, just as we deny the spirit world of \enquote{primitives} and their reverence for ancestors. We don't revere ancestors; we revere ancestry: inheritance, not inheritors.

If you averaged out capital buildup per person per life, established historians, rewriting history, tell us that things were becoming better and better during the industrial revolution. Concentration of wealth is ignored. The invention of a class perspective created an oppositional classification, revealing one population's misery. But even Marx accepted these constructs; the evolutionary inevitable. He only wanted to change direction once these \enquote{necessary} sacrifices had been made. Both Marxists and capitalists have played the same kind of game. They didn't challenge the basic, agreed-upon, evolutionary-schemata, the concept of capital itself.
