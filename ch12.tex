\chapter{}

If one could invent (or draw out of some infinite account) values created by populations yet to be born --- fictional people, which might include robots and simulations --- those populations have a real effect on our present. What \emph{space} do they live in? If they're fictional, how can we feel their presence here and now? Or if they are far away, even in the future, how can we feel their impingement, as if they were next to us? They exert mysterious influences through myths and interest rates, like the gravitational perturbations coming at us from past and future, annunciating the presence of an unseen planet or star. We search for them. They can be found by indirect methods; relating population to space, converting the numbers of those-yet-to-suffer into space time and people. This is done by the art of studying population pressures in the present, for new populations are being unleashed upon our planet, teleported to us, backward in time, emerging through the central processing units of financial institutions.

Segmented demographics (studies of populations in abstracted, discontiguous home territories, stacked up, perceived, recorded, retrievable, quite as if they were in some mass memory; memorialized particle-clusters, when networked, make their territories contiguous, while lack of communications-access makes the land of the poor quite seperated) allow us to play new games with the location of populations, to move them from place to place, even into invented countries. For example the Nielsen rating's \cA country, \cB country, \cC country. (There is a \cD country but that is discounted. Dante also had an \cA, \cB, and \cC country and, although he didn't write about it, a \cD country; for Dante never wrote about the poor and the ordinary.) People in \cA country can be scattered all over the world; they don't even have to live in proximity.

How were these populations get into these demographies? If we have totalled up real people --- hungering, sweating, copulating --- then we need real geography, at least at some point in time... perhaps in the past. With the advent of the need for targeted, reachable populations, demographic regions were invented, related, classified, archtypalized along buying\slash class\slash income\slash taste\slash interest\slash communication lines. We see continents dissolve into an archipelago. (Dante preferred to stack them in cones and concentric, high-velocity circles.) All nations have --- at least as far as some of their populations who live in transnational space --- dissolved.

Corporations transcend boundaries. Financial institutions hurl money across timespace. Pools of wealth flee the constraints of national taxation. Russian and Americans bank in off-shore havens. The French and Germans invest in Russia and the U.S. I.G. Farben, Dupont, Imperial Chemical and Mitsui had an explosives cartel throughout the Second World War. What does nationspace mean to them? They live in Paradiso. They absorb the energies of dead Germans, Americans, British, Japanese. This subset of French, German, etc., constitute a cultural homogeneity.

Such an approach generates endless lists for direct-mail and political polling, possibly electro-politics. One could even poll \emph{the compilations of taste-lists} (which would contain significant sociological, psychological and bio-medical data simulating humans), a thing the computer does very well. The information age becomes more scholastic and magical.

Given these geographies --- these weird topologies inhabited by real and fictional populations --- questions are raised: are we merely playing an intellectual game, or do these fictional people have some substance? Do they have a history? How were these list-populations generated?

In his extremely information-oriented, mechanistic, Laplacian \emph{Sociobiology} (genetics percieved not only as information, but as the motivators of every human act), E.O. Wilson proposes the notion of the feeding capacity of an environment, a region, a land, a territory, in relation to the population of animals, or humans. The relationship is called the density-dependence ratio. The amount of grass to cows, cows to humans, for example, in a set amount of geography. This limits the feeding capacity of cows and humans. If one factors \emph{percieved need} the ratio changes in a non-rational way. When class, and all that term entails (culture, for instance), one must erect a set of densitydependence equations and ratios for hierarchies of populations, relating masses of fictional capital to fictional populations to real populations, since capital needs to feed and be fed... and all of these to assigned to certain confined spaces, which constitutes mass. We are beginning to talk about frame-references of space, energy, time, money, genetics (both real and invented).

Grass converts sunlight; cows convert grass; humans convert cows by ingesting and transforming them. Commodity brokers, grain companies, agribusinesses, food-processors, chemical companies (for preservatives and fertilizers), refrigeration specialists, distributers, railroads, truckers and unions, politicians and bankers... intervene beween humans and their eating, seperating mouth from food, by inventing multiplexed and involuted distances which are a function of pricing through two kinds of intersecting transportation systems: physical transportation of goods and transportation of money, invoices... The complexities of credit and time, in all of their forms --- commodities, options, future, indices --- creates conversionary mediation-complexes and lengthens distance. The need of a loan in some distant place creates a shortage of liquidity on the local land, a sort of drought, symbolically and effectively equivalent to a plague of grasshoppers.

The more capital there is, the more population. A small, capital-intensive poulation consuming vast amounts of capital-intensive food with the aid of technology becomes equivalent to a huge population, crowding out the living from earthly space by the reproduction of hungering ghosts. No wonder the ancients fed ghosts, spirits, gods. And given cliometrics, demographics and the density-dependence equations, considerations of capital, telematic acceleration, surely we can find these antecedant considerations somewhere in our ancient texts.

We could estimate the amount of food, perhaps in calories, required to feed each individual, averaging consumption, and set a provisional standard for existence. But this is mere munch-democracy. It's not the way things work. Given real considerations of power and concentration, much capital never even reaches the earth, but remains in perpetual transit in the empyrean, going from countryless account to countryless account, money being invested in money (which embodies time) for high interest rates and other profitable instrumentalities, passing through odd logic gates and mythic addresses. Some populations not only eat for themselves, but for whole hordes. The bankers have, in fact, invented hyperspace concurrently with the physicists and have their own white and black holes. In short, relatively, we are speaking of gods. All they lack is actual, as against relative, immortality.

Wanting to know what his population is all about (consider a heaven or hell, full of ghosts, phantoms, spirits; the dead and their claims upon the living; the exploitation of the dead) we can, by decoding, assign characteristics to this population. We can ressurect the dead, or the unborn, though not, unfortunately, with their bodies intact. We can write histories. Suppose the recorders of life, the gossip columnists, the journalists, sat down and looked, along with the poet, novelist and biographer, over the bottom line of our enterprise's balance sheet, instead of looking at the illusory world around them. If our premise is right, why couldn't they construct (or deconstruct) a novel, a history, a biography from the numbers? How?

For example, one could use psychological, medical testing and diagnosis, to record and store up the signs of the living. We can get readings on medical machines: lie-detectors, EEG's, EKG's, stress-analyzers, voice prints, PET scanners, CAT scanners, NMR machines, varieties of brain scanners, and other kinds of instrumentation. We can create electronic simulations that fluctuate, as if alive. We can store those records electronically (at a price). We can establish a constant link between body and record (itself fragile, almost \enquote{alive}) so that as the body changes, the record changes. Each change alerts the medical monitoring machine to take action. The record becomes an analogue of the body. If the development of medicine has been correct (problematical) and the translations of bodily qualities into signs, numbers and waves is correct, and the instrumentation is correct, the simulation begins to approach autonomy (as long as the electricity is on; doubtful in these days). Now presumably this body-to-record-and-recorder can be reversed, so that when the records change (the simulation is given a disease and then transmits it to the body), the state of the body changes. We can invent whole library-demographics of electronically simulated bodies. We can also put them in synch with the state of the world's economy. So that a change in the numbers in a complex of account sheets would change the health of the simulated population and the health of the simulated population would affect the health of the world... And, indeed, isn't that what happens in the case of the sacred king? Oedipus?

We can abstract and average out the readings; we can even feed back pre-recorded physiological data into this ghost population and give it life. We can appropriate the stored-up work-effort of the sports athlete, the assembly-line athlete, the drug-consuming athlete, the neurosis-athlete (who struggles to produce new records in psychosis: anguish indices). Why, pain itself can be telemetered on different scales, even stored electronically in these computerized telemedical systems and transmitted, when needed, through wire or satellite, broadcast to far distant places. After all, the electrical \emph{instruction} to a prod applied to the genitalia doesn't have to be administered directly by a finger pushing the button that turns on the juice, button connected to finger connected to person in the same room with the tortured. The amounts of current can be calculated from far away (assessed by previous testings) and be transmitted by satellite to some banana republic, just as money (compressed and massed lives, sufferings) is transmitted every day through SWIFT.

We can even poll the indicators to find out the wishes of this phantom population. In fact no one has to poll real people, just query the taste and market-cluster data-banks containing the repository of psychosocial desires and physiological indicators. It is already clear that voting electronically can be like ballot-box stuffing, using votes of the dead, since the powerful make the powerless a transmitter and talk only to themselves. If a big stockholder votes his shares, why can't he vote his ghosts?

When we move into these realms of abstraction, we can play any game we want with the indicators, the concepts, the \enquote{stand- fors,} just as long as there is consensual agreement to credit, honor, have faith in these acts. And this is what's happening.

In Kaballistic thought, the elements of a language, ordered one way, reveals one world. Ordered another way, a completely different world with different laws emerges. Language precedes things and humans in enostic thought. Even history itself can be restructured (along with geography) in interesting ways. Events in \emph{Oedipus} can be placed next to the events in Joseph's Egypt (since they are about agricultural disaster), even made concurrent with a cash-flow crisis in the U.S., and related to the collectivization period in the Soviet Union, and to link them up we have the marriage of IBM, Comsat General and Aetna into Satellite Business Systems linked by Systems Network Architecture. After all, wasn't it Delphi's and Joseph's monopolization of knowledge that caused all the trouble?
