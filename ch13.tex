\chapter{}% 13

We have been moving from time-series to simultaneities. Serial and synchronous time threaten to become surreal time.

Speed and distance are functions of time. In the world of linked up computers, messages move faster at the center than at the peripheries where messages move an entirely different way. What's the center? One can propose a model: a set of rings. Messages in the inner ring move fastest: less distance to travel. Messages to and from the outer ring move slower. Dante's model. This is a conceptual device that expresses the state of communications today. However the center is in fact spread and networked all over the world. It is faster, for, say, Citicorp to get a message to Hong Kong from Lexington Avenue, than it is to deliver a message across Manhattan walking, riding a bicycle or taking a taxi. Citicorp-Hongkong is a center: 9,000 miles. Lexington Avenue-Eighth Avenue is a periphery: 1/2 mile.

When we take into account pricing and power, the the problem becomes even more complicated when the message traffic has to go through some center or complex of centers. It is asserted that if everyone is linked up by interactive terminals and microcomputers, this blazing center of knowledge will be available to all. This is nonsense. In the real world competetive advantage depends on your opponent's being relatively ignorant. We're not even beginning to talk about price and the horrendous effects, in the U.S., of the AT\&T divestiture. Prices of computers go down: this is true. But prices of communications not only go up, but will be unavailable to a large group of people. And anyway, one has to reeducate oneself to use these clumsy machines.

If we are to make a transition to the information economy, in which information is a certain kind of currency, certain steps must be taken. Treasure is meaningless if everyone has it. Treasure, and every good, has built into it a political and business version of the second law of thermodynamics. Maxwellian demons concentrate treasure, energy and information. These are shrunk, massed, concentrated into smaller and smaller class-spaces. When knowledge becomes treasure, the value of it is meaningless if everyone has it. But there's a problem. The spread of information is limitless. If we tell a number of people something, then they all have it. So the purpose of the information revolution is to put a value, a price on information and add to the rituals of learning by technologizing it so that few may have it. In the context of the present attempt to make the grand transition to this new era, we have come to see what this means. It is a way of recapitalizing the past and to undo what Lucifer or Prometheus did. Think of the whole complex of modern telematics as one, gigantic, central, country-spanning intelligence and counter-intelligence agency. It also means that everyone outside this informtion economy is doomed, and that, perhaps, is half the world's population. This is important to remember.

It is said that the speed of generating and processing messages inside of a computer may be faster than in the human brain. That's one way of looking at it. But, in fact, the permissible messages, their content and form, in a computer are enormously different than the message traffic inside of a brain, especially if one considers the development costs (which are in their way a function of time and energy).

The application of abstraction to things or people creates problems. One can say two, four, six...: obviously the next number should be eight. But we can also pick any number at all and make that the next step after six, and invent a logical proof for that choice. A logical proof can be invented to justify \emph{any} arrangement. (We are moving toward a consideration of time-series in a modern, quantumized, relativized, financial, informationalized context.)

There are values, variables, with a multiplicity of identifiers, from different yet convergant frameworks, assigned to the stored-up residues of past, present and future human activity. It may be a genetic identifier, a financial identifier, a cliometrical identifier, a literary identifier, a physical identifier. The arrangements of history and the sequence of the buildup of capital of all sorts (taking into account the falsified and adjustive historiography as common practice: for instance, CIA or Church historiography) is somewhat like a problem in scheduling information traffic in a computer. It must be controlled by timers managing the sub-routines, moving and saving bytes, using loops, querying the memory, all contributing to the flow of traffic, done as events happen, after events happen, before events happen; a sort of time-travel. Given something abstracted, but accepted as an act of faith and so lived-by, as a pool of credit, one can fill in any history one wants.

But in order to do so requires that one overcome deviant memories and histories. One has to fight to control the history, its event, its passions, its humans, its meaning. This we surely know: people died miserable to contribute to that pool. Defining the meaning of that pool becomes a political and ideological fight over good will. The winner writes history.

The derivation or invention of any series takes place both in historical contexts and according to \enquote{deeper needs.} But these \enquote{deeper needs} are not to be found in nature, or \enquote{Man,} but are the shared desires of a small part of the world's population who constantly fine-tunes the ancient methodologies of series\slash simultaneity-making. The \enquote{facts} --- whatever those are --- or processable specifications, establishes a background theory for those \enquote{facts.} The accumulation of many forms of capital is required, each as a contribution to the information economy, for we are no longer in that age when the wishes, ceremonies, sacrifices and incantations of priests and shamans seemed to control the universe: although the sacrifices still continue.

For capital to be accreted and stored, there must have been sets of people arrayed in some time-sequence, laboring to build it up (and also wasting it) during the historic process of production, circulation, consumption, storage and reproduction for that subset of humans who are series-makers and rememberancers. Certain goods may have decayed, but they can still be stored eternally, retrieved, called up, as information.

There's a limit to how long actual grain can be stored but there's no limit to how long we can store the abstractions standing for the grain. It is possible to sell a ton of grain harvested in Pharoahonic times now. The only thing is that it cannot be \emph{eaten}, only bought and sold perpetually. If the buyer and seller agree, one can sell the Pharoahonic grain and use the money to buy real grain. Perhaps it is only the designator, \enquote{Pharoahonic grain} which throws us. Can't we sell a cargo of grain a thousand times, symbolically moving it from port to port without that cargo actually moving?

At issue is the relation of symbols, information to the non-informational world. What happens if the informational world collapses? Panics, depressions, bubbles, inflation are all \emph{informational} collapses. The non-existant crowds out the living.

If we have a pool of symbolic capital, which stands for, and is used for, stored energy, stored value, stored time, stored space, dreams and aspirations, then we implicitly have an accompanying population-continuity and \emph{population-simultaneity}. It may be fictional but can also be considered a storage of real and fictional genetic sequences. We may consider how real people adapt to their changing environments, but we must also think about how fictional populations adapt to material environments and how real populations adapt to fictional environments. For if they are valued, their fictional lives impinge on the lives of the truly living.

What sort of time-sequence-storage does a genetic sequence in any one human represent? What we are supposed to have is life, enormously compressed, a serial simultanized, represented by pools of credit. The pools of credit are as folded up as any crumpled helix of genestrings. And if the production of engineered humans becomes possible --- given enough money (taken from where) to suspend the laws of nature --- capital and genetics can be compared, even equated. A look at the bio-engineering markets is in order. Where do these fictional populations \enquote{live}? On everted globes, on satellites and space colonies, or ribbon planets, in chip architecture, on paradisical islands before, beyond or at the end of time itself? What operations must we do with these time series\slash simultaneities, these lives, real and false? But what's time?

We have been bound by several perceptions of time, subject to various revisions. We have been tied to the tyrannous cycle of ageing, risings and settings of suns, rounds of seasons (and seen the priests control those rounds, inserting themselves between us and the sky), birth, growth, death: \emph{felt} duration. Our biological clocks can be fooled.

The perception of time became industrial gradually, introduced in the 15th century or so. Time's continuity was fragmented into equal lengths, matched up against factory and production ties; unit time, unit goods, unit prices, unit consumption, units of exchange, but all arranged into the cheerful, progressive, accumultory one-way-up trajectory. This vision was introjected into the conciousness of those inhabiting the industrializing world. It is being introduced now into the consciousness of those inhabiting the underdeveloped world.

Time zones were created in relation to the sun's passage, marking the business day and year: market time. But all renegotiated timeschema retained this long range trajectory, the primal beginning and the ultimate end.

Enter, just before the industrial revolution, the modern magicians. First wave: mathematicians, scientists, logicians, topologists (and technicians), the Founding Fathers of the New Age, \emph{circa} the 17th and 18th century... followed quickly by accountants and business topologists, the time and money managers.

But Leibniz and Descartes were primarily mystics: Galileo faked the results of experiments. As for Newton, the evidence is that he was more interested in gnostic\slash astrological\slash alchemical\slash hermetic thought than science. In astrological thought, for the stars to affect life, and conversely, \emph{instantaneous} transmission of forces are required. Perhaps for Newton the enterprise of regularizing the universe was required to give a sound and calculable foundation to astrology. The astrological requires order and regularity as well as an orderly medium for the transmission of heavenly signals affecting human life, thought and destiny. Newton tried to formulate a precise scientific methodology for dating events, using Scripture and Greek myths. For Newton, time was teleological. He related time to a history of royal, Hebraic dynasties. He matched up time, considered abstractly, to a special kind of ethnic\slash dynastic genetics (although he didn't use those words). He felt that the ancient Jews had secret knowledge which filtered down to the Pythagoreans. He considered the music of the spheres a metaphor for the law of gravity. He believed that the dimensions and configuration of Solomon's Temple concealed alchemical formulae which corresponded to a divine unity in nature. He explored sacred geometry, practised alchemy (along with Robert Boyle), and was of course that perfect kind of compulsive dualist in all things. Newton was also alchemically and financially involved with gold; he was Master of The Mint. Given this, Newton's \enquote{beginning} is religious, extrapolated to Nature.

Or maybe he wanted regularity and predictability because he lost money in speculation.

How much better than Velikovsky was Newton?

It was this complex of thought upon which the reconstructions in relativity and quantum physics are based.

With the introduction of artificial light, divisions into day and night begin to end. With sealed, climate-controlled environments, the seasons begin to become irrelevant. the conversion of the natural world into the artificial world, from the raw to the processed, continues. For some the world is already the atemporal control room of a space ship where the ever-chilled, perpetually running, energy-consuming computers, spinning out their fantasies, are attended.
