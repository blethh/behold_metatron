\chapter{}

Einstein's thoughts (building on and reconstructing Newton, the alchemist's thought) on simultaneity become instructive, as do the thoughts of modern bankers as they scan their electronic spread-sheets. For bankers and physicists meditate on states of simultaneity, relative to those who do not, \emph{cannot} have access to this distance-insensitive equipment. This is to say bankers enjoy abstract immortality in relation to the banked, along with quantum physicists and metaphysicians and magicians.

For Einstein a number of events can be considered to be taking place at the same time, but only in relation to an observor recording different signals from those events and timing them with coordinated clocks. (Compare the delicate timing of an international arbitrage operation which requires relative speed, or relative simultaneity, and at the same time requires relative ignorance --- or relative distance --- on the part of one's opponents.) Everything depends on the observor and the recording instruments and relative accumulations of knowledge (the instruments being a manifestation of assumptions built into them) and speeds of transmission to determine what looks like simultaneity.

Simultaneity has nothing to do with where those events are happening (unless you are being shot at from two directions), but rather when they are percieved. Perception is a function of distance and the tools required to transmit knowledge. A corporation, our chimera, in the \emph{contemplation} of law, this society of the anonymous and hidden, an organic being (more than the sum of activities of individuals and groups), with stored-up misery-and-energy, lives converted to credit, may be scattered over a wide geography and over time. All things in it and of it and about it may be considered simultaneous from the point of view of an auditor contemplating this balance sheet. When the accountant's sheet is computerized, constantly recieving information from all over the world and interconnected with the product of statistical projections from all over space and time, then all operations are happening simultaneously, including invested-in future events, since they are handled in the present.

Similarly, it is said that any complex of genetic material contains the complete history of the organism and all preceding organisms (a genetic archeology, but still alive). By the proscess of combination and recombination of its memory elements, amino acids, etc., since it contains all past, probable, potential and possible organisms, even pre-organisms; and like the elements of, the combinatorial rules of, say English (rules which are derived after the languge is mature), is subject to the same operations, containing all past and future literary works, even those that Shakespeare forgot to write. Implicit in all these observational operations is the notion of the simultaneity-observor of all genetic and\slash or linguistic, and\slash or monetary possibility. And thus genetics, if seen as information, is the rememberance of things past (all the organic --- and inorganic --- universe, all the realized and unrealized beings, objects and forces) which generate the organisms which invent various schema of rembererance, which then remember the complex that recollects it.

But for Einstein there was an upper limit: the speed of light. The quantum physicists, introducing indeterminacy, and the intervention of the observor and his instrumentation, implied that in a certain sense, all events were in fact simultaneous, regardless of clocks, for they had distributed Mind into the universe, a notion Einstein rejected.

Let's take a side trip to paradise and consider time there. What happens before the beginning, or after the end, is a question to which Kaballists and gnostics address themselves. This problem has been transmitted right down into modern times in a new form: what happened \emph{before} The Beginning, the Big Bang (itself a construct open to doubt)?

\emph{All} matter, energy, space and time (and thus all possibility) was contracted into a dimensionless point (or nothingness) ... so the mythic tale goes. Infinite mass, for if the point was suseptible to measurement, even of the most miniscule kind, then its mass was less than infinite. If all matter, space, motion, energy, time --- and potentially, perhaps inevitably, all life --- in the universe was massed into this dimensionless point, then there could be no time or space \emph{outside} this dimensionless, infinite-massed point. Infinitely compressed matter and energy (which included, potentiality, all life) meant that there was no one to measure it, not even any automated measuring tools. So all Mind was there too. Perfect simultaneity, \emph{but not objective simultaneity}, which, being involved with signals, distance and time, couldn't be measured. It was therefore eternal inside the dimensionless point, a feature of all paradises.

Time is not a term that stands by itself, nor does any other term: all terms are multireferential, bootstrapping every other term into the air where it hovers like some plasma, contained away from the apprehension of most people. But the multi-referentiality (as well as the breakdown into terms of a complex of things) indicates the intervention of mind inside the universe. If Mind is an emanation of that pre-moment --- physical, chemical, biological, monetary, literary, religious --- it might be possible to \emph{remember} this paradisical past now, as we \emph{remember}, say, Eden, a lower-level and later paradise. It is out of logic, the mathematics, the equations --- a form of metaphorical activity --- the statistical retrojection --- a specialized and leached-out form of memory --- and the observation of the distribution of matter in space, and the Red Shift, that we might infer, and remember, the Big Bang.

It is over this Big Bang that modern physicists, ancient priests and shamans, magicians, caballists and communications artists come together.

Given this compulsive activity, this associative, similizing, metaphoric, organizing, disorganizing, ordering, concentrating, distributing, interventionary, aesthetic activity of humans, and the compulsive nescessity to lay on one hundred and forty four interpolations where there appears the slightest gap (indeed to invent gaps) we can also say that all economic observations contain the purified metaphysical and stored-up simultaneous record of all activity. And we can add that the genetic material is not only an information-analogue, but a factory and clock analogue: the gamete's growing becomes an analogy to the ever repeated evolution, a biological mini-Big Bang.

Since all credit is meaningless unless linked to an active, perpetually moving market, linked to people to believe in it and to the institutions they inhabit, given these massive flows of \emph{perceptions}, symbols and signs standing for life, space, factories, and given the increasing velocities, we have reached the age of Einsteinianism in business, approaching (as in Dante's Paradiso), simultaneity \emph{inside} this system and serial, labrorious time and ageing (the post-paradisical universe) \emph{outside} the system. It is with the arrival of the computers and highspeed communications, with perpetually operating, around-the-world-all-time-on-line-markets, time and time-zones, for some speculators, mean less and less, but are needed more and more.

The zones, after all, are merely a hangover of local dawns and sun-settings, a way to start and end the business day (but at the same time have reference to eternity). For those, time becomes a manipulable commensurable, a function of price to be adjusted seasonally, daily, hourly, minutely to the financial needs of different credit\slash time\slash zone-spanning topologies. Linked-up, high-speed computers, telexes, with their internal and communicating velocities, working all the time to coordinate long and short term messages become like the matching up of different infinities.

Now, it is said, \emph{all} can be linked: financial markets, factories, laboratory work, electrical grid systems, voice conversations, graphic displays, on-line-in-real-time accounting systems, tax structures, banking operations, brokering, trading programs, games, military and political scenarios, telemedical diagnostic and treatment-delivery networks, point-of-sale processors, home banking and trading, data-retrieval... All change the notion of time and timing. The restlessness of these unsleeping telematic devices negates out older senses of time. The movements of humans, matched up to the \emph{movement of the records of humans and their endeavors} change the notion of time, timeing and human behavior. Real time begins to dance in time to telecommunicated computer time; capital-containing time-movements\slash time-containing capital movements. The human cycle of production and consumption falls out of line with the informational cycle of production and consumption. Life is driven by these abstractions and fictional populations are more suited to survive at these speeds than humans. It becomes bizzare when time becomes not only mensurable, but something that can be corelated to a set of logic games.

After the reconstruction, or rememberance of the Big Bang, time was said to move only in one direction: \enquote{forward.} An anomoly. This asymetricality was bothersome. After all, the universe is electromanichean. Then it was found, possibly, that for sub-atomic particles time may be bi-directional, an analogy deduced from the assumed bi-polar nature of charged particles. A question: why should time only go \enquote{forward,} along a \enquote{line,} as if trying to move away from, or perhaps forward into an enormous pool of pre- or post-existent, paradisical pre-creation. If there are chronons, positive, forward-moving time-particles, then there should be anti-chronons, negative, backward-moving timeparticles. (Or on the other hand, maybe the observational equipment, the human, subject to time, could not perceive time any other way --- except in dreams --- and thus projected this time, operationally, imposing --- as mind imposes --- a certain order in the universe.)

But, if time is moving forward, and matter is moving outward, expanding, attenuating, inflating, then entropic disaster faces us and preoccupies some small, but influential set of thinkers. It is astonishing that some far-off heat-death of the universe should affect this subset with despair, as if they were faced with a black and dusty cosmos a mere ten or twenty years from now. This hints at a religious or at least an ideological sensibility. The crisis demands reconcentration and reunifiction, these new forms of calculational and accumulative ideology which permeates all forms of thought. There is also a peculiar aspect to this thought, a sort of despair. If the universe perpetually expands, or if it is steady-state, or expands and contracts in cycles, all seems \enquote{purposeless.} Now we are not merely talking about religious thinking, but scientists have voiced these concerns. Purpose, as well as the imposition of order in all things, is negantropy.

This crisis, this terror of conceptual, informational, ideological inflation, is seen in finance, physics, cosmology, genetics... The universe-picture collapses for the physicists. All the functions of esoteric calculation-magic to keep the universe alive emerges in their logic-compared-to-the-universe. The banker's loans default, their world system is threatened with collapse, just as a star, using its energy too freely, burns out too quickly, collapsing back on itself into a black hole. The banker must reschedule or re-time his loans, or at least reform the calenders of his debtors ... although he cannot retime his debtor's lives. For the banker and the physicist, the universe must balance, all things in it, thus time itself: they must hold their two-aspected world together. For the banker it may be nescessary to reschedule or slow down time.

Can this be accepted by those living in a debtor --- low-energy, low-mass --- nation, since the tyranny of their bodies may not repond to this new schema? They cannot suspend their bodily functions and await the paradise of debt payoff or redemption in a hundred years. Therefore, the bankers must play games with time, population (genetics) and space. The physicists and cosmologists also reschedule time, making statistical projections and retrojections, equating (like the banker) all time with that contained in a massified and concentrated microworld which they can then manipulate with ease.

Operations can be performed with timequanta that distort our sense of what time is. We can add it; we can subtract it; we can make it go sideways, crowd centuries into minutes. How much time was spent by Dante, going through hell, purgatory and heaven on that Easter triad of days, 1300: subjective time inside these three realms as against objective time outside of it? If time can be accumulated, can it be sold? Truly sold as a commodity? Can it be consumed, metabolized? In some sense, yes. How? Be % TODO be -> by
making it into a commodity; the businessman's trick. Commodity means \enquote{the measured against.} Is time, like other commodities, deliverable? We sell time-sharing. But those are, after all, metaphors. We can sell it as interest. We sell money, we loan money, and if we are striving for a profit, time becomes expressed in interest rates already embedded in money. All money, all that is valued, contains time, both the time of its existence and the time incorporated into it. But, to buy it is still not to \emph{live} it.

Debt redemption is the redemption of time-price. (A famous work on slave cliometrics is called \booktitle{Time on The Cross}. After the original time on the cross, came the journey into another space and the ressurrection.) It is performance in production, events-to-come-treated as if they had already happened. Time is delivered from then to now. Ridiculous? If quarks are confined in larger particles and cannot be separated, but nevertheless calculated with as if separated, why not time? No one has ever seen a free quark; why not confined chronons? Matching time to value, we monetarize it, but in a \enquote{confined} manner.

There can be no such thing in our current financial system as a static pool of money, or near-money. Since it is restless, it has velocity, and if it has velocity, then it traverses space --- what is velocity without space --- in a variety of ways. A meter is now defined in terms of light travel. It was discovered that a rigid measuring rod shrunk, and so gained mass, in the direction of the movement relative to a measuring rod in a slower timeframe. Speeding clocks also slow up in the direction of velocity. We may also say that money, in perpetual motion, if massed, slows up time relative to slower moving money, or conversely, time inside a massed and concentrated pool of money slows up as the velocity of money increases. So now so many fraction-seconds of light traveled equals a meter, based on the assumption that light-travel is a constant in any time-frame. The same operation applies to money, which can define space (if not pure space, then at least real-estate). Increase the velocity (requiring what amounts of energy and investments?) to the speed of light and mass will begin to become compressed into smaller and smaller spaces approaching nothingness, infinite mass, and at the same time will be relatively eternal. Time, from the point of view of someone's going slower, appears longer. Infinity, but relative infinity. Immortality, but relative immortality. And when the speed of light is transcended (or possibly when there is enough treasure-energy piled up) then time might go backwards. We have arrived at the conditions of the black hole.

If we can talk about dollars per time-unit, we can now also talk about time-units per dollar. If we change our rates, we can talk about more dollars per time-unit and more time-units per dollar. If inflation takes place, value drops or dollars per item increases, and therefore rates increase to compensate. Or conversely, velocity also contributes to inflation. By the same token, chronons per unit can be inflated. Inflation creates time. We approach immortality.

If this seems like an intellectual game, without consequence in the real world, we must consider the effects on people of short and long-term debt, defaults, accelerated payments and production, in which the whole cycle must be speeded up in order to repay. Either people work faster (and live for shorter periods of time) for less pay, or fewer people work at faster rates (aided by the ghost army embedded in robotics and computers).

The speed of light is the ultimate standard, the limit, Einstein's \emph{primum mobile}. Since everything is defined in terms of everything else, there can be no such thing as a \emph{primum mobile} other than the one those who set the standards impose on us. While time is closely linked to light and the traversal of space, if we link time to compound iterest formulae, the parameters change. While we have been reticent, resistant to play with time because cycles of hunger, fatigue and death drain us, it is happening nevertheless, in our practice. But biological time still lurks somewhere in our perceptions; yielding to an artificial immortality terrifies us, like submitting ourselves to heaven and hell. It should be remembered that an infinite amount of money must be spent in order to become equivalent to an infinite amount of mass and energy, to an infinite amount of space, lives, energy, history, time... High technology, capital intensivity compresses the mass of commodities. Capitech-intensivity increases its mass-energy-time-velocity in relation to a slower-moving, starvation world.

All operations that can be performed with credit also indicates a relative immortality, for the possessor of the accumulation of credit possesses a huge accumulation of stored energy-lives, time, in a very small space: a great mass, which, because it still can't be metabolized, must manifest itself in certain expenditures; cars, houses, military potlatches, estates, hotel rooms, airplanes, pomp. This, of course, is reckoning backwards; we are deriving certain laws to explain the insane behavior of the rich. And yet, a dream persists: of one could only store enough mass-accumulation, then you can store, perhaps, enough time in a small enough space to transcend, or at least reach the speed of light. The question here is, of course, \emph{real}, \emph{actual} immortality (for whom and at what price to everyone else?)

The difficulty lies in any human's being able to \emph{metabolize} an extreme amount of time, or mass, or energy, or velocity, or convert information (which is also a function of all these terms) to something like flesh ... convert energy, etc., into usable energy in an assimilable form. (As Ahab, an insatiable hunter, desiring a sort of immortality, dreamed of swallowing the power of the sun concretized in the symbol of the White Whale, Moby Dick.) Thus, in the abstract and therefore in the real world, the rich create a vacuum around them by sucking up the abstractions, the information standing for the energies of the world, which then siphon off the actual energies of the universe.

But there is something saving after all. The real, ultimate constant may not be the speed of light at all, but \emph{felt duration}, after all, give or take a little of the average lived life. The felt duration of people ina different time and speed frame seems subjectively to be the same (although no one knows) even though the person in the fast lane is said to live aeons longer. It is such calculations that produce vast masses in time and timer-space, creating a blind impasse, the existence of the concept of black holes, out of which no light or energy escapes... Which yield amusing stories to frighten and amuse the young, but does not yield any Magellan's passage to an India-Paradiso.
