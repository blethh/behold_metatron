% 2
\chapter{}

How did this development come to be? Surely more forces were at work than \enquote{Progress?} This
essay is not a history of the information revolution, but some mention must be made in passing. At
some point during the Second World War, a series of decisions to computerize were reached. The
overriding concerns were military and intelligence applications. It should be noted that private
industry would never have invested in this, or any other development. Without government investment,
bankers are paragons of timidity. 

The founders of the information, or cybernetic age, were Alan Turing, John Von Neuman, Norbert
Wiener, Claude Shannon and later, Noam Chomsky. Hordes of electrical engineers---whether they
understood what they were doing or not---reworked almost every philosophical problem known to
humans in terms of circuitry and programming languages. These problems began, of course, centuries
ago. For instance the epistimological question: what is knowledge, how do we know, how do we know we
know, how does it relate to the world outside, who controls knowledge, who has it and who does not,
what is it worth, how do we talk about it (which is the question of what language we shall use and
how shall we talk), and what instrumentalities we perceive through. 

Questions of the technology of knowing must be interwoven with political and economic considerations
(within the confines of what is scientifically and technologically possible), which is to say
knowledge systems are structured like intelligence and counter-intelligence systems. There is to be
written a whole history of secret and coded knowledges\ldots\ priestly systems, rites, hierarchies and
ceremonies of learning and passage, memory systems, networks of initiates\ldots\ In addition, one should
ask: why did one set of systems triumph---that is to say, why were they preserved, and
remembered---and others fail? There is room for a history of the politics of the promotion, funding
and triumph of intellectual knowledge systems and this includes the rememberance of the major
streams of philosophy. Philosophy is one of the atmospheric backgrounds which provides for a general
and unified state of perception against which day to day knowledge is learned. 

The original choices for computers, binary, Boolean (Leibnizian, as Wiener would have it) logic,
reflected a dialectical, even a Manichean approach and was an unfortunate decision. Why these
choices? It was easier to design electrical circuits that could carry out the logic operations. 

The system began slowly, went on line massively with mainframes and minis in the fifties, mostly in
defense and intelligence applications, followed closely by banking and business. 

In the seventies, a massive campaign was mounted to \enquote{democratize} the computer. The micro
was developed by small, innovative businessmen-technicians. Sales propaganda was disseminated in the
name of enlightenment, efficiency, transcendance and power. Every possible sales technique known to
public relations, advertising and mythology was employed to sell the computer. Not only were ancient
and modern symbols deployed, but also fear. It became possible, we were told, to have a computer in
the home that was once as large as a building\ldots\ and did the same work. 

One notes the parallel developments and \enquote{needs}: The committment to the Great Theater of
perpetual war as the pressure system out of which innovation and invention and progress came. This
generated a need for a vast corps of mind-workers. Cheap education produced intellectuals. This led
not only to the further proliferation of mindworkers, but of mediators and mediational systems.
Intelligence and police (and their surveillance systems); psychologists and their theories; many
schools of psychotherapy; sociologists; anthropologists; analysts; coders and decoders;
cryptographers and decryption experts; disinformation-propagating operatives; advertisers;
public-relations flacks; consultants; historians in fifty modes; economists, both practical and
theoretical; financial manipulators, and the buyers of their services (bankers, securities dealers,
brokers, currency dialecticians); new critics; hermeneuticists; structuralists; semioticians;
deconstructionists; quantifiers; metricians; statisticians; propagandists; accountants and auditors;
lawyers and proliferators of law; interactivists (and their connecting machineries); cosmic and
microcosmic theoreticians; agronomists; doctors; philosophical logicians and inventors of yet newer
and newer mathematics; salesmen; priests and ministers and inventors of yet-new religions; logical
and scientific astrologers\ldots\ And now, in the present age, all this to be machined. 

They sought both unity and fragmentation. Now one must admit that there is a propensity in some
humans to generate new unifying theories and technologies while at the same time inventing and
proliferating new explanatory systems and new subtheories\ldots\ all of which promise to explain
everything. This seems to be a function of the density of intellectuals, in terms of availability of
jobs and competition, both relative and absolute, to a general and non-theorizing population. This
insures that a fair percentage of those theories will be nonsensical, if not fraudulent\ldots\ which is
no impediment to their triumph. 

In addition, general systems theory took hold, and every aspect of the universe was designated a
sub-system of some larger system and the largest---and unknown---system of all was a function of
these bureaucratically minded spinners of holisms. 

The early cyberneticians thought that this development would add to---if not exponentially, then at
least incrementally---the sum of human knowledge. Accompanying this development was an ancient
agenda: the compulsion to impose order, predictability, to eliminate risk and uncertainty. But as
far as this ancient agenda was concerned, the commitment should be shared, paid for by some part of
the public. New processes would in turn create still newer knowledge. And, as all things happen in
this modern society, the \enquote{system,} with all of its attendant confusions, complexities and
corruptions, with its intense conflicts among the different programs, systems and equipment
manufacturers, with its political and business battles, has been laid on in the most haphazard,
ridiculous, expensive, inefficient and disorganized way (repeating our earlier history of canals,
railroads, highways, transit systems, communications and technology in general). We now have a
conflict of computer, communicating and language-conversion systems with many fundamental problems
still unsolved. 


(And here, lest we forget that the problem is not merely \enquote{intellectual,} we must remember
concrete institutions with which intellectuals are connected, and who provide their funding. How,
and to whom, ideas are sold: we must think about AT\&T, Sperry-Rand, IT\&T, IBM, Citicorp and
Chase\ldots\ We must also not forget that there are unwritten and true histories to be done of the
Department of Defense, the National Security Agency, the CIA, all intelligence agencies of the
world, and how the intellectual thought of these agencies permeates every aspect of everyday life.
We must think about the politics of international and national communications policy and how these
issues are fought out in corporations, legislative bodies and regulatory agencies. We must think of
pricing, advertising, marketing, promotion, generations of faulty computers, paper computers,
imbecilic competiton, suppression of innovation, influence-peddling, lobbying, bribes, kickbacks and
the rest of the common paraphenalia of business\ldots\ especially at a time when business becomes
ever-more \enquote{intellectualized.}) 

There was a nescessity to translate all living and non-living forms, to simulate events and natural
processes, to chart their interactions and simulate thse interrelations and to begin to fill the
memory and data banks. This growing assemblage gradually becomes the total environment\ldots\ at least
for a few. These developments are new but are also, at the same time, the fulfillment of an ancient
desire: to control the material world by the manipulation of secret know]ledge (secret, in modern
times, by being priced, being made into intellectual property, being classified). How does this
differ from the practices of ancient priests, shamans, magicians? 

Ancient magicians thought they could control the environment. How did information control the
material world in the past? By assuming a connection between the internal system of intellectual
order and the \emph{external} system of \emph{material} order. One controlled the cosmos by the uses
of resonances and dissonances, rhythms compatible with the true natural rhythm of the spheres, by
the use of a chant, an incantation, a dance, a ritual; or one could apply sacred geometry,
controlling shapes that were analogous to the shape of the worlds one wanted to dominate\ldots\
magic. Magic embodies a primitive theory of electromagnetism and telecommunication. Magic desires to
achieve telepathy and teleportation. Voodoo, for instance, contains the notion of a communicating
medium and the communicants who believe in it. The Catholic Church is a communicating organism with
an apparatus of switches and relays and a communicating language for the input of prayers through a
churchly switchboard up to Heaven, and outputs returned to the supplicant. And above all, all
ancient and primitive systems implicitly propose the notion of an ordered, coherant universe,
expressible in a certain set of languages, the manipulation of which manipulates the universe. The
question is: do these systems manipulate the universe or a simulation of the universe? What certain
intellectuals in modern society propose is electromagic. 

