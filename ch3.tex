\chapter{}

Beginning, perhaps, in the 17\ts{th} century, a few had embarked on a program of "modernizing' society; shattering old categories and languages while inventing new ones. Leibniz, for instance, dreamed of a logical\slash mathematical-based universal language. One of the great agendas of the 18\ts{th} and 19\ts{th} centuries was a vast program of reclassification. There was also an attempt to trace back all modern languages to a primal Indo-European tongue. Past and present humans, societies, languages, plants and animals were arranged on a progressive scale (and this was a continuation of the Renaissance, which had introjected the past, ancient Greece, into its program of liberation from medieval thinking). New theories and new disciplines emerged: economics, politics, psychology, sociology, history, the physical sciences, mythology, anthropology\ldots\ all split off from philosophy. These new disciplines began to atomize and reconstruct, emphasizing quantification. They were partial fictions and suffered all the difficulties of translation; each developed their own jargons, hard and soft tools, aesthetics, formal modes of organizing the perception of the world, creating new mediating lenses between humans, and between humans and the natural world. In time, each one of these disciplines claimed to be a total world view \ldots\ as did each mitotic sub-discipline. General systems theory and interdisciplinary studies began to emerge in the early twentieth century. Now, in the tail end of the 20\ts{th} century, are remelted into the general category of information and communication theory.

The information age required a vast new enterprise: an enormous translation or conversion project; a reduction of all disciplines into a kind of symbolic, quantified representation --- a new universal language which would translate the languages, dialects and jargons of all languages and disciplines --- appropriate to the basic circuit logics in the computers. Bit by bit the differences between disciplines and disparate bodies of knowledge (as well as living and non-living bodies considered as language) are becoming eroded. This endeavor implied a perhaps fictional notion; that the universe and everything in it is logico-mathematical. It also implied that all things and forces in the universe could be treated as a cryptogram, a code, a text that could be \emph{read}, sooner or later. Another and muted implication was that all things in the universe were in some sense \emph{perceptually} simultaneous.

The general computer-compatible\slash general systems-schema runs something like this:
\begin{enumerate}
	\item Anything (or anyone) that can be exactly specified can be automated.
	\item Inferential, judgemental, learned or adaptive behavior can be specified (which raises the problem of translation or conversion of knowledge to information).
	\item Intuitional and creative activity can be indistinguishably simulated by machine (the drive for artificial intelligence).
	\item All this can be communicated from machine to machine, for the speeds of transmission means that messages are distance-insensitive (relatively speaking).
	\item Which means that one has to deal with complexes of social sets and the way in which they, or the information they have, or that represents them (not the same thing) can be communicated.
	\item Information is passed among (or taken from, or imposed on) the sets (but they also frequently resist this passage or appropriation of knowledge about themselves: this implies hierarchy of information systems).
	\item The forces which produce stability inside these social sets create instability among the sets.
	\item From the point of view of the general systematizers, an improvement between and among all social sets (and the way they interpret themselves and the world\ldots\ or the way in which they are interpretable) leads to a better management of the metasystem's information.
\end{enumerate}

But from whose point of view?

By the \enquote{social set} we mean a population which has a language, a mode of discourse and a set of customs (by which the language it uses is processed) existing in a variety of domains or environments, using sets and subsets of natural and artificial languages; bureacracies, corporations, secret societies, individuals, professional societies, classes (in the social sense), ethnicities and races, disciplines, nations, regions, hierarchies\ldots\ and so forth\ldots\ in whatever ways society has been split, conceptually and actually. These, of course, overlap. It is apparent that for all these groupings, the means for universal discourse hasn't been invented yet and what's more, many resist translation actively. All the propositions point directly at the problem of translation, or the generation of a universal language.

Systems-building has gone on since the beginning of the appearance of humans. Even the most \enquote{primitive} of groupings builds all-encompassing (and complexly muddled) systems. Underlying this newest global climacteric, this vast re-writing program, was a not particularly new set of assumptions: that any set of things, events, forces linking people and events could be represented by some language, or set of languages, logics, numbers, letters, symbols, signs\ldots\ That there is an ultimate and fundamental language, a deep structure in the universe \ldots\ and that it is mathematicological and is discoverable and translatable\ldots\ These representations could be linked in several ways: language to language and language to the world represented by these languages ... into interactive and mobile structures that in some way match, dance in time to the underlying and fundamental language of the universe (automated natural language translation is a disaster). When things and people move, the signs representing their existence are communicated to this informational technosphere. Conversely, when signs, symbols, language elements, variables of all sorts are moved, people, things, whole economies, the universe and all that is in it, should move. This manipulator's dream is possible only if information is connected to the universe in some concrete way, requiring sensors, languages, translators, categories and levers.

The sensors (eyes, ears, skin, writers of books, typists, telescopes, microscopes, electronic sensors of all kinds ... and so forth) must \enquote{read}, transmit and input these signs of movement into some kind of storage where language could work on them (meaning the incredible complex of miniscule and high speed movement in the circuits, in and out of the various logical devices and timers and storages ... ). Contrariwise, a set of language-motivated output levers could, theoretically, energize and change the configuration of the universe. (In quantum, operationally-oriented physics, this interventionary notion, that mere thought and its instrumentalities affect the universe---in yet unmeasurable ways---is implicit. By extension, mere thought affects the universe, but in as yet unmeasurable ways.) This desire reflects an ancient obsession; the Archimedean dream of minimal expenditures of energy moving great masses, for example shifting the great nebula in Andromeda into a better orbit.

All of these desires occasions the search for the universal system-langauge which is, at the same time, the \emph{real} language of the universe, the ultimate \enquote{machine code.} A recourse to what can be considered gnostic wisdom, Pythagoreanism, or Kaballism: these are used as key words to exemplify a way of thinking. Pythagoreanism was both a mathematical and a magical system. Number translates into space and converely; all is number and geometry. Kaballism and gnosticism are fundamentally literary. Cartesian thinking carried this obsession further, turning space into a vast, suburban real-estate development. Kaballa views the universe as \enquote{word} (although \enquote{word} translates into number games: \emph{Gematria}). Considered from the perspective of these ancient magical systems: gnosticism, hermeticism, the religion of the Jains, the I-Ching, Rosicrucianism, alchemy and astrology, all the material universe is translatable. But this is to throw a net of language out into the universe, and is the precursor, perhaps, to quantum physics and operational indeterminacy.

None of this denies the need for the creation of language but points toward a recurrant obsession with language as the ultimate reality. What drives this obsession?

The hunt for the ultimate, sacred, or secular, usable, transmissible knowledge or information is like a vision of, a penetration to a sacred realm where total, instantaneous, universal and all-purpose code, exists. This kind of thinking assumes an underlying, unified universe. To match this universe, somewhere, somehow, there exists, and is decodeable (if only in visions and dreams) an underlying language, an ultimate metalanguage, a deep structure of grammar, a boss language of all boss languages to match that reality. And that meta-language is basically mathematical, logical, rational. \enquote{In The Beginning was The Word and The Word was made Flesh.}

(In anticipation, let's propose several such languages: the language of genetics, the language of quantum-relativistic particle physics, the language of finance, the language of mathematical logic, the language of literature \ldots which includes the psychoanalytical disciplines. There are more.)

However out of this uniform ur-language has come Babel. That is to say pure Word, pure light (ultimate information), being made into Flesh, yielded corruption, decay, dialects, death, a plethora of languages. Or, from the evolutionary geneticist's point of view, diversity, uniqueness, adaptability to material conditions, non-repeatability \ldots \emph{quality}. Diversity is the way to disorder, chaos, entropy, a confusion of languages. Specialized knowledges divide into languages, sub-languages, jargons (even putting it this way assumes primal unity); specializations fragment futher the possible wholeness into cultures, sexes, nations, races\ldots Truly unlike languages have unlike assumptions behind them; they cannot translate. This arises out of observing the bewildering array of languages, proliferated into mutually incomprehensible dialects, a diversity of natural languages, in a short period of time. But worse, the presence of, the possibility of incompatible languages intimates that the underlying \emph{physical} universe is not one. Again, chaos. These \enquote{mutations} puzzle and anger the unifiers. They take steps to correct the process of decay and dissolution. (As the first Rockefeller put it: \enquote{Combination is the wave of the future,} in more ways than one.)

If there is a fundamental oneness in the universe---particles seen one way, waves another---and all things, people and events, all singular events are manifested out of this fundament, then all change and process and people are mere aspects, illusion, perhaps developed over time. This \emph{ur}ness should have its own language, shouldn't it? Why this diversity? Or perhaps this \emph{ur}-language existed before the beginning of time? What happened to fragment pre-temporal paradise (which will become balanced at the other end by post-temporal paradise), this primal oneness, this \emph{quality}, leading to the shattering of Eden into cosmic Babel? The Fall? Sin? Imbalance?

It is to offset the effects of a modern Babel that languages are constantly converted into one another, hopefully without losing anything. Work into COBOL; COBOL into Aramaic; Hebrew into digital \ldots and so forth. The universe and all that is in it is assumed to contain a secret code or cryptogram; the new language project, this drive toward fusion is designed to unlock the code (or perhaps its purpose is to \emph{invent} the code). Strings of biochemicals, DNA and RNA, are called a code or cryptogram. The code expresses and replicates to duplicate itself (with mutational and combinatorial variations) and becomes a body which eats food, converts it into energy to give growth to another body-shell in order to perpetuate a code, a language, a message. The body as a message transmitter.

The wisdom of the east handles this problem another way: it announces that diversity is an illusion. The west, at present, holds that the clue to the ultimate bottom language is supposed to lie in the human mind: in some way this primal unity can be \emph{remembered}.

(Remember Lucifer's tale before Eden was built? Consider the role of knowledge in the tale of Eden. There is a struggle over the possession of information and thus a fight to control the sacred language, or Lucifer's tale before Eden was built. Satan becomes the Lord of Diversity, the Lord of disinformation, disunity, chaos, entropy: Prometheus becomes the lord of stolen knowledge.)

Word-obsessed Kaballistic or Gnostic lore anticipates the Big Bang. Many religions anticipate the Fall. The Big Bang leads to entropy: entropy leads to diversity. Humans, we are told, are themselves the creators of negentropy (holding the center together conceptually, if not physically) in the expanding universe, and so they invent unification and then start on the project\ldots The mental act of unification has evolved into a technological and informational endeavour. Diffusion is death; unification is eternal life. 

The division of knowledge into disciplines to pre-conceptually \enquote{observe} society was problematical to begin with. What assumptions were brought to this task? The newest synthesis raises new problems without solving the old ones. In order to achieve this translation, one should look at some underlying assumptions of western, perhaps human, thought. One fundamental tenet of this kind of thought is that one can take wholes, break them down into fundamental units and rebuild all up from those units, providing the structural operations, the \enquote{grammar} to string the units, complexes of units, into whole, new languages. In this newest approach, wholes are broken down into a language of irreducible particles (which are easy to account for, and match up to units that are either measureable and countable, modeleable, mappable, comparable: specifiable) and are built up again.

Now, structuralist thought (which allows one to reorganize the positioning and sequencing of any text and relate it in new ways to any other text \ldots an exercise in simultaneity) and semiotics begins to treat life---including literary and media artifacts--- like a complex cryptogram, a treasure, always oblique, to be disinterred. It should be noted, however, that while this can be done with artifacts, with fictions, with records of the dead, it cannot be done in real life.

One of the first rules of this game of interdictions is that almost nothing is allowed to mean what it first seemed to be. Novelists, priests, poets, mythmakers, magicians, have practiced this combinatorial and sequencing, this matching-up and conversionary pythagoreanism for centuries. Novelists, tied to certain traditions, were permitted to only see a limited set of realities and not others. In much the same way poverty is invisible in the board room, suffering is not a category to be found in an annual report.

But the new constructs in the present contain an accretion from the past (a sort of memory) which is then used to rewrite and reconsider the past. (The act of \emph{primal} creation---and its time, or the \enquote{beginning} of time, and timing---didn't happen \emph{then} but is reinvented again and again, and happens \emph{now}, just as history is rewritten again and again to justify the present in order to assert that all events could only lead up to the inevitable present.)

The newest instance of breakdown and buildup has led to several crises: atomic theory is in trouble. The breakdowns threaten to become endless. \enquote{Fundamental} particles proliferate; gravitons and chronons are invented alongside quarks which require prequarks. Every sub-atomic particle must be specified and recorded, creating firestorms of indeterminacy between all boundaries of thought (and reality), which allows for certain excesses of the imagination, the possibility of new transformations, shapeshifting and chimeras, creating arcane juxtapositions in the life of things, operations which once belonged to the realm of dreams. Unit-quantum thinking contains a history of obsessional perception since Democritus: the universe particularized ... wholes fragmented into quanta... This difficulty is apparent in psychosocial, statistical disciplines which study groups and individuals... atoms and wholes. And yet, at the same time, the universe is percieved as a unified and contiguous whole in which the most distant parts affect one another \ldots sooner or later.

Another complex contradiction to be considered: all could be viewed as the agglomeration of force-fields, electromagnetic waves, gravitational waves, frequencies, which, when they reach some critical density, change into some other \enquote{quality.} (Or geometries, numbers, values, dimensions, symbols, images, gods, spirits, phantoms, cash, talk, drawings, dances \ldots ) One can look at humans as\slash of\slash in these fields in many ways, from many angles, through a variety of disciplinary lenses. Humans, for instance, could be considered as manifestations of the cosmological/astrophysical (and astrological) whole ... brothers, sisters, spawn of the stars; as biological manifestations concreting out of Fourier processes, complex waves inventing complexes of waves in order to explain the self. If so, then medicine based on physics, chemistry, molecular biology should, in time, be replaced by electromagnetic wave therapy ... which is what voodoo is based on.

When the universe is waved, or when a universal language is discovered or invented, the boundaries between objects and objects, and between objects and languages blur. As boundaries were blurred, the disciplineseperated currents of the past dissolved. It became desireable to create the logical links that united the now-unbounded contents of once artificially seperated areas of thought. Indeed, the newest developments in computer thought demand this unity... but in a special way. Language domains can, in principle, be interpolated into any form of discourse, past and present, spatially seperated, including literary and psychological discourse (we consider those psychologies that don't take account of the nervous system to be literary; word games). Now a re-reading and re-critique of all \enquote{great traditions} becomes possible. But, as in all translation, much has to be left out, cast aside as irrelevant or dangeous dross, basically untranslatable (or not desireable to translate). New problems of classification are raised, for one cannot say \enquote{OM} and have it stand for the \emph{All}.

The long and corrective project that some humans are in the process of inventing and reinventing leads to reunification and reconcentration. In the legend, humans are created to replace fallen angels, but they must go through aeons of ‘‘development."' Through a long-range process, this concept mutates into \enquote{evolution.} All of history is a trip toward ascendance and transfiguration, or transsubstantiation: in modern times this reads as a recombinant genetics project for the manufacture of immortal angels. (And in the Golem myth, information placed into dust brings the dust to life.)

How is this fragmentation of languages, of civilization, of energy to be cured? Perhaps by creating the appropriate thesauri, slide rules, categories, classes, conversion matrices for comparative mapping of realm onto realm. A mental act should make it possible to describe n-dimensional hyperspaces in which the languages of, say, poetry, finance, or relativity theory are seen to be one. Recreate a post-ur-language at one of the timeends of the universe? New vision? Not really. Kaballa anticipates these alternate spaces as language-manipulation. Primitive religions describe journeys in folded and short-circuited spaces which can be matched up to the hoard-spaces and passages through which bankers hurl their money around.

If fragmentation is the way to death, then the parts of the person can still be united. How? Resurrection: by the parts being conjoined (keeping track of them, memorializing the whole and its subsequent parts in a file), \emph{communicating} in a magical medium... If not a mystical medium, then possibly an \emph{externalized} nervous system, a great, artificial brain. Who does the connection? In the old days, the witch, the shaman, the priest, The Church, the observor, the remembrancer, the tale-teller. The information establishment plays the modern role. Modern medical information and telecommunication systems may memorialize representations of whole, or parts of people in different and seperate data repositories even in different and distant countries (since the speed of communication is distance-insensitive, and time-insensitive, relatively speaking), ready to join the parts together in the twinkling of an eye. By maintaining communications and preserving a joining-together algorithm, one can create a modern version of the Resurrection through wire or wave transmission, reuniting the parts. The presence maintained after death. Ghosts, of a sort.

Let's consider a practical problem: the operation of the multi-purpose computer in the defense sectors: the Situation-Room. There are various defense and intelligence situation-rooms, and presumably the ultimate and best one in the White House. The purpose of the situation room is to recieve data from all over the world, so that a response to a political or military problem is instantaneously possible. It is a vast intelligence input/output, monitoring device,  % TODO fix
presumably in a form understandable to nonexperts. (After all, for the crisis-data to arrive in machine language, or any other primal computer languages, would be meaningless.) Information arrives from all over the world. Given a change of any variable in the political, military, economic orders, creates changes in the computer: it's a sort of electronic spread-sheet.

In order for this to work a number of conditions must be met. Various sensing systems must be recieving and sending data at all times: a world network of National Security Agency stations monitoring all electronic traffic, decoding and interpreting it, the Central Intelligence Agency recieving a stream of reports, Internal Revenue System, Federal Bureau of Investigation, Political analysis, economic reportage, inputs from the civilian sector \ldots and so forth.

This incredible influx of data must not only be translated, compressed and graded in terms of importance, and matched up to already existant data (to determine significant change), but must be arranged according to some overriding set of scenarios for ready response\ldots scenarios into which go various modes of analysis which had, at some point or other, to be automated, must match up, in computer languages, to those original systems which generated the scenarios in the first place. That is to say, military, economic, agricultural, trade, political, sociological, psychological, anthropological, even medical data, country studies, all, at one time or another, have to be translated into machine-readable forms. And, as much as possible, this influx must be in real time. Of course when one considers the variables, the incredible proliferation of disciplines and their attendant languages which developed before the computer arrived, we see that what constitutes actionable facts are hard to deal with and are imperfectly specifiable, translatable or programmable without enormous distortions.

But another consideration is more practical and that has to do with the question of whether or not individuals, classes, social sets share, or don't share their data. While on the one hand there is a striving for grand unification, at the same time centrifugal forces --- competition, protective secrecy, a proliferation of sub-disciplines --- work in the opposite direction. After all, to take one example, given an age of high taxation, many groupings and individuals have a vested interest in concealing their data.

But, in practical terms, when projected production of oil and its consumption do not conform to electronic wishes and statistical projections of energy companies, then the instrumental Archimedean levers to correct this deplorable situation becomes the Marines, the CIA, the torturers, moving in to fit a preconcieved notion of immediate long-term gain and growth.

Of course, beneath all of this there are larger agendas, whole world-views which are, by nature, metaphysical. Views of human nature (psychological theories), views of nature itself, progressive --- military-assisted --- Hegelian-Calvinistic destiny.
