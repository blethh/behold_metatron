\chapter{}

The limitations of the new information
languages, the limitations of the machines,
storages, operating systems, circuitry, machine-compatible logic, programs diminish
what was once far richer. The old words
were broader; they packed complexes of
implication within them; their ambiguities
allowed for richness and latitude, for rethinking, redefinition from time to time.
They contain treasuries of implication within them; the \enquote{amounts} of information they
contain are staggering. Consider Eliot's \booktitle{The Wasteland}:

How do you represent, in terms of specification, and thus bits of information (if that
is even the way to put it) the endless galaxy of implications contained in Eliot's poem?
In the first place the poem is a rhythmic index, a memory-system. The allusive and
apparently self-contained word or phrase opens up into other poems, and histories.
That is to say they are references to memory
storages. The first line reads \textquote{April is the cruelest month...} 
\emph{All} Aprils included;
the April of Chaucer, the April/Easter of the
crucifixion, \booktitle{The Divine Comedy}, all the
Aprils contained in \booktitle{The Golden Bough}, and
all the rites of spring, sacrifices and renewals
to insure fertility, of dying and being planted
in the earth to spring up in a new form. Are
we to mix our terms here? For example,
political, social, economic conditions, to say
nothing of genetic evolutionary and \enquote{adaptive strategies} 
and continuities to be considered as the stuff of literary concerns\ldots\ at
least not \emph{directly}.\footnote{Although the naturalists had tried to meld science into literature.}
Nor had passion, hatred, character, conflict,
ceremony been allowed to be part of science.
But once we consider the universe to be
language, information, then fiction and
magic permeates it all. When experience and
reality are processed by computer, the usual
domains and disciplines are mixable. Once
discrete realms shatter again, their languages
melt, float, interface. Multi-lateral Pythagoreanism. But then, hadn't this modern fusion
of realms been anticipated by dreams and surrealism,
but in a non-quantified, anti-particleized way?

Agriculture itself as a metaphor for death,
all resurrection, and conversely. In ancient
thought, of course, the spoken and acted-out
rite is a ceremony pre-operational to planting: it is a planning and management 
system. It \emph{precedes} the actuality of material life, as the modern rite of manipulating the memory of past and future wheat crops inherant
in money \emph{precedes} ali spring plantings and
growths.

There is the question of metaphor and
simile, which the computer \emph{cannot} handle
unless given specific instructions, and then
within a limited set of circumstances (which
may grow, but becomes unwieldy). Metaphor
is a form of \emph{fused association}, sometimes of
completely \emph{unrelated} terms (except in the
mind ofa poet, novelist, or advertising copywriter), to create a third term, but in a pecu-
liar way. In the world of program-driven
computers, one list of items may be matched
to another list. What is compared are two or
more strings of \emph{stored-up impulses} (given in
one or another set of computer languages).
Beneath, on the level of circuitry and
machine-language, something different happens than thought. \emph{Items are not being
compared.}

All items in natural language are not
bounded by the compartmentalization required by a computer: they have no true
boundaries. Truly different sets of items
\emph{cannot} be compared unless a tedious and
endless program is written on the order of "if
[\![ this \ldots\ ]\!] then [\![ that \ldots\ ]\!]" 
An algorithm describing the way metaphors are generated can
be easily written; it cannot be implemented
or generated by a computer language. The
instructions could be written such that
\enquote{whenever \enquote{April,} then \enquote{cruel}}, but only
applies to these terms whenever \enquote{April} and
\enquote{cruel} appear. But supposing that another
poet appears, working out of another framework, to speak of April in conjunction with
the autumns of the southern hemisphere?

In order to unlock the poem's meaning,
one already has to have access to a vast
knowledge of literary, religious, anthropological, political, historic, mythological and
psychological compilations (and those edited) in order to summon up the full text of
references. The way the fragments are allusively juxtaposed may be analogized to the
way information is stored, organized and
sequenced on a disc ... in non-sequential
fragments with memory addresses. Well and
good.

It may be said that the retrieval process of
the brain-body is somewhat like the retrieval
process in a computer, but there are significant differences. (That is if anyone knows
what goes on in the brain). The computer was
orginally likened to the brain. The terms
were reversed and the brain was likened to
the computer, leading to ridiculous assesments—tried out experimentally—that the
brain \enquote{processed} information in an on and
off way. Later this was modifed and it was
said that the brain \enquote{parallel-processed.} The
mental processes are associational and quasi-random, and frequently get confused, yielding felicitous mixes. The computer processes
are much more rigid and limited. It is when,
in the poem, the fragments are fused that the
difference becomes more apparent. The
computer cannot fuse associations into a
seamless whole.

Fiction, drama, poetry, non-quantifiable
psychology and other traditional modes of
discourse are partial but stand for wholes;
they have long incorporated complex modes
of organization of people, events, matter into
dramatic sequences ... novels, plays, epics,
poems, psychological theories ... Creating
indices, hierarchies, queues, maps, models,
simulations, translations, sets, classes are
some of the problems raised by information
handling. Each reorganization raises these
problems again and again in new ways.

All literary works contain, among other
things, indices and every such work solves
the problem of hierarchy, or of queueing
without specifically delineating these modes
of sorting as problems. Literaure does not
accept polarities as absolute oppositions.
Dialectics is the emanation of crippled and
self-constrained minds ... the realm, really,
of accountants. Hierarchy? It's all in Dante.
Sets? Borges deals with them wittily. Indexes? See Eliot. In literature (and literary
psychology ... such as Freud's or Jung's) all
these modes are dynamic, allusive, \emph{multireferential.} 
As for set-theory, this is, of
course, the mathematician's and logician's
whimsey. As in life, literature shatters sets.

In real life, all sets are fuzzy. For example;
in a complex, transnational, transtemporal
holding company which owns other companies and parts 
of companies and constantly seeks to conceal itself behind a thousand
portals represented by the shell-names of
companies distributed in a lot of countries,
which is the set of all sets? The subsidiary or
subset may contain (by control) the set of all
sets. In fact, taken all in all, the contiguity of
economic (and political) activity eventually
links and sends the representations of each
part of itself at incredible speeds to every
other economic and political body \emph{in the
world} these days, especially in the age of
advanced telematics. In the modern world,
America, France, the USSR
(since transnationals overlap their boundaries) ... are indeed fuzzy sets, better expressed in literary
language... indeed, the language of \booktitle{Finnegan's Wake}. (And one may say that a set,
called for convenience Mafia, is conjoined
and interpenetrates those sets called corporations and those sets called governments
through other interpretive realms called
politics, by using a set-violating, economic
activity called bribing... It is, to be sure,
primarily and economic organization, but
on the other hand, requires primitive rituals
as part of its sustaining power.)

The way of the fiction writer and mythmaker is a function of long stores of knowledge, arrayed in certain ways, drawing from
a taken-for-granted memory bank. The world of information processing is a world of partial faiths and fictions. Tests for truth that
match reality are meaningless in this world: inner consistency is what counts.

In literature (its application to the new information world will soon emerge) one
converts the experiences of the self and
others into words. (What sensors does one
use to acquire the experience of others?
Through what set of filters --- ideas, technology, legends, myths, psychological and social theories, artificial memory/intelligence --- 
does one sift one's own personal experience?) The writer uses imagery, similes,
symbols, signs, translations, conversions, comparisons,
metaphors, tropes, compact representation, character, emotion, conflict,
drama in certain limited situations. All
dictated by a body of traditions. Experience,
real people, furniture, space, action, geometry, geology, geography are converted into
evocative words and are arranged into some new structure, perhaps
more neatly --- or more amusingly, or startlingly --- than in life.
Fiction, poetry, drama are programs designed
to transmit energy which \emph{amplifies} as it goes
through its \enquote{circuits.} There is selection.
One cannot write about everything.

The compilers and refiners, the preservers and organizers, the abridgers who assembled and trimmed the treasure, the great
canon of literature (surely there was always editing and censorship involved), assert that
their Great creations represent whole populations ... Man and Woman writ large. But
in fact these are \emph{sampling} techniques, which
can provisionally be said to have been invented by the Greek Tragedians. But what is
left out? Among other things, the living stuff
of humans, which involves neuronic, hormonal, enzymatic, chemical, metabolic, genetic, electromagnetic activity had not been
fit for literary language. Yet one could come
up with a prototypical set of biological reactions accompanying (indeed, initiating and
sustaining) mythological, literary and religious prototypes and archetypes.

When Proust bites into the tea-soaked
Madeleine, the taste is a stimulus to releasing memory which then pours out and is
recorded as a printout. We have a report of
Proust's brain operating, but in another language. We could also write a biochemical
essay which replicates the event. We could
also say that Proust wrote a treatise on information storage and retrieval, on the long and
short-term memory, in which the tea-soaked
Madeleine is the key word, and the instruction, that begins the memory dump.

When a marketing study is prepared,
psychology and sociology is used to create
the stimuli, the association of ideas that will
generate the \emph{feeling} which reaches the memory to release biochemicals, electrical charges,
symbols. All emotions, behaviors, dramas
and tastes are tallied, sampled and compared with standardized and representative
beings: prototypes, concurrent archetypes:
the hypostatic buyer. Frequently life-signs are monitored by biological telemetering
equipment. But, at the same time, the marketers must try to turn diverse populations
into Standard Consumers; infecting the archetypal consumer's memory with prototypical
hunger. Persuasion.

\emph{True} archetypes, \emph{original} archetypes, prototypical figures, their deeds and emotions
are communicated down through the ages.
From time to time their attributes are altered
to adjust to newer ages. These are used as
templates to force people into those imageforms. Consider that message called \emph{Oedipus}
and how it has been used again and again.
But were such recorded, aboriginal events
prototypical? Who, what set of people made
them that way and why? If Jung talks about
the collective unconcious, one must not only
ask, \enquote{who collected it?}, but who keeps
reminding us? And what are we not reminded of? Who speaks of the archetypal
revolutionary, the primal guerrilla?

Some set of \enquote{orginal,} transcribed events
and characters are made into standards,
broadcast and rebroadcast. Those who follow
are made to resonate to the original
signal. People through the ages are made
into transcievers. The emotion of a perhaps
once-lived life in turn gains power to motivate people across time and space. In fact,
the people, the transcievers, the relays, are
frequently more emotionally moved by these
compulsive fictions than by their own life or
the people they live among. They are taught
to screen their own experience through
these long-transmitted stereotypes, reassessing their own lives, comparing, matching,
referencing, denying what they are in fact
living out, viewing what is around them
through mediational scrims.

Some humans, fearing death, watch the
butterfly emerge from its chrysalis, or they
see the seed planted, going into the winterdead ground, and grow up as grain, and
dream of the time when they can metamorphose, transubstantiate into angels and gods.
The whole algorithm of agriculture is described in Christian thought, represented by
a human\slash divine figure in its metamorphoses
into divinity. So they invent varieties of
immortality. The way to immortality lies
through death; it begins with a dissolution, a
liquidation and ends with a reconstruction,
recombination, resurrection.

But the progrssive series of deaths and
resurrections lead somewhere. (Not \enquote{true}
death, for energy and matter are conserved.)
If we view the world, the universe in a
quantum-operational sense,
in which the observor intervenes in the observed, then, as
we have said, \enquote{Mind} permeates the universe and must constantly be sustained and
reproduced or the universe will cease to
exist. Thus, \emph{all} the instruments of perception, of past and present, place mind in the
univese and the universe in\slash as\slash of mind. True, the referential frameworks change and
the rules change. Thus dreams permeate through skins, flow through all boundaries
and are shuttled back and forth in history;
empty space becomes a plenum which manifests itself into sometimes virtual entities,
fields of particles and waves, sometimes real
and permanent entities which are viewed
with astonishment. Life is an illusion; the
flow, the dance, the perpetuation and evolution of language is all, and bodies lyse into
language, symbols, matter, space, velocity, energy, bodies, money... (Perhaps humans
resist this liquidity. When humans are converted into economic symbols, one form of
capital, the being is liquified, transsubstantiated into capital. What is capital but a set of
numbers which will metamorphose again into factories or other humans?) The first law
of thermodynamics tells us that matter and energy are conserved. Then the instruments
that view these events are also conserved. That is to say mind is eternal. (Really?) And
perhaps, so is time. And clearly we can see the whole program of the birth, life and
death of the universe speeded up for us in a reconstructive program.
But more, we can now populate our computer with a plethora
of virtual, interconnecting and concurrent spaces and minds, as the Kaballists had proposed so long ago. Perhaps we will see
them, since these fictions will have themselves become instruments of perception.

In the pre-Judeo\slash Christian past, the metamorphoses were circular or cyclical, 
happening without any purposive, long-range
strategy. The Jews and Christians introduced
two complementary --- yet opposing --- long-range strategies; non-repeating, goal-oriented
history, which led to collective immortality,
out of which sprung metaphysical evolution. Events in time were arranged in an
ascending sequence leading to collective
death, transfiguration, resurrection and escape into paradise ... a reuniting with that
from which they had originally been seperated from.

With the advent of the industrial-scientific-technological-capital
revolutions, the transformation programs \emph{appeared} to be secular. Events were arranged into a \emph{non-}Judeo\slash Christian, long range strategy of transcendance. \emph{Qualitative}, metaphysical transubstantiation was replaced by an accretive, recombinatorial, technology-assisted march
toward transubstantiation (since everything
can be prised apart into units, numbered and
rearranged). This invention was called progress, though still incorporating the old notions of immortality. At certain stages, the
accumulation would reach a critical mass
and a quantum leap into a new period would
take place. Permanent revolution. This was a
reaction to that perpetuated, obsessive dream
of a faded paradise, the Roman Empire, and
the descent into Feudal chaos. How many
expulsions from timeless and non-progressive paradisos haunt our \emph{collected memory}?
This kind of thinking led to concepis of
rational forecasting, retrocasting, planning,
management.

But since events had to be scheduled in a
practical, materialistic, exponentially rising
line, in order to gain escape-velocity,
in order to escape entropic fate, it was clear that
man's ascendant journey required a series
of engineered metamorphoses. The mechanisms of natural evolution seemed to be
gone. And anyway, Man wouldn't let any
new form that emerged from him, challenge
him. This meant that the journey toward
immortality required tapping the earth's and
the universe's energy. And if one were to
harness the universe's energy, that would
cost a lot, and anyway, most people, while
fearing death, were not nescessarily interested in immortality: the enterprise had to be
sold. Enter \booktitle{Faust}.

\booktitle{Faust} stands for a key word, one of many
in a directory of transcendance strategies.
\booktitle{The Divine Comedy} is another. \booktitle{Faust} is
dynamic and action-oriented: Dante is static.
Dante (and The Church) encouraged saving
of energy and cutting down of indulgence-expenditure --- called sin --- while Faust encouraged profligacy as a means toward
progress.

One general form of the transcendance
algorithm runs this way: Hero seeks, or is
impelled to seek knowledge; hero has dream,
or dies, or passes into another realm; the
new, or next world is revealed; hero comes
back, or leaves directions in the form of a
book. Note the role of knowledge or information.

\booktitle{Faust} is the great poetic myth representing
the transition from the medieval to the modern age, from medievalism to capitalism,
from agricultural\slash feudal society to industrial
(and then to the information society), from
one kind of magic to another. (And it is
Eliot's lament that we had taken the wrong
path.) \booktitle{Faust}'s ascent is built around one
concrete and one mystic project. The concrete project involves a dam and land-reclaimation. 
\booktitle{Faust} is in every marketing strategy
the computer and software manufacturers
generate. The mystic project is the transubstantiation of, the rejection of the body and
the earthly life, earthly events and time,
history, and mundane procreation, utilizing
a meta-sexual image. \booktitle{Faust}
becomes, as it were, one of the programs of modern society.
The dependance on mystic knowledge, information, is very strong.

