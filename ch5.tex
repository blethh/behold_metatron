\chapter{}

Even the most primitive of tribal entities, those most materially deprived, seem to have extraordinarily complicated and sophisticated intellectual systems. And, along with generating wonderous classification systems, they stack up tales of underlying, fundamental orders, and the creation of these orders, into layered versions. In contrast, modern, western thought is relatively simple and simplifying. 

\emph{Formally} speaking, all these \enquote{pre-modern} systems are very alike: in practice, quite different. The introduction of modern formal analysis may be a particularly western mode for making unlikes alike, a strategy for destroying singularity and quality, still lurking in ultra-modern, western civilization. We see the imposition of this western formalism on the underdeveloped world. On a formal level, these older systems seem to bear a curious resemblance to modern ones. Is this a function of modernist and leveling perception? And all, of course, address themselves to origins and endings.

A cosmic or godly event begins the series. New versions are invented. Beginnings and endings proliferate. Origins are projected and analogized to inconsistent historical and mythological structures, which are then rationalized and united. The Kaballists ask; what happened \emph{before} the Biblical version of creation? The Hindus retroactively add geological strata of explanation and, \emph{literally}, concretized them out of mountains of stone, chiseling the \emph{Kamasutra} of creations: a sexual version. The Mayans and Aztecs choked their cosmos with Gods of retroactive explanation. But, as far as the Judeo\slash Christians were concerned, nothing less than the hunt for \emph{The One} would do, worshipping at the shrine of The Great Unifier and Explicator. But, yet, in entropic time, the cathedrals of nuclear particle and information priesthoods abound with sacred Fundamental Particles and Forces (or operators). Looking backward, the god or culture hero, the calculator-supreme, the order-bringer who will, after a hunt for a mystic vision, will enter the spaces of inspiration and blinding insight: there he will view an inscape, not of earth. After which he will return to deliver the Message of the Oneness of all things.

All information theorists erect their knowledge-processing gods: Prometheus, Lucifer, Thoth, Metatron, Hermes, Hermes Trismegistus, Simon Magus, Giordano Bruno, Jesus with his tidings of great Joy. Now we can look at the latest culture hero in a new light: as \enquote{truth-bringer.} Under the hero's mythic appearance are subsumed physicists, astronomers, molecular biologists, financiers, geneticists, neurologists ... is the five-faced Cyberneticist: Alan Turing, John Von Neuman, Norbert Wiener, Claude Shannon and Noam Chomsky (it should be understood that these names are mere variables for which other names may be substituted).

The great cosmic battle that calls them into being are the first two world wars. The civil war in western civilization is matched up to that great pre-primal civil war in heaven, the aboriginal revolution. This great Initial\slash Final struggle provides the impetus to machine the universe and go into business. Before earth existed, God and Lucifer battled. Earth was created (a rearrangement of space, matter, energy and time) as a tool and a battlefield in the Grand Struggle. The Grand Alliance vs. the Comecon legions of Pandemonium. Before the earth and the galaxies were invented (or reinvented), the Communists sinned against The Light and had their Great Fall from Grace. Or conversely, the Capitalists sinned against the light of Primitive and Paradisical Socialism.

The Delphi where their cybernetic or information thought was born are the Macy Foundation, the Rand, IBM, Bell Telephone, bank wire rooms, the coding, cryptography and signal intelligence systems of government, the military and the universities. Rather than being anything new, this mode of thinking was designed to reorganize and incorporate the long tradition of western rationalizing and simplifying: Greek and Judeo/Christian thought. This scattered and compartmentalized Delphi strove to organize all divisions under its aegis, devouring and engulfing all before it, metabolizing diversity into this new, computational-assisted, gray pap. This assumption, this reclassification of diversity into fundamental order and unity in the universe, all things in it and all interrelationships among them, may, in the long run, be an act of faith ... the propagation of a magic spell to colonize all minds.

Magic depends on a community of belief. The more who believe, the stronger the dominating resonance of the vision, which is then broadcast out into space like an incantation. If, indeed, as the quantum theorists would have it, the observing mind and its prosthetics intrudes into the cosmos, then wouldn't a cacaphony of visions and systems lead to a plethora of spaces? To make sure that chaos is averted, peer review, and ceremonialized, hierarchical orders of permissible discourse allow only for glacial change. Experiments which might prove incontrovertable diversity, or lead to a final skepticism are not funded. Goodness: what if Ernst Mach, who disputed atomic theory, was right?

In modern thought, each version of universal order, developed in history, subsumes and tries to erase past or contradictory versions, negotiating away genuine differences or, at best, converting them into polar opposites or contradictions. But to call something a contradiction is to subsume it. The early efforts of unification included the pantheonizing activities of the Greeks, Romans, Christians, Mohammedans, Hebrews, Buddhists ... who attempted to compress, resolve and rationalize the many gods (who had many attributes and stood for many things), spirits, realms and cultures. Early unified-field theorizing. (And at the same time, intractable organic trees were converted into rows of stone pillars with stone leaves on top to control the wild proliferation of nature and to house this riot of now-domesticated gods.) Not only did the arcane formations of the past lurk on, visible to the eye, palpable to the touch, but in dream-states (since brain activity generates a wild, a surreal associationism, generating metaphors and conceits); the ever-present archeology of past formations collected in social thought (the true artificial intelligence/memory) which was the matrix through which education is sifted. These re-emerge when unifying attempts run into trouble in their encounter with intractable reality.

When order, universality, oneness, falls apart, when classification fails, fallback positions are prepared. Contradictions are invented, the perniciousness of dialectical thought. Human thought (or at least the thought of some subset of humans) seems obsessed with the use of polarities to explain what will not fit; similarities percieved as counter-identities. (Is a lobster the opposite of a human?) All this finds its way into computer thought, based, simply enough, on addition ... mechanistic thinking built on the limited operations that, first logic, then switching devices and logic gates can perform (input-output and feedback devices). For all its complexity, any computer has to use a symbolic logic, which is limited by the control of the flows of electricity. The speed of a vast amount of miniscule operations is mistaken for complexity. The messiness outside this logical world --- whole living ecosystems in wild and wonderous irregular shapes, plants, marine shells, animals, microorganisms, a memory of jungles, sea bottoms, a casual distribution of galaxies --- must be reduced to binaries, Cartesian\slash Leibnizian pixels on an image-processor's screen, or a printout. And this presumably matches the world outside this hermeneutical cave of transistors.\footnote{Computerized image-recognition depends on building up a reference library of simplified, sensor-apprehensible images, compatible with computer recognition.} All this is another way of, as Aristotle put it, holding up the mirror to nature. We see what a few of us, who have designed the sensors, expect us to see; the designers have their preconceptions reflected.

Economic behavior is also obsessed with input-output polarities; systems which account for-and-of capital flows, the \emph{yin} of debts and the \emph{yang} of credits; male gain and female loss; dark and light; the dialectic in which money in a bank is a liability and money out on loan is an asset; the hurling of a profit in Frankfurt to a loss-column stationed in Panama (but both in an electronic balance sheet in, say, New York), or anywhere. All quite Hegelian. Hegelian thought is particularly applicable to accounting systems. In the first place, it is ideal, which is to say that it deals with representations, not actualities. In the second place, implicitly shows progressive and inevitable, even divine growth --- an organic metaphor --- which, when tied to evolutionism, fulfills itself, or God's Purpose, in Time.

Now, the question is: was Hegel the father of the modern, automated balance sheet, or does his thinking derive from double-entry accounting practices? We have been trapped since double-entry bookkeeping and unit-pricing was invented by the ancients. Another question is, how to deal with the unexpected, random, \emph{risk}, and uncertainty? If no one knows what happens \enquote{out there,} but projects: if no one knows what happened \enquote{back then,} but retrojects (doing a long run ... for which a computer is ideally constructed ... if the events can be specified), the past, present and future can be procrusted into a prophet's or risk-analyst's dream. (Consider Joseph in Egypt, whom we will mention again.)

In physics, this long range dialectic concerns itself with entropification, for which quite Manichean Maxwellian Demons were invented to reconcentrate dissolving matter and overcome long term loss. Maxwellian Demons dealt with particle matter, but couldn't deal with the quality of matter; it was a Statistical conceit. Norbert Wiener derided the very notion of \enquote{quality,} which he considered a Medieval hangover. How can you measure or chart a quality? Humans, inventing particle physics (physical biologists) see humans as agglomerations of quanta.

As for business, the non-quantifiable aspects, compulsive behavior, fundamental irrationality, the rites attending successes or failure, the ceremonies of contacts and connections, the accompanying trade in prestige and rank (and frequently women), panoply, display, to say nothing of individual obsessions, trade in contracts, stupidity, shortsightedness, favoritism, structural theivery, fraudulent accounting, altered and destroyed records, computer glitches (and bad design), invented numbers, nepotism, bribery, kickbacks, the giving and receiving of presents, acts of abject faith, all are ever-present and must be factored in ... however indirectly.

These sorts of thinking (particlization, accretion, mutational variation within combinatorial limits, dialectical contradiction, fleshly pythagoreanism, accretive historicism on the pilgrim's progress to transubstantiation and resubstantiation), when applied to the replication of organisms, give rise to the notion of genetic structures as information. This living, quivering biology, this jellyware, comes to be seen as mysterious codes, cryptograms, instructional algorithms for the development of bodies. In sociobiological thought, bodies are mere totality shells, packets, envelopes for transmitting genetic messages along a carrier wave, that becomes embodied from time to time, down the ages. Genetics is not only information, but it is memory. (And, as an instrumentality that interferes in the universe's workings, generates mind which then generates it.)\ Perhaps, at time's end, the messages will reach such a level of accretion and recombination that humans will evolve and transmogrify into angels. (Hence, one burden of this essay is time itself.)
