\chapter{}

At this point in history, the conceptual
and theoretical constructs are distilling and
summarizing the past into programs that
mimic natural and human activities. And
conversely, the rich paper records are being
concealed, secreted away in caves like the
treasure of the Niebelungs. The distillation
remains in databases,
hoarded by large corporations and governments.

More and more the world is seen in terms
of information no matter what the reality is.
Just look at the account books, the numbers,
the projections, the returns. But computerized account books tend toward a sort of
semi-autonomy --- market-to-market, interactively linked --- and drive this outer reality
before it. The investment in computer-compatible thought is so great that more and
more we become trapped in this new culture
and they cannot admit that we have been led
down the wrong fork in history's decision
tree.

If all is fundamentally the same, it follows
that a data base in one language should have
the power to talk to data bases of other
disciplines in other languages (mediated, of
course, by programmers, protocols, translators, modems, computers, networkings...).
One might have to descend into the primal
language and then, choosing the right fork in
the decision tree, emerge into the proper
language. If only one can design the right
protocols, ones that will not only link among
unlike, competetive machines with unlike,
competetive architectures --- IBM's,
Control Datas, Apples, Crays, DEC's --- but also unlike
transmittal systems run by competetive companies. Languages can be united because
each field and domain, each way of looking at things,
should be a subset of the one,
universal, primal language. Perhaps what is
expressed in one domain should be considered as an encryption of what is expressed
in another domain.

However, not only do computers in different disciplines not translate into one
another well, but different manufacturers
and communicating companies (to say nothing of nations), while proclaiming one world,
one language, falling prices, one global village, and universal compatibility, fight one
another tooth and nail. They erect a maze of priced mediations and product differentiation,
countering speed and directness of transmission with profitable labyrinths, in
different time-zones, each turn and gate
tolled and tariffed, competing and maintaining secrecy, organizing those to whom they
sell services on a need-to-know-and-pay
basis, playing the differentials among different states of being, business and knowledge. (Citicorp, for instance, computerizing
and gaining speed, places its headquarters in South Dakota in order to --- taking 
advantage of the laws --- gain advantage which allows it to keep checks for a certain time and
thus enjoy a float in the empyrean.)

There are certain laws to be deduced from the observation of business practice. Information
management, traffic control and
pricing follows the timeless strategy of railroads in the past: which is to say, given a
certain limited distance, the problem becomes to increase distance by increasing price.
Economies of scale are developed, need for certain volumes regardless of content, development risk to be paid for by the consumer.
Tesseracts of tax shelters spring up. An incredible maze of contradictory laws emerge
requiring incredible expenditures of intellectual energy and computing time.
Information theorists always leave out the costs.
Claude Shannon quantified information;
AT\&T and IBM priced it. Shannon's theory did not develop in a vacuum; he did his work
for Defense and Bell. Where did the money come from? What did the funders want and
what did they not want? What other enterprises cross-subsidized these developments?
What solids were melted down, who was liquidated to fund the Great Enterprise? No
different than the practices of the ancient Phoenecians, Babylonians, Greeks, Romans,
Venetians, Fuggers, or any other merchants in history. (In addition, of course, the amounts
of energy, in terms of electricity, required to run and cool computers is staggering.)

If we take into account the human, informal, anti-organizational, shadow-organizational networks,
the person-to-person
contacts, those who emerge to resist this development,
those who have an interest in not
sharing information, we see vast, centrifugal
forces at work. On the one hand, the emergence of a unified system, a sort of electronic
Catholic Church: on the other, a sort of
electrofeudalism.

Given all this potential convertibility,
how can money talk to nuclear particles,
pension funds speak recombinant genetics,
prime numbers retrieve fictional heroes\ldots ?
Can we really create a translation program,
which is to say a unified field theory? Or
should we, not having been invited to the
initial feast of reason, create a \emph{disunified field theory}?

The primal-language business, like the
origins business, is highly competetive (since
the costs of computer runs is much more
than paper experiments). One of our many
ultimate transformational and alchemical
media, a primal liquidity in which all life is
dissolved, reconstituted and redisolved is
genetics. What is the market value of bioengineering as expressed in some form, with
purchases involved, with manufactured products and processes at the end, investible
end-products and investors screaming for
their dividends, trying to hurry time up?
Will it cost the world's savings to transform
humans and will we be left with one creature
at the end?

We raise the same questions about particle-wave physics and its ruinously expensive paraphenalia.
Finance, literature, genetics, nuclear physics: four (of many) primal languages; three
media in which translations from realm to realm can be seen as new versions of progressive metamorphoses.

