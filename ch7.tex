\chapter{}

So we begin again from another angle, using one of the most highly computerized of modern entities: the corporation. We will talk for a while about entities, borders (or skins), sets, organisms, time-series, populations and genetics.

We live in a time when corporate operations have become speeded up. They have become mobile: they waltz across the sea: they take wing and fly into the night: they grow slender and slip through 30 gigabyte-wide needle's eyes. They attack and pillage one another. They split into pieces and recoalesce. They live in this age and other ages. Metanational entities gobble up chunks of their national hosts. Nothing new about that: what is new is the informational paraphenalia and the high-speed, distance-insensitive, time-devouring equipment which pressages a vast political change.

Corporations are --- at least legally, and metaphorically --- seperate entities. At the same time they are not. They are linked into para/meta/supra/sub-networks by investors, interlocking consortia, joint ventures, cartels... They defy the notion of discrete sets. A corporation has a boundary only for purposes of classification and identification. (By the same token, one can also say that conceptual imprisonment denies the human their individuality.)

A corporation is defined as a living being in the \emph{contemplation} of the Law. A corporation is diverse. The contemplating Law is also not only diverse, but dynamic and changeable, requiring lawyers, plaintiffs, litigants, defendants, judges and legal memory banks (precedents) assembled through the ages. (Law too is being placed in data banks, where into a system called Lexis.) a corporation spans nations, And it is contemplated by several systems of law at the same time. Are we expected to believe that the emanation of a complex of people, events and memories, the law, can \emph{contemplate} the abstracted emanation of a complex of people, events and memorigs, a corporation? That is to say a fiction, or anthology, or novel, contemplates another fiction, another anthology, another novel. A sort of organic \emph{character}? We need hunt no further. Here, truly, is artificial intelligence. Why spend any more R\&D money?

The corporation can be represented as information in a pure, financial form, which is a slice of the composite life of lives and works in progress, taken at some point in time. (Financial institutions, especially banks, are considered here because of all business, the financial entity is the most heavily intellectualized, the most heavily dependent on computers and communications. The technology is inextricably bound up with the value-flows, the calculations being fundamentally simple.) While appearing abstract, ideal, it is neither ideal nor platonic, nor is it static. Its positions can be caught in an account sheet, but not in its motion. Its motion can be caught, but the positions are lost. Sound familiar? Its existence now requires the constant intervention of humans and machines, managing, evaluating, working, integrating it into a market structure. It includes several methodological histories; the evolution of the notion of evaluation and the history of its accretion of value. This is to say humans and machines producing, trading, buying and selling money or near-money. If it is to have life, it must have lives to keep on working. Its abstract operations have concrete results that drive the lives of the people inside of it and outside of it.

With the advent of the high-speed calculator, robots, and programs (based on long statistical runs of past performances, accretions of admonitory history-scenarios and tortured equations) which mimic some financial transactions, such trading and instantaneous as automated communication programs to link buyers and sellers into an electronic market, it becomes possible to concieve of a pure, automated and constantly adjusted financial corporation, one totally devoid of humans, territories, factories...It would do all the things financial organizations do: move money around, trade, arbitrage, take deposits, account, merge, acquire, make long and short term loans, divest, invest, liquidate, grow, collect, lobby for laws, pay or dodge taxes, contemplate risk and probability, now and then order an assasination, all while living in exotic, anaerobic climes.

If there is no plant in the physical sense, there have to be virtual \emph{plants}, as-if factories, represented on paper or in computers, stored in data banks, to guide bankers in their moves, driven by modes of evaluation, lists of people and institutions to borrow from or loan to, risk studies about good and bad investments (prophetic programs; Joseph scenarios), market-switching-and-routing programs for heaving pools of money this way and that.

But to be meaningful, this fictional being must be connected to other markets and other forms of endeavor in real time, a something, somewhere in the universe to connect to, someplace to enter inputs, a someone, or set of someones or the simulation of someones to credit these messages from this totally automated financial institution. Most of all, such an ideal financial being, in order to exist, must be credible: that is, accepted as a motivating act of faith, by other institutions, other beings, and Law. This has not happened yet. As we talk about corporations, we still detect humans ... somewhere, if only living in palaces. It still needs humans to interact. Humans to be affected by these interactions. Although one can see a time when some automated complex of corporations consume what some automated complex of corporations produce.

We talk about the life of the corporation in several ways, as we talk about the life of an organism. Humans are stockholders, directors, traders, officers, consumers, workers. All who contribute to the life and existence of this contemplated being have histories, both social and biological. They may also be defined as a gene-pool, although differing from a race. This population is the result of diversified reproduction as against non-diversified reproduction (a tribe, clan, race, ethnicity, nation; those with subsets of shared genes). There is, to be sure, such a thing as a family corporation, a form that bridges the gap between a dynasty and shareholder \enquote{democracy.}) Thus one can say that the corporation has a sort of genetic structure, one composed of sets of genetic structures, each set composed of parts of other sets, defined or ensetted not so much by family or race consanguinity as by their participation in the enterprise. Indirectly, this corporate organism has, so to speak, a genome, which is to say a complete genetic constitution. At the same time its participants invest, give life to, guide the destinies of other such entities which have different genomes, for one must spread risk and diversify.

Another way to think about the corporation is from the perspective of investments (stock, bonds, etc.), plus other rules for life, governance and growth (even replication). This combination of investments and rules can be considered as the genes of the corporate, fictional organism. These metagenes are said to express themselves into living organisms. Genes, information of living beings express themselves into other living beings and also into metagentic forms: capital, which has its own rules of continuity and metamorphosis.

Investments shapeshift into material life. Here we have begun to introduce the concept of modern significant demographies in which contiguity, as well as continuity is provided by investment, participation and modern communication. Replication does not require face-to-face existence. And so the question of space and time is raised again. Offshore receptacles with names, await; Panama is said to have at least one hundred thousand corporations of all sorts. Some are real. Some are shells, mere names of fabulous beasts that are inspirited and informed by a shower of electronic gold.

It is possible to perform a mind experiment. Consider the genes, in their informa- tional aspect, of all who interact with the corporation, and search for something in common between genetics and corporate life, belongingness, in turn related to the equity\slash debt\slash assets expressed as stock, directorship, management, etc. This new set of translations relates to the organism's two totalities and is exemplified in the annual report, the balance sheet, which is a slice of life, frozen in time, but as transtemporally allusive to past and future histories as, say, Eliot's time-meditations in \booktitle{The Four Quartets}.

This being's genome can be seen as the result of a long stretch of accreted historic information (the story of the buildup of equity and credit) which can, through a series of reconstitutions (as one reads past organisms, arranged in a historic sequence, in the genetic memory of any being), be remodeled backwards on to life, production and reproduction of gods and humans. The \emph{names} of those past and present humans, the lives as they live, replicate in parallel, can be mapped to the corporation's replication pattern. Or we can go forward again from these humans to dissolve them, their essences, into this legal fiction, this corporate being. It's a simple mapping problem; genetic information encrypting corporate information and conversely. Once again; we should not forget that both sets of information involve human activity, or at least the \emph{still-living memory} of human activity. Considering the age-old continuity of some dynastic fortunes, this is not too arcane to believe. After all, the corporation is a metaphor embodying real human existence. It is a chimera.

Corporate mergers are referred to in sexual terms; marriage, even rape. Are these mere words, anthropomorphisms used to describe a phenomenon too complex for words? Shorthand? Key words, which when decrypted, open up a vision of vast legal and credit data banks containing huge record depositories of swiftly shifting law, money, reports, memos, accountings, as well as histories of the mixing of complex social groupings, populations, evolving, revolving through a variety of forms? An analogy?

There is a problem here: are we to take a metaphor seriously? Is it truly descriptive of a complex reality? No, but... The keyword always becomes an intrinsic contentual part of the phenomenon descibed; it is wrenched loose with the greatest difficulty. What do lawyers and accountants argue about? They argue about \emph{words} and \emph{semantics}. They structure, deconstruct, semiotisize, mine the law-bodies for hermeneutic nuggets: they do \emph{literary} criticism and linguistic analysis. They invent new, post-dated critical theory.

And if we are talking about fictions, try to think of what the Oedipus Complex describes without the word \enquote{Oedipus} to unlock the memory bank. Try to think of the merging, through a marriage, involving two individuals, representative of two complex dynasties (corporate entities) and fortunes (bloodlines or genes; property and treasure); royal families. Whenever we are talking of corporate or dynastic marriage, in terms of informational essences, we are describing an alchemical wedding, a marriage of heroes, gods, archeiypal figures, and enterprise. But what is it that is mated? People? Yes. But also information: bonds, representations, money, deeds, abstractions symbols (stock, for land) which are generalized information, not specific to any corporations, resolvable into things --- people, factories --- as the genetic information is potentially resolvable into bodies of living beings, races, once properly expressed.

A vast practical and ceremonial apparatus is required to bring alchemical essences together. A vast, practical, biological and cere- monial apparatus is required to bring the genetic essences of any two people together. When it comes to the marriage of dynasties or fortunes, ceremonial behavior increases (to say nothing about a great to-do about contracts). Mating dances. (Stock in \emph{this} company or \emph{that} company allows for certain kinds celebratory rites and participations. When stock is loaned against currency, that currency is potentially part of \emph{any} corporation, but only when traded for new securities with their particular, limited set of behavioral instructions.)

If a human is a combination of two halved gene-sets, a kind of information-bearing and organism-producing program, then a corporate merger is a multi-sexual, abstract orgy. It requires dozens, thousands of essence-sets to conjoin, to participate, to be transferred in order to materialize into another kind of existence. Certainly as long as human activity goes on, as long as computerized and abstraction-activity goes on, as long as it is recognized in the contemplation of law and the faith of people, this wonderous being lives. Go tell someone that IBM does not live.

Do analogies of corporate marriage incorrectly define this entity? While we wait --- probably forever\footnote{Sorry, Sol. --- S.W. Editorial} --- for pure, artificial intelligence to come on line and carry on the purified and transcendent sum of knowledge, making new, autonomous decisions, we can say: no humans, no corporations. No human reproduction, birth, death? No corporations. The corporations may not actually mate, but mating and reproduction must go on somewhere. Is this any sillier than saying, as the sociobiologists do, that the body is nature's way of producing more DNA? The exchanges, assets, liabilities, accretions, all the other signs of operations in the corporation's informational sphere can be said to be determinants of social behavior inside and outside (wherever its influence reaches) this being. It's individuals are just as caught up in a sort of fate as poor Oedipus (a dynastic drama) was. Considering the rhetoric; invisible religions, acts of abject faith, superstition, lurk beneath the most rational, mathematical and scientific works. Magic continues to shape human behavior.

In ancient mythology there were organisms that had lion's heads, wings, human bodies, snake's tails, the heads of hawks and owls... We could in principle make a genetic map of such a being. A modern, diversified transnational being is, of course, infinitely more complex and amorphous, for have an organism composed of living we things, dead things, and the rememberances of things dead and past, no less phantasmagoric than those ancient sphinxes, chimeras, minotaurs and hydras.
