\chapter{}

Let's backtrack to wherever it is that the Origins of Western Civilization are reinvented and then go forward to try a little progressive meta-history of this most modern of organisms. Greece, a tiny time-slice inserted far down, towards the beginning of the ascent-trajectory of western Man. Our art, philosophy, logic are said to begin here. What if all memory of Greece were erased?

All countries, cultures and religions secrete, value and store up selective histories and mythologies to legitimate themselves. What do Europe and America consider themselves to be without this passing nod to Greek origins? Even Hitler's and Mussolini's regimes featured Greek-derived iconography. To make these histories, an amalgam of events, myths, legends, folk tales, sagas, religious deeds, epics (to say nothing of vast storehouses of paper, as well as appropriate mnemonic architecture and monuments) were collated. Thus a quasi-arbitrary sequence is created, one linking events into causal chains in order to create the feel of mystical inevitability, celebrating the everconstant triumph of the present. Mistakes or horrors are leached out or are considered to be the price one pays for progress. Continuity, progress, history is a smoothing out of the sudden, the disruptive, the violent, the random and unaccounted-for. Even the most rationalistic sequences must include the cultural, for logic too is a \emph{culture}.

The building up of Western Civilization is a story of grand strides toward unity, yet requiring breakdowns and fragmentation of old formations: that is to say, re-feudalization before new reorganization (note the re-feudalization of AT\&T). This totalizing sequence constitutes a grand myth --- given an ascendant trajectory which will avoid breakdowns --- called progress or sometimes evolution --- beginning in the past few hundred years when some percieved that the industrial revolution had to rewrite the old myths into new ones, forever ending circles and cycles and introducing exponential curves reaching into the empyrean (sometimes called space). To question this accretion at any point is to disrupt the spell the Grand Ascent has over us.

It may be argued that certain events were chosen arbitrarily; an association of ideas about event sequences, which happen to have --- we have not accounted for pure invention --- operated in a certain temporal sequence, not nescessarily in a cause-effect line. A politico-mythico-Lockian mnemonic. Ultimately G\"{o}del makes the point that the linking of all logical steps depends on an act of faith, just as the assignment of meaning to a pool of securities is an act of faith.

We can follow some of the many arbitrary feeder streams that empty into this ocean of Western Culture. Oedipus, for example, is not merely a Greek play about a hubris-ridden, stubborn figure, one prototypical individual, blinded and suffering. It is the story of the restoration of balanced ethical and divine budgets. The tale is also about a power struggle, a plot in the face of an environmental disaster: medical (plague), economic (starvation), genetic (dynastic legitimation: though it is not specifically mentioned as such, genetic information as fate --- as explained by the retroactive prophecy delivered by Delphi --- are intertwined), and demographic catastrophe (sterility in women and potential population decline). The tale is also about an information-search (among other things for genetic origins). Who passed ownership and title on to whom and how, based on breeding lines of descent, a story of false and contaminated claimants. It is about the relationship of genetic purity to rulership, property, the laws that define those relationships, the rules of mating. The health of the nation is at stake.

Usually this play is interpreted for us as a moral, psychological and sexual tragedy (incest). Freud led us in the wrong direction, positing the erotic rather than the reproductive consequence, falling into a trap set by the old Hebrews and Christians. Lust, especially incest, leads to death. But then, we may ask, what did Freud know about life? What little he did know, he lied about.

Genes, contaminated by \emph{deeds} (acquired characteristics, or fate?), the inadvertant sin of Oedipus and Jocasta have brought together the wrong genes, illegitimately. The sins of the rulers, or putative founders of this kingdom of Thebes (the information) contaminates the body politic, its well being, its health, and affects all nature. Did the Greeks know about genetics? No, but they were obsessed with plant and animal breeding and dynastic breeding strategies. The forced comparison between human activity and nature's response is interpreted in a magical way and skews human behavior. (That the play is also Athenian propaganda against Thebes and Delphi is another story in itself.) If the thought of the pre-Socratics, Socrates, Plato and Aristotle go into modern rational thought, why not these tales, which are \emph{logical} constructs disguised as dreamlike tales of scandals.

Another take: Helen of Troy gives birth to the Hellenes. We can now understand one of the reasons for the furor over Troy. Only Helen can give birth to the Hellenes and she---or the symbolic and real reproductive apparatus inside her beautiful body---has been stolen. Helen, mythic figure, like a queen ant or bee, is a collective emanation. As a carrier of a certain nation or race-spawning mechanism, she is, from an information-perspective, also collective. \booktitle{The Iliad} concerns a conflict of ethnicities, a polarity called Asia vs. Europe, an East-West struggle for the control of trading routes to the profitable Scythian hinterland. The war is used as a means of uniting diverse tribes into one mega-ethnicity. The first Grand Alliance, the first NATO-like Allies. These themes are summoned up again during the East-West war against the Persians. The Iliad includes a brilliant, albeit indirect, essay on set-theory disguised as a list of tribes. The Hellenes are the set of all Hellenic sets: a genome of a corporate body called a race.

The Old Testament can be seen as a set of linked stories and myths about the Hebraic relationship to God's Design (covenant, contract, law ... rules of the game, more honored in the breach than the observation), a history of dynastic continuities; rules for the preservation of the Hebrew genepool and its royal subsets: the living propagation of God's word, or perhaps the Word made into code...God provides the operating system; the Hebrews write the software). The Design has rules for mating, accretion of power, and how to transmit \emph{informational} treasure (\emph{The Talmud}). The Old Testament is also a meditation on the rules of economic behavior, negotiation of conflicts, law, contracts, politics, nations.

These notions of a nation encompasses a genetic mystique. An ethnicity contains the idea of birth from a set of primal founders, all of whose descendants have shares in a commonality of genes and are related.

Implicit in the Old Testament and Judaic law (and fulfilled in Kaballa) is a sort of evolutionary trajectory along a linear path. The history of Jews, unlike the history of \enquote{primitives} is not cyclical except in the longest sense. Gratification and fulfillment are deferred. The Hebrews introduce the notion of the long-range trajectory, completed by the coming of the Grand Recombinatorial and debt-redeeming wizard, The Messiah. This Messiah will reunite the scattered Hebrews (for how can they mate if they are far apart?). And \emph{this} Messiah is only for the Jews, only for \emph{one} genepool, no one else. Kaballistic lore, a heretical meditation on the same long-range cycle, is a wave with only one fall and rise, spread throughout the universe. The Messiah becomes a way of reuniting, reconcentrating the power that the disaspora fragmented, a kind of ultimate meta-history of mergers.

Still another event: What strategy did Joseph use to monopolize Egypt's grain production for the Pharoah? Prophecy, dream interpretation, fear, land-reform, expropriation of commodities, rationalization of production (a kind of early, political, state-run agribusiness), storage facilities, new gathering techniques. Clearly Joseph must have used some form of accounting and econometric projection to affect a grand, structural and political change. His interpretation of Pharoah's dream is risk-analysis. Not only did Joseph change the environment around him, but his thought radiates down through the ages and affects agribusiness today. The short-range, forced march collectivization in the Soviet Union, the long range \enquote{collectivization} and concentration into agribusiness in the U.S. (using genetic cropping strategies), use the same pattern. Maybe we should call Jung's idea the collectivized memory.

Up to the New Testament, the Hebrews are heterosexual. The Jews \enquote{give birth} to the Christians, or that's the way we tell the story. God uses Mary as a child-producing\slash nurturing vehicle (expropriating variant tales out of the past). Here is a departure into androgeny: self-replication, an ancient, sinful theme, for this is what Lucifer did too. Incest? Autocest? Autogamy? Or is God a wierd insect? New rules for sexual mating enter the picture and implicitly alters the relationship of dynasty to property. By proslytyzing \emph{all}, not only Jews, the nation, the ethnicity dissolves and becomes a sort of corporation: shares in redemption are open to all. The Messiah will redeem \emph{all} debt. The Old Testament is endogamic: the New Testament is exogamic.

Historiography is expressed as reproduction, human continuity with attendant \enquote{qualities}; genetics as myth. But does the myth enter into the genetics of the present, as many scientists turn back to what was religion and magic in the past? (One may pour enormous resources into a project and use scientific techniques in pursuit of magical aims.) The Christian God's experiment enthralls the moderns and they try to replicate it, using the Holy Spirit to fertilize a laboratory vagina.

Genetics and the right to rule, to own property, come for the Romans at a time when both myth and documentation exist side by side inside the dynasty-obsessed empire. The Empire funds the history of its own origins, a vast, dynastic myth called \booktitle{The Aeniad}, which traces the ancestry of the Romans to the Trojans, not the Greeks.

For the Christians and the Romans, these tales explain discontinuity while maintaining continuity. The new birth stories insure discontinuity between the Romans and the Greeks. In the same sense, Jesus is not only divine, but a Jew and not a Jew. Aeneas is a Greek, but not a Greek.

At time's end, the Christians promise -- theme and variation -- Resurrection and redemption of all believers (investors). They will bring to material life the dead and dissolved who remain as memories. But how is this Ressurection to be accomplished? Will there be an information-search for the dissolved and scattered, the transformed, decayed human material of the world, and then a sort of reconstruction, a retro-combination? Clearly, the parts, the very atoms, have memory addresses. Or are the \emph{memory} elements to be reinfused into dust, and that dust brought to life, as the Kaballists claimed to be able to do? Or are the memory elements inserted in machines? No matter. The point is that the \emph{thinking} behind these tales descends to us. (Of another transubstantiation myth which has to do with capital itself, more later.)

These political and social materials, this kind of thought, woven back into the whole traditional memory-corpus of dramatic, fictional, religious works, permits us to see that a trans-disciplinary whole operated even in the ancient past, albeit using coded languages. These were preserved, by some, with consequences for the present. On the other hand, what was selectively forgotten and buried? We never get to hear the peasant's side of the story in the Joseph tale. The oppressed have no dynastic history. The concreteness, the mundanity of the past, has been generalized, its particularity eroded, just like the start-up equity put into a corpoation. The information of the past shapes the information processing of the present.

The econometric predictions of this present (think of the Russian-American wheat deal of 1972-73, the decision to corner the grain market, extensive planning) uses, with minor variation, the Josephian scenario. In fact, to tell the story again, the disaster Joseph predicted was not a disaster of underproduction, but too many years of overproduction, with attendant deressed prices, requiring either a famine to be manufactured, or at least an \emph{informational} famine created by cornering the market, leading to real hunger. Prices? Is this really in \booktitle{The Bible}? Joseph, after all, as Pharoah's agent, sold the surplus grain on the world market during a world famine. New discourses inserted back into past events disrupt the holiness of the memory time-series, and question the legitimacy of modern thought-buildup, indeed, the legitimacy of all present-based\slash obsessed-with-past, the soft cause-effect linkages built into history.
