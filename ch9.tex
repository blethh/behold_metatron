\chapter{}

Why raise these questions? To challenge an obsessional mode of thought which annunciates itself as new and seems to become more rational every day, but which is a capital intensive, ghost-haunted complex, stealing thought and memory away to hoard it. To attack a long and endless, even boring set of preoccupations; theme and variation on a few original musical phrases; a casting out, an obliteration of cacaphonies. But do the original themes tell us anything of real social behavior of the message-senders, the people who generated them? Can we really seperate passion from ideas? Even the ambition to succeed can distort ideas. After all, how do the preoccupations and passions (to say nothing of the confrontation with, or the bowing to power) enter the idea-stream? After all, people have killed one another to preserve their ideas.

What would happen if we altered the memory-references to, say, the ancient Greeks, and their culture (sculpture, philosophy, logic, myth, history)? What if we included the practical, day-to-day thought and considerations of power? This disruption of the sequence, described as a progressive and ascendant evolutionary movement from the simple to the complex would ripple back up again to the present, causing glitches and noise in the pipelines.

When Marx said that the philosophers had only talked, never done anything, he was wrong. They indeed did something, for the propagation of messages requires a climate of belief which they generated. Marx himself refused to let go of any of this past, merely rewriting the perspective.

Consider: Socrates never answers Thrasymachus satisfactorily, in that order-obsessed schema, \booktitle{The Republic}. If philosophy is a sequence, each succeeder building on each preceder, then philosophy never got off the ground. We still wait for a response to that question: power is justice. What followed is nonsense. On the other hand, the merchants and politicians, the soldiers listened carefully to Thrasymachus.

And does it mean anything at all that the great themes sounded by the Athenian playwrights and philsophers, and upon which the great symphony of western thought is composed, were all homosexuals, but nevertheless required to mate with women and replicate? Is there a hidden content, a secret sexual message in philosophy, a movement toward body-purified thought? This has bearing on the question of heterosexual reproduction, the desire to escape the tyranny of Grand Design-serving matings. A homosexual population generally doesn't replicate; it must recruit. Will it put artificial reproduction on the agenda? Do dreams of non-heterosexual reproduction lead to designs for immortality and eternal youth...a longing for transcendance, a covert desire to escape the decayable body?

In both \booktitle{The Divine Comedy} and \booktitle{Faust} there is a seeking to find a route to escape the bodily and heterosexual mode of reproduction, to seperate the erotic from the reproductive, which begins to lead to algeny. Faust turns his back on earthly marriage and love, to mate with a \enquote{female} principle in heaven, seeking and using knowledge and deeds in his journey. Dante glimpses Paradiso, seeing shining intelligence and bodiless love. As light\slash love\slash intelligence radiates into space, it sinks into the blackness of body, matter and sin. Divine love and knowledge are a metaphor sent down the ages. Alchemical wisdom. But in the past, there were no major idea-and-body-replicating devices to carry down the notions of hermeticists, gnostics, Kaballists, magicians. Heterosexuality was necessary. How to escape this trap? Later, John Von Neuman dreams of machines, gathering knowledge, becoming autonomous, reproducing.

To understand the present one must look over this long series of \emph{regularized} events, saying in each case, \textquote{this time is like that time} or \textquote{this time is not like that time.} To regularize events is to force similarities on these happenings. This is possible only if time-segments are everywhen the same. It is disconcerting to think that each individual in history was unique, singular, unrepeated and unrepeatable. How can you retrieve their thought? How can you resurrect?

Modern, rational thought requires even greater precision, since we think, and replicate thought through this \emph{capital-intensive} mode, one which cannot handle true singularities. Order, repeatability, similarity, pattern, structure, identity is introduced into past sequences, otherwise how can there be such a thing as a series? Consider the search for missing links (a medieval, logical technique left over from the invention of the Great Chain of Being) in order to smooth out the series of evolution, to eliminate great and cataclysmic jumps. If we allow these ruptures, we leave room for the reintroduction of the divine, the inexplicable, wild and truly random chance. Evolutionary frauds, counterfeit artifacts, faked documents are manufactured and inserted in the attempt to create legitimizing historical series where there were none before, or at least no record of any. How much easier this is to do with the computer as it projects, constructs, simulates to fill the unfillable gaps. If we cannot think without the aid of the computer, then thought itself is capitalized.

There are several kinds of capital (or value) here. One involves the accretion of knowledge. Another can be likened to the yet-to-be-valued good will of an ongoing enterprise. Good will is an intangible, but it can be quantified and entered into the account books of an enterprise, then bought and sold. A third kind is more mundane: energy ingathered and stored up as wealth, credit, money, which is information. A fourth kind is the good antecedants, the precedants, the legitimizing \enquote{genes.} A fifth kind is structure; the orderly arrangement of timed events into a sequence of inevitability: critical-path operations, program evaluation and review technique, scheduling ... one kind of chronology as against other and disruptive chronologies. All together, they constitute the strategic program for running the modern enterprise, the extended \booktitle{Talmud} of western civilization.

Whatever really happened in \booktitle{Oedipus} and in the Joseph story is suppressed. The making of the sequence requires rewriting the elements long after the facts, if there were any. These tales are supposed to be exemplary, instructional, fragments of an algorithm. The enterprise must be saved from disruptive thought, from noise, and requires, first and foremost, the storage of a transmittable message. It is in that sense that this sampling of literary\slash mythic works mentioned here are an intrinsic part of the good will of this capital-intensive enterprise called Western Civilization.

Now, the chief operating executive in Thebes, a subsidiary of a diversified banking and religious consortium called Delphi (also a data bank and intelligence-gathering operation) was called a king. Oedipus. The king has broken the rules; he's an anamoly. In order to prevent the enterprise's demise, Oedipus must be deposed, the management changed to demonstrate that the enterprise continues under the aegis of fate (what \enquote{appears} to be an inevitable series) rather than individuls. The board of directors meets at the shrine of Delphi; it is they who plan Oedipus's deposal, implying retroactively that not only was it fated, but the cause of the crisis lay in Oedipus's very genes (as later Eliot will re-sound this theme, using the key word, \enquote{Tiresias,} to express the barrenness and sterility of modern life). The board are kingmakers. At the same time, this ruling elite, in order to restructure the trajectory, also plan a meta-demise, a long-range, exemplary message which will be transmitted down the ages, a program-scenario for the sacrifice of kings, managers, chief executive officers to maintain order and sequence. This is to be celebrated thousands of times.

What if \emph{all} the characters in \booktitle{Oedipus} were disgusting, then why bother to preserve that memory? But in fact, that's what they were: greedy, grasping, selfish, monstrous, in \emph{no} way noble, and in that sense archetypical.

If we are dealing with no more than a revisionist, mythic history of a political struggle, then the classical Freudian interpretation --- indeed the whole Freudian industry, which needs one interpretation and not others --- loses this stored-up good will. The credibility of one of the foundations of our enterprise goes down the drain. In one version, Oedipus is a hero; Jocasta is not his mother. In another we see the story of a coup against Oedipus. In a third, it is Jocasta who ordered Laius slain. In a fourth we see that indiscriminate sexual behavior, including casual incest, is everyday behavior in royal families. In a fifth we see the play as the celebration of the coup from the point of view of Athens; the defeat of Delphi and the assumption of control by Athens as it struggles to establish hegemony. Ina sixth, we see Oedipus trying to replicate himself through incest. If Oedipus is not exemplary, then good will is devalued. Oedipus could just as soon be Boss Tweed.

One could make a communications flowchart of these paramemories, trace how this good will was transmitted down through the ages by humans who acted as recievers, repeaters, relays, enhancers, gates, transformers, noise-eliminators, interpolaters, adders, switches, coders and decoders, error-correctors. They recieved and sent these stories down by voice, in writing, or by use of rites, chants, liturgies, ceremonies, dramas, storing them in any variety of devices. From time to time they were retrieved to be used as guidelines to correct present and future behavior. At the same time, humans themselves, as biological creatures, also transmitted a different kind of information and memory: genetic messages. Alongside these two streams, treasures, credit, good will, capital was sent. The memory of a memory: one, events; the second, biology; the third, capital. Built in was an adjustable, timesequencing rescheduler for calender reform. None of the tracks can exist without the other.

All forms of knowledge intertwine to pass down a climate of opinion, a meta-environment, which becomes part of the present percieved \emph{physical} environment. Indeed, as futurists --- equipment-sellers all --- talk about the next evolutionary step into the information age --- which is also a whole environment --- and annunciate this adaptation to this new climate, \emph{property} seems to become less physical and more ephemeral. The burden of our argument becomes clearer: inheritance.

(For example: the one subject, the true, corporate --- or embodied --- \enquote{hero} of all of Dickens's works is Inheritance. His characters are always involved with claims on Inheritance. Inheritance as meta-genetics, manifests itself into shells called humans, or characters. The role of Dickensian characters is to move Inheritance through history. Each one of Dickens's novels involves a search for a programming error, which, when corrected, allows for the continuity of inheritance outside of the fates of the characters involved. To use inheritance self-indulgently is to descend into sin.)

The newest version of these old inheritance stories is sociobiology, a kind of biodeterministic Calvinism. Into the observation of nature are inserted these birth-mythologies and breeding-and-replication logics of ancient Israel, Greece, Egypt, Rome, and the consanguinity-obsessed Middle Ages. The carrying down of these treasures, saved from the ruins of shattered civilizations, finds its way into modern myths of adaptability and evolution, even within the present twenty year span. To this theme is added inevitable causality.

As an asset to Western Civilization, what kinds of valuation can be placed on these long gone events? How can they be calculated into the asset picture? One must begin by reviewing, assessing, quantifying and valuing these intangibles, these pools of good will. How do these carry-forwards contribute to the development of rational calculation in its newest, computer-assisted modes, translated into assets and liabilities? What distortions, mythologies, religious superstitions have crept in and how did they get there? Or, since everything can be informationalized (if specified) and assigned a currency value, does it matter?
